\documentclass[12pt,a4paper]{article}
\usepackage[utf8]{inputenc}
\usepackage{amsmath}
\usepackage{amsfonts}
\usepackage{amssymb}
\usepackage{graphicx}
\usepackage[left=2cm, right=2cm, top=2cm, bottom=2cm]{geometry}
\usepackage{hyperref}
\usepackage{booktabs} % For nicer tables
\usepackage{longtable} % For tables that span multiple pages
\usepackage{enumitem} % For more control over lists
\usepackage{xcolor} % For colored text if needed
\usepackage{listings} % For R code
\usepackage{float}
\documentclass[12pt,a4paper]{article}
\usepackage[utf8]{inputenc}
\usepackage{amsmath}
\amsfonts
\usepackage{amssymb}
\usepackage{graphicx}
\usepackage[left=2cm, right=2cm, top=2cm, bottom=2cm]{geometry}
\usepackage{hyperref}
\usepackage{booktabs} % For nicer tables
\usepackage{longtable} % For tables that span multiple pages
\usepackage{enumitem} % For more control over lists
\usepackage{xcolor} % For colored text if needed
\usepackage{listings} % For R code
\usepackage{float} % For better figure placement control

\newcommand{\Rpackage}[1]{\texttt{#1}} % Command for R packages
\newcommand{\Rfunction}[1]{\texttt{#1()}} % Command for R functions
\newcommand{\Robject}[1]{\texttt{#1}} % Command for R objects/variables
\newcommand{\Rcode}[1]{\texttt{#1}} % For inline R code


\hypersetup{
    colorlinks=true,
    linkcolor=black,
    filecolor=magenta,      
    urlcolor=cyan,
    pdftitle={BAN404 Exam Companion},
    pdfpagemode=FullScreen,
}

\lstdefinestyle{Rstyle}{
    language=R,
    basicstyle=\ttfamily\footnotesize,
    keywordstyle=\color{blue},
    commentstyle=\color{green!40!black},
    stringstyle=\color{purple},
    showstringspaces=false,
    breaklines=true,
    frame=single,
    numbers=left,
    numberstyle=\tiny\color{gray}
}
\lstset{style=Rstyle}


\title{BAN404 Statistical Learning - Exam Companion}
\author{Brede N. Espelid}
\date{Spring 2025}

\begin{document}
\maketitle
\tableofcontents
\newpage

\section{General R Setup and Tips}
\subsection{Commonly Used Libraries}
\begin{itemize}
    \item \texttt{ISLR} or \texttt{ISLR2}: For datasets from the book.
    \item \texttt{boot}: For bootstrap functions (\texttt{boot()}, \texttt{cv.glm()}).
    \item \texttt{glmnet}: For Ridge and LASSO (\texttt{glmnet()}, \texttt{cv.glmnet()}).
    \item \texttt{tree}: For classification and regression trees (\texttt{tree()}).
    \item \texttt{randomForest}: For Random Forests (\texttt{randomForest()}, \texttt{varImpPlot()}).
    \item \texttt{gbm}: For Gradient Boosting Machines (\texttt{gbm()}).
    \item \texttt{gam}: For Generalized Additive Models (\texttt{gam()}, \texttt{s()}, \texttt{lo()}).
    \item \texttt{e1071}: For Support Vector Machines (\texttt{svm()}).
    \item \texttt{MASS}: For LDA, QDA (\texttt{lda()}, \texttt{qda()}).
    \item \texttt{tidyverse} (or individual packages like \texttt{dplyr}, \texttt{ggplot2}): For data manipulation and plotting.
    \item \texttt{fastDummies}: For creating dummy variables (\texttt{dummy\_cols()}).
    \item \texttt{insuranceData}: For \texttt{dataCar} dataset (Exam 2021).
    \item \texttt{Ecdat}: For \texttt{Computers} dataset (Exam 2022).
\end{itemize}

\subsection{Key R Operations}
\begin{itemize}
    \item Setting seed: \texttt{set.seed(number)} for reproducibility.
    \item Splitting data (50/50 example):
\begin{lstlisting}
n <- nrow(mydata)
set.seed(123)
train_indices <- sample(1:n, floor(n/2))
train_data <- mydata[train_indices, ]
test_data <- mydata[-train_indices, ]
\end{lstlisting}
    \item Converting to factors: \texttt{mydata\$variable <- as.factor(mydata\$variable)}
    \item Model fitting:
    \begin{itemize}
        \item Linear Regression: \texttt{lm(y \string~ x1 + x2, data=train_data)}
        \item Logistic Regression: \texttt{glm(y ~ x1 + x2, data=train_data, family=binomial)}
        \item Prediction: \texttt{predict(model_object, newdata=test_data, type="response")} (type="response" for probabilities in logistic)
    \end{itemize}
    \item Evaluation:
    \begin{itemize}
        \item MSE: \texttt{mean((predictions - actuals)\textasciicircum{}2)}
        \item Confusion Matrix: \texttt{table(predicted_class, actual_class)}
        \item Accuracy: \texttt{sum(diag(conf_matrix)) / sum(conf_matrix)}
        \item Proportion table for confusion matrix: \texttt{prop.table(conf_matrix, margin=1)} (row proportions)
    \end{itemize}
    \item LASSO/Ridge: Remember to create a model matrix (e.g., \texttt{x <- model.matrix(y ~ . -1, data=train_data)}) and a response vector \texttt{y <- train_data\$y}. Factor variables need to be converted to dummies first if not handled by \texttt{model.matrix} as desired.
\end{itemize}

\section{Lecture Summaries and Key Concepts (ISLR)}

\subsection{Lecture 1: Introduction to Statistical Learning }
    \subsubsection{Defining Business Analytics}
        \begin{itemize}
            \item \textit{Analytics (Davenport \& Harris, 2007)}: "the extensive use of data, statistical and quantitative analysis, explanatory and predictive models and fact based management to drive decisions and actions."
            \item \textit{Business Analytics}: Applying analytics in the context of firms making the best possible decisions efficiently using available data.
        \end{itemize}

    \subsubsection{Categories of Business Analytics }
        \begin{itemize}
            \item \textbf{Descriptive Analytics}: Understanding past and present through data exploration (e.g., salary structures). Not model-based, not prescriptive.
            \item \textbf{Predictive Analytics}: Forecasting future events or unknown properties (e.g., stock prices, probability of tax evasion). This course primarily focuses here.
            \item \textbf{Prescriptive Analytics}: Recommending actions based on optimization or statistical models (e.g., resource allocation, best medication choice).
        \end{itemize}

    \subsubsection{Essential Components for Business Analytics Problems }
        \begin{itemize}
            \item Management Science / Operations Research: Formulating decision problems from business questions.
            \item Data Analysis / Statistics: Choosing data and methods to estimate/predict quantities of interest.
            \item Programming / Database Management: Implementing solutions.
        \end{itemize}

    \subsubsection{Core Concepts in Statistical Learning }
        \begin{itemize}
            \item Definition: A vast set of tools for understanding data.
            \item Example: Improving sales based on advertising budgets (TV, Radio, Newspaper).
                \begin{itemize}
                    \item Inputs (Predictors, Independent Variables, Features): $X_1, X_2, ..., X_p$.
                    \item Output (Response, Dependent Variable): $Y$.
                \end{itemize}
            \item The Data Generating Process (DGP): $Y = f(X) + \epsilon$
                \begin{itemize}
                    \item $f$: Systematic information that $X$ provides about $Y$. Unknown.
                    \item $\epsilon$: Random error. Zero mean, independent of $X$. Represents irreducible error.
                \end{itemize}
        \end{itemize}

    \subsubsection{Goals of Estimating $f$: Prediction vs. Inference }
        \begin{itemize}
            \item \textbf{Prediction Focus}:
                \begin{itemize}
                    \item Goal: Estimate $Y$ using $\hat{Y} = \hat{f}(X)$.
                    \item Accuracy depends on reducible and irreducible error.
                    \item \textit{Reducible error}: Error from $\hat{f}$ not being a perfect estimate of $f$. Can be reduced with better models.
                    \item \textit{Irreducible error}: Due to $\epsilon$. Inherent variability/noise.
                \end{itemize}
            \item \textbf{Inference Focus}:
                \begin{itemize}
                    \item Goal: Understand the relationship between $X$ and $Y$.
                    \item Key questions: Which predictors are associated? Strength of association? Functional form of $f$?
                \end{itemize}
        \end{itemize}

    \subsubsection{Types of Learning}
        \begin{itemize}
            \item \textbf{Supervised Learning}: Both predictors $X_i$ and response $Y_i$ are observed.
                \begin{itemize}
                    \item Regression Problems: $Y$ is quantitative (e.g., price, blood pressure).
                    \item Classification Problems: $Y$ is qualitative/categorical (e.g., disease/no disease, purchase/no purchase).
                \end{itemize}
            \item \textbf{Unsupervised Learning}: Only predictors $X_i$ are observed.
                \begin{itemize}
                    \item Goal: Discover structure (e.g., clustering, PCA).
                \end{itemize}
        \end{itemize}

    \subsubsection{Assessing Model Accuracy }
        \begin{itemize}
            \item \textbf{Measuring Quality of Fit - Regression}:
                \begin{itemize}
                    \item \textit{Mean Squared Error (MSE)}: $MSE_{Train} = \text{avg}(y_i - \hat{f}(x_i))^2$.
                    \item Goal is to minimize $MSE_{Test} = E(Y_0 - \hat{f}(x_0))^2$ for new unseen data $(x_0, Y_0)$.
                \end{itemize}
            \item \textbf{Measuring Quality of Fit - Classification}:
                \begin{itemize}
                    \item \textit{Training Error Rate}: $\frac{1}{n} \sum_{i=1}^{n} I(y_i \neq \hat{y}_i)$.
                    \item \textit{Test Error Rate}: $\text{Avg}(I(y_0 \neq \hat{y}_0))$ for new unseen data.
                    \item \textit{Bayes Classifier}: A theoretical ideal that minimizes test error rate. Assigns to class $k$ for which $P(Y=k|X=x_0)$ is largest.
                    \item \textit{K-Nearest Neighbors (KNN) Classifier}: Practical method, estimates $P(Y=k|X=x_0)$ based on fraction of $K$ nearest neighbors belonging to class $k$.
                \end{itemize}
        \end{itemize}

    \subsubsection{The Bias-Variance Trade-Off }
        \begin{itemize}
            \item For a given $x_0$, the expected test MSE: $E(y_0 - \hat{f}(x_0))^2 = \text{Var}(\hat{f}(x_0)) + [\text{Bias}(\hat{f}(x_0))]^2 + \text{Var}(\epsilon)$.
            \item \textbf{Bias}: Error from simplifying a complex real-world problem. More flexible models typically have lower bias.
            \item \textbf{Variance}: How much $\hat{f}$ would change with different training data. More flexible models typically have higher variance.
 
            \item \textbf{Overfitting vs. Underfitting}:
                 \begin{itemize}
                    \item \textit{Underfitting}: High bias, model too simple (e.g., linear model for non-linear data). Low training and test performance typically.
                    \item \textit{Overfitting}: Low bias (on training data), high variance. Model fits training noise. Low training error, high test error. (Slides p. 17-22 illustrate with polynomial regression on limited training data, showing poor generalization to test data).
                 \end{itemize}
        \end{itemize}

    \subsubsection{The Curse of Dimensionality }
        \begin{itemize}
            \item Phenomenon: Test error tends to increase as dimensionality ($p$) grows, unless the additional predictors are truly associated with the response.
            \item Reason: In high dimensions, data becomes sparse. Nearest neighbors can be very far away, making local methods (like KNN) less effective.
            \item Volume of space increases exponentially: $s^p$. To maintain density, $n$ must grow exponentially.
            \item Most points in a high-dimensional hypercube are close to the boundary. (R code `cube.R`, `hypercube.R`, `edges.R`).
\begin{lstlisting}[caption={R code: Fraction of points near boundary (edges.R concept)}]
GetRatio <- function(dim){
  1 - (0.9)^dim # (1 - 20.05)^dim is volume of inner hypercube
}
p_vals <- 1:10 # For demonstration, ISLR uses up to p=10 for nearest neighbor distance plots
ratios <- sapply(p_vals, GetRatio)
plot(p_vals, ratios, type="b", xlab="Number of Dimensions (p)",
     ylab="Fraction of Volume Near Boundary (within 10% of range)",
     main="Curse of Dimensionality: Volume Concentration")
abline(h=0.5, col="red", lty=2) # Shows when over 50% of volume is "near" boundary
text(5, 0.4, "p=5, ~41% near boundary", pos=4)
text(10, 0.6, "p=10, ~65% near boundary", pos=4)
\end{lstlisting}
        \end{itemize}

    \subsubsection{Understanding Loss Functions }
        \begin{itemize}
            \item Squared Error Loss: $L(Y, \hat{f}(X)) = (Y - \hat{f}(X))^2$. Minimizing expected squared error leads to prediction $\hat{f}(x) = E(Y|X=x)$.
                \begin{itemize}
                    \item Unconditional case (predict with constant $\theta$): $\theta_{opt} = E(Y)$. Sample estimate $\bar{y}$.
                \end{itemize}
            \item Absolute Error Loss: $L(Y, \hat{f}(X)) = |Y - \hat{f}(X)|$. Minimizing expected absolute error leads to prediction $\hat{f}(x) = \text{median}(Y|X=x)$.
                \begin{itemize}
                    \item Unconditional case: $\theta_{opt} = \text{median}(Y)$. Sample estimate is sample median.
                \end{itemize}
            \item Implication: The "best" prediction can change depending on how we define "loss" or "error."
        \end{itemize}

    \subsubsection{Lab: Introduction to R }
        \begin{itemize}
            \item Basic commands, graphics, data indexing, loading data, summaries.
            \item Relevant files: `ch2-lab.R` (from GitHub), `ex2.8.Rmd` (Canvas).
        \end{itemize}

    \subsubsection{Exercise: ISLR 2.8 (College Data)}
        \begin{itemize}
            \item Application of descriptive statistics and basic plotting in R.
            \item See `ex2.8.Rmd` (Canvas) or `ch2-applied.R` (GitHub) for solution ideas.
        \end{itemize

\subsection{Lecture 2: Linear Regression and K-Nearest Neighbors (KNN)}
    \subsubsection{Revisiting Overfitting with Polynomial Regression }
        \begin{itemize}
            \item \textbf{Context}: Example using a small training dataset (10 observations, $x, y$) to fit polynomial models of increasing degrees ($k=1,2,3,4$). The true underlying relationship is quadratic ($k=2$).
            \item \textbf{Training MSE Behavior} (Slides L2 p. 5):
                \begin{itemize}
                    \item Linear model ($k=1$): \Rcode{trMSE1 = 14.37876}
                    \item Quadratic model ($k=2$): \Rcode{trMSE2 = 0.4588453}
                    \item Cubic model ($k=3$): \Rcode{trMSE3 = 0.4320397}
                    \item Quartic model ($k=4$): \Rcode{trMSE4 = 0.4003628}
                    \item \textit{Observation}: Training MSE decreases as model flexibility (polynomial degree) increases. The 4th order polynomial, being most flexible, has the lowest training MSE. This is typical: more flexible models can better fit the noise in the training data.
                \end{itemize}
\begin{lstlisting}[caption={R code for Training MSE (Conceptual - from Slides L2 p.5 values)}, label={lst:trainmse_poly}]
# Assuming y_train are the true training y values
# and reg1, reg2, reg3, reg4 are lm objects for k=1,2,3,4
# fitted ONLY on the first 10 training observations.
# y_pred_reg1 <- predict(reg1, newdata=training_data_first_10)
# trMSE1 <- mean((y_train_first_10 - y_pred_reg1)^2)
# ... similar for reg2, reg3, reg4
\end{lstlisting}
            \item \textbf{Test MSE Behavior} (Slides L2 p. 6-8):
                \begin{itemize}
                    \item The full dataset actually has 20 observations. The first 10 were training, the next 10 are used as a hold-out test set.
                    \item Linear model ($k=1$): \Rcode{teMSE1 = 3522.564} (Significant underfitting)
                    \item Quadratic model ($k=2$): \Rcode{teMSE2 = 0.9843595} (Best test performance, matches true DGP)
                    \item Cubic model ($k=3$): \Rcode{teMSE3 = 173.0981} (Overfitting starts)
                    \item Quartic model ($k=4$): \Rcode{teMSE4 = 8582.551} (Severe overfitting)
                \end{itemize}
            \item \textbf{Visualizing Over/Underfitting} (Slides L2 p. 7): The plot shows the linear model (black) clearly missing the curve. The quadratic model (blue) tracks the full dataset well. The 3rd/4th order polynomial (green, if shown for k=4 on slide 7, or specifically for k=4 on slide 22 of L1 slides) fits the initial 10 points well but deviates wildly for the test points.
            \item \textbf{Conclusion} (Slide L2 p. 9): Models that are too flexible (e.g., high-degree polynomials here) can achieve low training error by fitting noise, but generalize poorly to unseen test data (high test error). The goal is to find a model complexity that balances fitting the signal without fitting the noise, typically leading to the lowest test error.
        \end{itemize}

    \subsubsection{Simple Linear Regression }
        \begin{itemize}
            \item \textbf{Model}: $Y = \beta_0 + \beta_1 X + \epsilon$. $Y$ is response, $X$ is predictor, $\beta_0$ is intercept, $\beta_1$ is slope, $\epsilon$ is error term.
            \item \textbf{Estimating Coefficients using Least Squares}:
                \begin{itemize}
                    \item Objective: Minimize the Residual Sum of Squares (RSS).
                        $$ RSS = \sum_{i=1}^{n} e_i^2 = \sum_{i=1}^{n} (y_i - \hat{y}_i)^2 = \sum_{i=1}^{n} (y_i - (\hat{\beta}_0 + \hat{\beta}_1 x_i))^2 $$
                        (Slide L2 p. 28 shows the minimization objective $\min_{\alpha,\beta} g(\alpha, \beta)$).
                    \item Least Squares Estimates:
                        $$ \hat{\beta}_1 = \frac{\sum_{i=1}^{n} (x_i - \bar{x})(y_i - \bar{y})}{\sum_{i=1}^{n} (x_i - \bar{x})^2} = \frac{S_{xy}}{S_{xx}} $$
                        $$ \hat{\beta}_0 = \bar{y} - \hat{\beta}_1 \bar{x} $$
                    \item R Implementation (\Rpackage{ISLR} \Robject{Auto} data, predict \Robject{mpg} from \Robject{horsepower}):
\begin{lstlisting}[caption={Simple Linear Regression in R (ISLR Ch 3 Lab, Slides L2 p. 29, L2 p.36)}]
# Using Auto dataset from ISLR package
library(ISLR)
attach(Auto) # or use Auto$horsepower, Auto$mpg
lm.fit <- lm(mpg ~ horsepower) # shorthand for lm(mpg ~ horsepower, data=Auto)
summary(lm.fit)
# Output from summary(lm.fit) (similar to Slide L2 p.36 for TV~sales):
# Coefficients:
#              Estimate Std. Error t value Pr(>|t|)    
# (Intercept) 39.935861   0.717499   55.66   <2e-16 
# horsepower  -0.157845   0.006446  -24.49   <2e-16 

plot(horsepower, mpg)
abline(lm.fit, col="red", lwd=3)
# To manually calculate coefficients (Slide L2 p.37 concept for X, y):
# X_design <- cbind(1, horsepower) # Design matrix
# y_response <- mpg
# beta_hat_manual <- solve(t(X_design) %% X_design) %% t(X_design) %% y_response
# print(beta_hat_manual) # Matches lm.fit coefficients
detach(Auto)
\end{lstlisting}
                \end{itemize}
        \end{itemize}

    \subsubsection{Assessing Accuracy of Coefficient Estimates }
        \begin{itemize}
            \item The true relationship is $Y = \beta_0 + \beta_1 X + \epsilon$. $\hat{\beta}_0, \hat{\beta}_1$ are estimates.
            \item \textbf{Standard Error (SE)} of estimates: Measures sampling variability. If we refit the model on different random samples, SE tells us how much $\hat{\beta}_j$ would typically vary.
                \begin{itemize}
                    \item $SE(\hat{\beta}_0)^2 = \sigma^2 \left[ \frac{1}{n} + \frac{\bar{x}^2}{\sum_{i=1}^{n}(x_i - \bar{x})^2} \right]$
                    \item $SE(\hat{\beta}_1)^2 = \frac{\sigma^2}{\sum_{i=1}^{n}(x_i - \bar{x})^2}$, where $\sigma^2 = \text{Var}(\epsilon)$.
                    \item $\sigma^2$ is usually unknown and estimated by $RSE^2 = RSS/(n-2)$ for simple linear regression.
                    \item Distribution of $\hat{\beta}_1$ (Slide L2 p.30, assuming $\epsilon_i \sim N(0, \sigma^2)$ and are independent):
                      $$ \hat{\beta}_1 \sim N\left(\beta_1, \frac{\sigma^2}{\sum(x_i - \bar{x})^2}\right) $$
                \end{itemize}
            \item \textbf{Confidence Intervals (CI)}: A $(1-\alpha)\%$ CI for $\beta_j$ is a range that contains the true $\beta_j$ with $(1-\alpha)\%$ probability in repeated sampling.
                \begin{itemize}
                    \item For $\beta_1$: $\hat{\beta}_1 \pm t_{\alpha/2, n-2} \cdot SE(\hat{\beta}_1)$. (Slide L2 p.31 uses $1.96$ from $z$-dist for large $n$).
                    \item Example from \Rfunction{summary(lm.fit)}: The \Rfunction{confint(lm.fit)} function gives CIs.
                \end{itemize}
            \item \textbf{Hypothesis Testing} for $\beta_1$:
                \begin{itemize}
                    \item $H_0: \beta_1 = 0$ (No linear relationship between $X$ and $Y$).
                    \item $H_1: \beta_1 \neq 0$ (There is a linear relationship).
                    \item Test statistic: $t = \frac{\hat{\beta}_1 - 0}{SE(\hat{\beta}_1)}$. This $t$-statistic follows a $t$-distribution with $n-2$ degrees of freedom under $H_0$.
                    \item p-value: The probability of observing a $|t|$-statistic as large or larger than the one computed, assuming $H_0$ is true. A small p-value (e.g., < 0.05) provides evidence against $H_0$.
                    \item \Rfunction{summary(lm.fit)} output provides $t$-values and p-values (`Pr(>|t|)`) for each coefficient.
                \end{itemize}
        \end{itemize}

    \subsubsection{Assessing Overall Model Accuracy }
        \begin{itemize}
            \item \textbf{Residual Standard Error (RSE)}:
                $$ RSE = \hat{\sigma} = \sqrt{\frac{RSS}{n-p-1}} $$
                (For simple LR, $p=1$, so $n-2$ in denominator. Slide L2 p. 38).
                \begin{itemize}
                    \item An estimate of $\sigma = \text{std.dev.}(\epsilon)$.
                    \item Represents the average amount that the response will deviate from the true regression line.
                    \item Measured in units of $Y$.
                \end{itemize}
            \item \textbf{R-squared ($R^2$)}:
                $$ R^2 = \frac{TSS - RSS}{TSS} = 1 - \frac{RSS}{TSS} $$
                where $TSS = \sum (y_i - \bar{y})^2$ is Total Sum of Squares. (Slide L2 p. 38)
                \begin{itemize}
                    \item Proportion of variance in $Y$ that is explained by $X(s)$.
                    \item $0 \le R^2 \le 1$. Closer to 1 indicates better fit.
                    \item For simple linear regression, $R^2 = (\text{Cor}(X,Y))^2$. (ISLR Ch3 Exercise 3.7 provides a proof for this, which involves algebraic manipulation of RSS, TSS and the formula for correlation).
                    \item $R^2$ always increases or stays the same when more predictors are added, even if they are irrelevant. This makes it less suitable for comparing models with different numbers of predictors.
                \end{itemize}
        \end{itemize}

    \subsubsection{Multiple Linear Regression }
        \begin{itemize}
            \item \textbf{Model}: $Y = \beta_0 + \beta_1 X_1 + \beta_2 X_2 + \dots + \beta_p X_p + \epsilon$.
            \item \textbf{Estimating Coefficients}:
                \begin{itemize}
                    \item Minimize RSS: $\sum_{i=1}^{n} (y_i - (\hat{\beta}_0 + \hat{\beta}_1 x_{i1} + \dots + \hat{\beta}_p x_{ip}))^2$.
                    \item \textbf{Matrix Notation} (Slides L2 p. 33-35):
                        \begin{itemize}
                            \item Response vector $\mathbf{y}$ ($n \times 1$).
                            \item Design Matrix $\mathbf{X}$ ($n \times (p+1)$), first column usually all 1s for intercept.
                            \item Coefficient vector $\boldsymbol{\beta}$ ($(p+1) \times 1$).
                            \item Model: $\mathbf{y} = \mathbf{X}\boldsymbol{\beta} + \boldsymbol{\epsilon}$.
                            \item RSS: $g(\boldsymbol{\beta}) = (\mathbf{y} - \mathbf{X}\boldsymbol{\beta})' (\mathbf{y} - \mathbf{X}\boldsymbol{\beta})$.
                            \item OLS Estimator: $\hat{\boldsymbol{\beta}} = (\mathbf{X}'\mathbf{X})^{-1}\mathbf{X}'\mathbf{y}$.
                            \item Variance-Covariance Matrix of $\hat{\boldsymbol{\beta}}$: $\text{Var}(\hat{\boldsymbol{\beta}}) = \sigma^2 (\mathbf{X}'\mathbf{X})^{-1}$. Diagonal elements are $SE(\hat{\beta}_j)^2$.
                        \end{itemize}
                \end{itemize}
            \item \textbf{Hypothesis Testing in Multiple Regression}:
                \begin{itemize}
                    \item \textbf{For individual coefficients $\beta_j$}: $H_0: \beta_j=0$ (predictor $X_j$ has no effect, holding others constant).
                        \begin{itemize}
                            \item $t$-statistic: $t_j = \frac{\hat{\beta}_j - 0}{SE(\hat{\beta}_j)}$. Follows $t_{n-p-1}$ under $H_0$.
                            \item Provided in \Rfunction{summary(lm.fit)}.
                        \end{itemize}
                    \item \textbf{Overall F-test (Model Significance)}: $H_0: \beta_1 = \beta_2 = \dots = \beta_p = 0$ (none of the predictors explain $Y$).
                        \begin{itemize}
                            \item $F = \frac{(TSS-RSS)/p}{RSS/(n-p-1)} = \frac{MSR}{MSE}$.
                            \item Under $H_0$, $F \sim F_{p, n-p-1}$.
                            \item If $F$ is large (small p-value), reject $H_0$, meaning at least one predictor is useful.
                            \item Provided in \Rfunction{summary(lm.fit)}.
                        \end{itemize}
                \end{itemize}
            \item \textbf{R Implementation} (Example from ISLR Ch3 Lab):
\begin{lstlisting}[caption={Multiple Linear Regression for Boston Housing Data}]
library(MASS) # For Boston dataset
lm.fit.multi <- lm(medv ~ lstat + age, data=Boston)
summary(lm.fit.multi)
lm.fit.all <- lm(medv ~ ., data=Boston) # Using all predictors
summary(lm.fit.all)
\end{lstlisting}
        \end{itemize}

    \subsubsection{Prediction and Confidence/Prediction Intervals }
        \begin{itemize}
            \item Prediction for a new set of predictor values $\mathbf{x}_0$: $\hat{y}_0 = \mathbf{x}_0'\hat{\boldsymbol{\beta}}$.
            \item \textbf{Confidence Interval for Mean Response $E(Y|X=\mathbf{x}_0)$}:
                \begin{itemize}
                    \item Quantifies uncertainty about the average value of $Y$ for a specific $\mathbf{x}_0$.
                    \item $\hat{y}_0 \pm t_{\alpha/2, n-p-1} \cdot SE(\hat{E}(Y|X=\mathbf{x}_0))$.
                    \item For simple LR (Slide L2 p.32): $\hat{Y} \pm 1.96\hat{\sigma} \sqrt{\frac{1}{n} + \frac{(x_0 - \bar{x})^2}{\sum(x_i - \bar{x})^2}}$. The $1.96$ is $z_{0.025}$.
                    \item In R: \Rfunction{predict(lm.fit, newdata=..., interval="confidence")}.
                \end{itemize}
            \item \textbf{Prediction Interval for Individual Response $Y_0$ at $X=\mathbf{x}_0$}:
                \begin{itemize}
                    \item Quantifies uncertainty about a single future observation $Y_0$ for a specific $\mathbf{x}_0$.
                    \item Accounts for both uncertainty in $\hat{\boldsymbol{\beta}}$ and irreducible error $\epsilon$. Always wider than CI.
                    \item $\hat{y}_0 \pm t_{\alpha/2, n-p-1} \cdot \sqrt{RSE^2 + SE(\hat{E}(Y|X=\mathbf{x}_0))^2}$.
                    \item For simple LR (Slide L2 p.32): $\hat{Y} \pm 1.96\hat{\sigma} \sqrt{1 + \frac{1}{n} + \frac{(x_0 - \bar{x})^2}{\sum(x_i - \bar{x})^2}}$.
                    \item In R: \Rfunction{predict(lm.fit, newdata=..., interval="prediction")}.
                \end{itemize}
\begin{lstlisting}[caption={Confidence and Prediction Intervals in R}]
# Using simple linear model lm.fit <- lm(mpg ~ horsepower, data=Auto)
new_hp_values <- data.frame(horsepower=c(98, 150, 200))
# Confidence intervals for average mpg
predict(lm.fit, newdata=new_hp_values, interval="confidence")
# Prediction intervals for individual car's mpg
predict(lm.fit, newdata=new_hp_values, interval="prediction")
\end{lstlisting}
        \end{itemize}

    \subsubsection{Model Extensions and Potential Problems}
        \begin{itemize}
            \item \textbf{Qualitative Predictors}:
                \begin{itemize}
                    \item Create $k-1$ dummy variables for a $k$-level categorical predictor. One level is baseline.
                    \item Coefficients are interpreted relative to the baseline.
                    \item R's \Rfunction{lm} handles factors automatically. \Rfunction{contrasts} shows encoding.
                \end{itemize}
            \item \textbf{Interaction Terms}: $X_1 \cdot X_2$. Allows effect of $X_1$ on $Y$ to depend on the level of $X_2$.
                \begin{itemize}
                    \item Syntax in R: \Rcode{Y ~ X1 + X2 + X1:X2} or \Rcode{Y ~ X1X2}.
                    \item Hierarchical Principle: If an interaction term $X_1:X_2$ is included, the main effects $X_1$ and $X_2$ should also be included, even if their p-values are not significant.
                \end{itemize}
            \item \textbf{Non-linear Relationships}:
                \begin{itemize}
                    \item \textbf{Polynomial Regression}: Include $X^2, X^3, \dots$ as predictors.
                        In R: \Rcode{lm(Y ~ X + I(X^2))}, or \Rcode{lm(Y ~ poly(X, degree=2))}.
                    \item \Rfunction{anova(lm.fit1, lm.fit2)} can compare nested models (e.g., linear vs. quadratic).
                \end{itemize}
            \item \textbf{Potential Problems in Linear Regression}:
                \begin{enumerate}[label=\alph)]
                    \item \textit{Non-linearity of Data}: If true relationship is non-linear, linear model is a poor fit. Detect with residual plots (residuals vs. fitted values, or residuals vs. predictors). Solution: transformations (log, sqrt), polynomial terms, or non-linear models.
                    \item \textit{Correlation of Error Terms ($\epsilon_i$)}: Standard errors will be underestimated, CIs too narrow, p-values too small. Occurs often with time series data. Detect with Durbin-Watson test, plotting residuals vs. time, or ACF of residuals.
                    \item \textit{Non-constant Variance of Errors (Heteroscedasticity)}: Variance of $\epsilon_i$ depends on $X_i$. Funnel shape in residual vs. fitted plot. Invalidates SEs, CIs, hypothesis tests. Solution: transform $Y$ (e.g., $\log Y$, $\sqrt{Y}$), use weighted least squares.
                    \item \textit{Outliers}: Observations with $y_i$ far from model prediction. Large residuals. Can unduly influence model. Detect with studentized residuals (residuals divided by estimated SE; values > |3| are suspect). Solution: remove if data entry error, otherwise be cautious.
                    \item \textit{High-Leverage Points}: Observations with unusual $x_i$ values. Leverage statistic $h_{ii}$ (diagonal of hat matrix $\mathbf{X}(\mathbf{X}'\mathbf{X})^{-1}\mathbf{X}'$). $h_{ii}$ between $1/n$ and $1$. Average $h_{ii} = (p+1)/n$. Points with $h_{ii}$ much larger than average have high leverage. Can have large impact on $\hat{\beta}$.
                    \item \textit{Collinearity/Multicollinearity}: Two or more predictors are highly correlated. Difficult to separate individual effects. SEs of $\hat{\beta}_j$ become large, t-stats small. Overall F-test might be significant while individual p-values are not. Detect with correlation matrix of predictors, Variance Inflation Factor (VIF). $VIF_j = 1/(1-R^2_{X_j|X_{-j}})$. VIF > 5 or 10 indicates problem. Solution: drop one correlated variable, combine variables (e.g., PCA), use ridge regression.
\begin{lstlisting}[caption={Checking VIF in R}]
# Assuming lm.fit.all <- lm(medv ~ ., data=Boston)
library(car) # for vif function
vif_values <- vif(lm.fit.all)
print(vif_values)
# Example: tax and rad often have high VIF in Boston dataset
\end{lstlisting}
                \end{enumerate}
        \end{itemize}

    \subsubsection{K-Nearest Neighbors (KNN) Regression }
        \begin{itemize}
            \item \textbf{Concept}: A non-parametric method that predicts $Y$ for a given $x_0$ by averaging the $Y$ values of its $K$ "closest" neighbors in the training data.
            \item \textbf{Algorithm for prediction at $x_0$} (Slide L2 p.14):
                \begin{enumerate}
                    \item Identify the $K$ training observations $(x_i, y_i)$ that are closest to $x_0$. This set of neighbors is $N_0$.
                    \item The KNN regression fit for $x_0$ is $\hat{f}(x_0) = \frac{1}{K} \sum_{x_i \in N_0} y_i$.
                \end{enumerate}
            \item \textbf{Distance Metric}:
                \begin{itemize}
                    \item For a single predictor $x$: $d(x, x_0) = |x - x_0|$.
                    \item For multiple predictors $\mathbf{x}$: Euclidean distance $\sqrt{\sum_{j=1}^p (x_j - x_{0j})^2}$. It's crucial to scale predictors to similar ranges if they are on different scales, otherwise predictors with larger scales will dominate the distance calculation.
                \end{itemize}
            \item \textbf{Choice of K (The Tuning Parameter)}:
                \begin{itemize}
                    \item $K=1$ (Slide L2 p.17): Lowest bias, highest variance. The prediction function is very flexible and interpolates the training data. Can be very noisy.
                    \item Large $K$ (e.g., $K=20$ on Slide L2 p.18, or $K=n$): Higher bias, lower variance. The prediction function becomes smoother. If $K=n$, it predicts the global mean $\bar{y}$ for all $x_0$.
                    \item Optimal $K$: Balances bias and variance to minimize Test MSE. Typically chosen using cross-validation.
                    \item Slides L2 p.21-24 illustrate this by plotting Squared Bias, Variance, and Test MSE vs. $K$ for a simulated dataset, showing a U-shaped Test MSE curve with an optimal $K \approx 36$.
                \end{itemize}
            \item \textbf{R Example for KNN Regression} (Slides L2 p.15, and extended concept from `ex3.8.R`):
\begin{lstlisting}[caption={Conceptual KNN Regression in R}]
# Function from slide L2 p.15 (for single predictor)
knn_function_single_pred <- function(x0, x_train, y_train, K_val=20) {
  distances <- abs(x_train - x0)
  ordered_indices <- order(distances)
  neighbor_indices <- ordered_indices[1:K_val]
  predicted_y <- mean(y_train[neighbor_indices])
  return(predicted_y)
}

# Example usage (conceptual, needs a loop or apply for multiple x0)
# library(ISLR) # For Auto dataset
# attach(Auto)
# x_values_sorted <- sort(horsepower)
# y_predictions_knn <- sapply(x_values_sorted, knn_function_single_pred, 
#                             x_train=horsepower, y_train=mpg, K_val=5)
# plot(horsepower, mpg)
# lines(x_values_sorted, y_predictions_knn, col="blue", lwd=2)
# detach(Auto)

# For actual KNN, use packages like 'FNN' or 'class' (for classification)
# library(FNN)
# knn.reg(train = as.matrix(horsepower_train), test = as.matrix(horsepower_test), 
#         y = mpg_train, k = 5)
\end{lstlisting}
            \item \textbf{Comparison to Linear Regression}:
                \begin{itemize}
                    \item \textit{Assumptions}: Linear regression assumes a linear functional form. KNN is non-parametric and makes no such assumption.
                    \item \textit{Performance}:
                        \begin{itemize}
                            \item If the true relationship $f(X)$ is close to linear, linear regression usually performs better.
                            \item If $f(X)$ is highly non-linear, KNN can outperform linear regression, especially if $n$ is large and $p$ (number of predictors) is small.
                        \end{itemize}
                    \item \textit{Curse of Dimensionality}: KNN's performance degrades rapidly as $p$ increases because the "nearest" neighbors can be very far away in high-dimensional space, making the local average less meaningful.
                    \item \textit{Interpretability}: Linear regression coefficients are easy to interpret. KNN is less interpretable (a black box).
                \end{itemize}
        \end{itemize}
    \subsubsection{ISLR Chapter 3 Lab Highlights }
        \begin{itemize}
            \item \textbf{Simple Linear Regression} (using \Robject{Boston} data, \Robject{medv} vs \Robject{lstat}):
                \begin{itemize}
                    \item Fitting: \Rfunction{lm(medv ~ lstat, data=Boston)}.
                    \item Output: \Rfunction{summary}, \Rfunction{names}, \Rfunction{coef}, \Rfunction{confint}.
                    \item Prediction: \Rfunction{predict} with \Rcode{interval="confidence"} and \Rcode{interval="prediction"}.
                    \item Plotting: \Rfunction{plot(lstat, medv)}, \Rfunction{abline(lm.fit)}.
                    \item Diagnostics: \Rfunction{par(mfrow=c(2,2))}, \Rfunction{plot(lm.fit)} (Residuals vs Fitted, Normal Q-Q, Scale-Location, Residuals vs Leverage). \Rfunction{hatvalues}.
                \end{itemize}
            \item \textbf{Multiple Linear Regression}:
                \begin{itemize}
                    \item Fitting: \Rfunction{lm(medv ~ lstat + age)}, \Rfunction{lm(medv ~ ., data=Boston)}.
                    \item VIF: \Rfunction{vif(lm.fit)} from \Rpackage{car} package.
                    \item Updating models: \Rfunction{update(lm.fit, .~.-age)}.
                \end{itemize}
            \item \textbf{Interaction Terms}: \Rcode{lm(medv ~ lstatage)}.
            \item \textbf{Non-linear Transformations}: \Rcode{lm(medv ~ lstat + I(lstat\textasciicircum{}2))}, \Rcode{lm(medv ~ poly(lstat, 5))}, \Rcode{lm(medv ~ log(rm))}.
                \begin{itemize}
                    \item Comparing models: \Rfunction{anova(lm.fit.linear, lm.fit.quadratic)}.
                \end{itemize}
            \item \textbf{Qualitative Predictors} (using \Robject{Carseats} data):
                \begin{itemize}
                    \item Factors are handled automatically. \Rfunction{contrasts(Carseats\$ShelveLoc)}.
                \end{itemize}
        \end{itemize}
    \subsubsection{Key Insights from ISLR Chapter 3 Exercises}
        \begin{itemize}
            \item \textbf{ISLR 3.7 (Conceptual)}: Proves that for simple linear regression, $R^2 = (\text{Cor}(X,Y))^2$. The proof involves expanding the definitions of $R^2$, $TSS$, $RSS$, and $\text{Cor}(X,Y)$, and showing their algebraic equivalence using the formulas for $\hat{\beta}_0$ and $\hat{\beta}_1$.
            \item \textbf{ISLR 3.8 (Applied - Auto dataset)}:
                \begin{itemize}
                    \item Fit \Rcode{lm(mpg ~ horsepower)}.
                    \item Interpret relationship: Significant negative relationship.
                    \item Predict \Robject{mpg} for \Robject{horsepower=98}, get confidence and prediction intervals.
                    \item Plot data and regression line.
                    \item Diagnostic plots: Reveal non-linearity (curved pattern in residuals vs. fitted).
                \end{itemize}
            \item \textbf{ISLR 3.9 (Applied - Auto dataset)}:
                \begin{itemize}
                    \item Scatterplot matrix: \Rfunction{pairs(Auto)}.
                    \item Correlation matrix: \Rfunction{cor(subset(Auto, select=-name))}.
                    \item Multiple regression \Rcode{lm(mpg ~ . -name, data=Auto)}.
                    \item Identify significant predictors (displacement, weight, year, origin).
                    \item Diagnostic plots for multiple regression: Similar non-linearity issues.
                    \item Explore interaction terms (e.g., \Rcode{cylindersdisplacement}).
                    \item Explore non-linear transformations (e.g., \Rcode{log(weight)}, \Rcode{sqrt(horsepower)}, \Rcode{I(acceleration\textasciicircum{}2)}).
                    \item A better model might involve transforming the response, e.g., \Rcode{lm(log(mpg) ~ ..., data=Auto)}, which can improve linearity and homoscedasticity of residuals.
                \end{itemize}
            \item \textbf{ISLR 3.10 (Applied - Carseats dataset)}:
                \begin{itemize}
                    \item Fit \Rcode{lm(Sales ~ Price + Urban + US)}.
                    \item Interpret coefficients: Price (negative effect), Urban (not significant), US (positive effect).
                    \item Model selection: Refit without Urban: \Rcode{lm(Sales ~ Price + US)}.
                    \item Compare model fit (RSE, R2).
                    \item Confidence intervals for coefficients of the selected model.
                    \item Check for outliers and high leverage points using diagnostic plots.
                \end{itemize}
            \item \textbf{ISLR 3.13, 3.14 (Simulation Exercises)}:
                \begin{itemize}
                    \item Simulate data with known $\beta_0, \beta_1, \sigma^2$.
                    \item Fit models, compare $\hat{\beta}$ to true $\beta$.
                    \item Observe effect of noise ($\sigma^2$) on model fit and confidence intervals.
                    \item Demonstrate effect of collinearity on coefficient estimates and their significance.
                    \item Effect of outliers/leverage points.
                \end{itemize}
        \end{itemize}
\subsection{Lecture 3: Classification - Logistic Regression}
    \subsubsection{Introduction to Classification }
        \begin{itemize}
            \item \textbf{Goal}: Predict a qualitative (categorical) response variable, $Y$.
            \item Unlike regression (quantitative $Y$), classification assigns an observation to a class.
            \item $Y$ can take values in a set of $K$ classes, e.g., $\{1, 2, \dots, K\}$.
            \item The probability that $Y$ belongs to class $k$ is $p_k$, with $\sum_{k=1}^{K} p_k = 1$.
            \item Models estimate $p_k$ as a function of predictors $X_i$: $p_k = P(Y_i=k | X_i=x_i) = f_k(x_i)$.
        \end{itemize}

    \subsubsection{Measuring Prediction Quality in Classification }
        \begin{itemize}
            \item \textbf{Error Rate}: Proportion of misclassified observations.
            \item Training Error Rate:
                $$ \text{trainER} = \frac{1}{n} \sum_{i=1}^{n} I(\hat{y}_i \neq y_i) $$
                where $I(\cdot)$ is the indicator function (1 if true, 0 if false).
            \item Test Error Rate:
                $$ \text{testER} = \frac{1}{m} \sum_{i=n+1}^{n+m} I(\hat{y}_i \neq y_i) $$
                Calculated on unseen test data of size $m$.
        \end{itemize}

    \subsubsection{Why Not Linear Regression for Classification? }
        \begin{itemize}
            \item If $Y$ is binary (0/1), linear regression $\hat{Y} = \hat{\beta}_0 + \hat{\beta}_1 X$ can produce probabilities $\hat{p}(X)$ outside $[0,1]$.
            \item The coding of classes (e.g., 0/1 vs. 1/2) affects the linear regression fit, which is undesirable for a classification method.
            \item Linear probability model (LPM): Using OLS on a 0/1 coded $Y$. (Slides L3 p. 7-8 illustrate this with \Rpackage{ISLR} \Robject{Default} data, showing predicted probabilities $<0$ or $>1$).
\begin{lstlisting}[caption={Linear Probability Model Example (Slides L3 p.8)}]
# Using Default dataset from ISLR
# y is 0/1 coded default status
# linprob <- lm(y ~ balance, data=Default)
# summary(linprob) # Shows significant relationship
# plot(balance, y)
# abline(linprob, col="red") # Line goes below 0 and above 1
\end{lstlisting}
        \end{itemize}

    \subsubsection{Logistic Regression Model }
        \begin{itemize}
            \item Models the probability $p(X) = P(Y=1|X)$ using the logistic function:
                $$ p(X) = \frac{e^{\beta_0 + \beta_1 X_1 + \dots + \beta_p X_p}}{1 + e^{\beta_0 + \beta_1 X_1 + \dots + \beta_p X_p}} = \frac{1}{1 + e^{-(\beta_0 + \beta_1 X_1 + \dots + \beta_p X_p)}} $$
                This ensures $0 \le p(X) \le 1$. (Slide L3 p.10 shows $p(x)$ for one predictor).
            \item \textbf{Logit Transformation / Log-Odds}:
                $$ \log\left(\frac{p(X)}{1-p(X)}\right) = \beta_0 + \beta_1 X_1 + \dots + \beta_p X_p $$
                The logit is linear in $X$. $\frac{p(X)}{1-p(X)}$ is called the odds.
            \item \textbf{Interpretation of Coefficients} (Slide L3 p.11):
                \begin{itemize}
                    \item A one-unit increase in $X_j$, holding other predictors constant, changes the log-odds by $\beta_j$.
                    \item Equivalently, it multiplies the odds by $e^{\beta_j}$.
                    \item Example (Slide L3 p.11, \Robject{Default} data, $Y=\text{default}$, $X=\text{studentYes}$):
                        \Rcode{glm(default ~ student, family="binomial", data=Default)}
                        Coefficients: (Intercept) -3.5041, studentYes 0.4049.
                        $e^{0.4049} \approx 1.50$. The odds of default for a student are 1.50 times the odds for a non-student.
                \end{itemize}
        \end{itemize}

    \subsubsection{Estimating Coefficients: Maximum Likelihood Estimation (MLE) }
        \begin{itemize}
            \item Goal: Find $\hat{\beta}_0, \hat{\beta}_1, \dots, \hat{\beta}_p$ that maximize the likelihood of observing the given data.
            \item \textbf{Likelihood Function} for binary $Y_i \in \{0,1\}$:
                $$ L(\beta_0, \dots, \beta_p) = \prod_{i: y_i=1} p(x_i) \prod_{i': y_{i'}=0} (1 - p(x_{i'})) $$
                where $p(x_i) = P(Y_i=1 | X_i=x_i; \beta_0, \dots, \beta_p)$.
            \item Often easier to maximize the \textbf{log-likelihood function} (Slide L3 p.14-15):
                $$ \ell(\beta_0, \dots, \beta_p) = \sum_{i=1}^n \left[ y_i \log(p(x_i)) + (1-y_i) \log(1-p(x_i)) \right] $$
                $$ \ell(\beta_0, \beta_1) = \sum_{i=1}^n \left[ y_i (\beta_0 + \beta_1 x_i) - \log(1 + e^{\beta_0 + \beta_1 x_i}) \right] $$ (for single predictor, derived from the formula on Slide L3 p.15)
            \item Maximization is done numerically (e.g., Iteratively Reweighted Least Squares - IRLS).
            \item R: \Rfunction{glm} with \Rcode{family=binomial} uses MLE.
        \end{itemize}

    \subsubsection{Making Predictions and Evaluating Accuracy }
        \begin{itemize}
            \item Prediction is a two-step process (Slide L3 p.10):
                \begin{enumerate}
                    \item Estimate $P(Y=1|X=x)$ using the fitted logistic model: $\hat{p}(x)$.
                    \item Classify as 1 if $\hat{p}(x) > \text{threshold}$ (often 0.5), else 0. (Slide L3 p.9 example: predict default if $p > 0.5$, implies balance > 2000).
                \end{enumerate}
            \item \textbf{Confusion Matrix} (Slides L3 p.16-17):
                \begin{tabular}{cc|cc}
                & & \multicolumn{2}{c}{Predicted Class} \\
                & & Negative (0) & Positive (1) \\ \hline
                Actual & Negative (0) &  (TN) &  (FP) \\
                Class & Positive (1) &  (FN) &  (TP) \\
                \end{tabular}
            \item Common Metrics from Confusion Matrix:
                \begin{itemize}
                    \item \textit{Overall Accuracy}: $(TN+TP) / (Total)$
                    \item \textit{Sensitivity (True Positive Rate, Recall)}: $TP / (TP+FN)$
                    \item \textit{Specificity (True Negative Rate)}: $TN / (TN+FP)$
                    \item \textit{Precision (Positive Predictive Value)}: $TP / (TP+FP)$
                    \item \textit{False Positive Rate}: $FP / (FP+TN) = 1 - \text{Specificity}$
                \end{itemize}
            \item R Example (\Robject{Default} data, predicting \Robject{default}, Slides L3 p.17-18):
\begin{lstlisting}[caption={Logistic Regression and Confusion Matrix (Slides L3 p.17-18)}]
# Fit model (can be on full data or training data)
# logprob <- glm(default ~ student + balance + income, data=Default, family="binomial")
# pred_probs <- predict(logprob, type="response")
# pred_class <- ifelse(pred_probs > 0.5, "Yes", "No") # Using 0.5 threshold
# conf_matrix_full <- table(Default$default, pred_class)
# print(conf_matrix_full)

# Example with train/test split (Slide L3 p.18)
set.seed(123)
n <- nrow(Default)
train_indices <- sample(1:n, n/2)
train_data <- Default[train_indices,]
test_data <- Default[-train_indices,]

logprob_train <- glm(default ~ student + balance + income, data=train_data, family="binomial")
pred_probs_test <- predict(logprob_train, newdata=test_data, type="response")
pred_class_test <- ifelse(pred_probs_test > 0.5, "Yes", "No")
conf_matrix_test <- table(test_data$default, pred_class_test)
print(conf_matrix_test)
#   FALSE TRUE
# No   4809   20
# Yes   116   55
accuracy_test <- sum(diag(conf_matrix_test)) / sum(conf_matrix_test)
# accuracy_test is approx 0.9728
\end{lstlisting}
            \item \textbf{Issue with Imbalanced Classes} (Slide L3 p.19): If one class is much larger (e.g., "No default" is 96.7\%), a naive classifier predicting the majority class for all observations can achieve high overall accuracy.
                \begin{itemize}
                    \item Example: Predicting "Not default" for all in the test set on Slide L3 p.18 (4809+20 = 4829 "No" in test) would give $(4809+20)/(4809+20+116+55) \approx 4829/5000 \approx 0.9658$ if "No" was the only prediction for the No actuals. More accurately, if we only predict "No", accuracy = (TN + 0) / Total = (4809+116)/(5000) if predicting no for "yes" actuals also. The slide implies classifying all 4839 actual "No" correctly, which means a classifier always predicting "No" gets TN=4809, FP=0, FN=116+55=171, TP=0. Acc = 4809 / 5000 = 0.9618. The slide's calculation 4839/5000 seems to refer to the number of non-defaulters in the test set, implying a naive rule "predict No Default" would be correct for 4809 true "No"s, if we consider total test obs as 5000. If the test set had 4839 actual "No" and 161 actual "Yes", and we predict "No" always, accuracy = 4839/5000 = 0.9678.
                    \item Point: High overall accuracy is misleading here. Sensitivity (correctly identifying "Yes" defaults) is often more important for rare events.
                \end{itemize}
        \end{itemize}

    \subsubsection{Multiple Logistic Regression }
        \begin{itemize}
            \item Model: $\log\left(\frac{p(X)}{1-p(X)}\right) = \beta_0 + \beta_1 X_1 + \dots + \beta_p X_p$.
            \item Estimation and interpretation are analogous to simple logistic regression.
            \item \textbf{Confounding}: Relationship between a predictor and response can be distorted if other relevant predictors are omitted.
        \end{itemize}
    \subsubsection{Logistic Regression for >2 Response Classes (Multinomial Logistic Regression) }
        \begin{itemize}
            \item Not explicitly covered in slides, but ISLR discusses it.
            \item One class is chosen as baseline. Model $K-1$ log-odds ratios relative to the baseline.
            \item E.g., for 3 classes (1, 2, 3), baseline class 3:
                $$ \log\left(\frac{P(Y=1|X)}{P(Y=3|X)}\right) = \beta_{01} + \beta_{11}X_1 + \dots $$
                $$ \log\left(\frac{P(Y=2|X)}{P(Y=3|X)}\right) = \beta_{02} + \beta_{12}X_1 + \dots $$
            \item Probabilities sum to 1.
        \end{itemize}
    \subsubsection{Exercise: ISLR 4.13a-d (or old 4.10a-d) }
        (Corresponds to `ex4.13.R` file, first part)
        \begin{itemize}
            \item (a) Explore \Robject{Weekly} data: \Rfunction{summary}, \Rfunction{pairs}, \Rfunction{cor}.
                Look for patterns, e.g., Volume increases over Year. Lags don't show strong correlations with Today's return or Direction.
            \item (b) Logistic regression \Rcode{Direction ~ Lag1+Lag2+Lag3+Lag4+Lag5+Volume}:
\begin{lstlisting}[caption={Logistic Regression on Weekly Data (ISLR 4.13b)}]
# library(ISLR)
# data(Weekly)
# glm.fit <- glm(Direction ~ Lag1 + Lag2 + Lag3 + Lag4 + Lag5 + Volume,
#                data = Weekly, family = binomial)
# summary(glm.fit)
# Output: Lag2 appears to be statistically significant (p-value ~0.03).
\end{lstlisting}
            \item (c) Confusion Matrix and Interpretation:
\begin{lstlisting}[caption={Confusion Matrix for Weekly Data Logistic Model (ISLR 4.13c)}]
# glm.probs <- predict(glm.fit, type="response")
# glm.pred <- rep("Down", length(glm.probs))
# glm.pred[glm.probs > 0.5] <- "Up"
# conf_matrix <- table(Weekly$Direction, glm.pred)
# print(conf_matrix)
# #          glm.pred
# # Direction Down  Up
# #      Down   54  430
# #      Up     48  557
# accuracy <- mean(glm.pred == Weekly$Direction) # (54+557)/1089 = 0.561
# # The model correctly predicts "Up" 557/(48+557) = 92.1% of the time it goes Up.
# # It correctly predicts "Down" 54/(54+430) = 11.2% of the time it goes Down.
# # The model is biased towards predicting "Up".
\end{lstlisting}
            \item (d) Train/Test Split, fit model with \Rcode{Lag2} only, evaluate on test:
                \begin{itemize}
                    \item Training: Years < 2009. Test: Years 2009-2010.
                    \item Fit \Rcode{glm(Direction ~ Lag2, data=train_data, family=binomial)}.
                    \item Predict on test data, create confusion matrix, calculate accuracy.
                    \item \textit{Result in `ex4.13.R`}: Accuracy $\approx 0.625$. This is better than chance (50\%) and slightly better than always predicting "Up" for the test period (which would be $61/104 \approx 0.5865$).
                \end{itemize}
        \end{itemize}


\subsection{Lecture 4: Classification - Discriminant Analysis \& KNN}
    \subsubsection{Linear Discriminant Analysis (LDA) }
        \begin{itemize}
            \item Alternative to logistic regression, especially when classes are well-separated or $n$ is small and predictors are approx. normal.
            \item \textbf{Bayes' Theorem for Classification} (Slides L4 p.3-4):
                \begin{itemize}
                    \item Assign observation $X=x$ to class $k$ that maximizes posterior probability $p_k(x) = P(Y=k|X=x)$.
                    \item $p_k(x) = \frac{\pi_k f_k(x)}{\sum_{l=1}^{K} \pi_l f_l(x)}$
                        \begin{itemize}
                            \item $\pi_k = P(Y=k)$: Prior probability of class $k$. Estimated as proportion of class $k$ in training data ($\hat{\pi}_k = n_k/n$). (Slide L4 p.12)
                            \item $f_k(x) = P(X=x|Y=k)$: Density of $X$ for an observation from class $k$.
                        \end{itemize}
                    \item Bayes Classifier: Ideal, but $f_k(x)$ is usually unknown.
                \end{itemize}
            \item \textbf{LDA Assumptions (for one predictor $p=1$, $K=2$ classes)} (Slide L4 p.8):
                \begin{enumerate}
                    \item $f_k(x)$ is Gaussian (Normal): $X|Y=k \sim N(\mu_k, \sigma_k^2)$.
                    \item LDA specifically assumes common variance: $\sigma_1^2 = \sigma_2^2 = \dots = \sigma_K^2 = \sigma^2$.
                \end{enumerate}
            \item \textbf{LDA Discriminant Function ($\delta_k(x)$)} (Slides L4 p.10-11):
                \begin{itemize}
                    \item Maximize $p_k(x)$ is equivalent to maximizing $\log(p_k(x))$, and further equivalent to maximizing $\delta_k(x)$ after removing terms common to all classes.
                    \item For $p=1$:
                        $$ \delta_k(x) = x \frac{\mu_k}{\sigma^2} - \frac{\mu_k^2}{2\sigma^2} + \log(\pi_k) $$
                    \item Decision rule: Assign $x$ to class $k$ for which $\delta_k(x)$ is largest.
                    \item The decision boundary between two classes $k$ and $l$ (where $\delta_k(x) = \delta_l(x)$) is linear in $x$.
                \end{itemize}
            \item \textbf{Parameter Estimation for LDA ($p=1$)} (Slide L4 p.12):
                \begin{itemize}
                    \item $\hat{\pi}_k = n_k/n$ (proportion of training obs in class $k$).
                    \item $\hat{\mu}_k = \frac{1}{n_k} \sum_{i: y_i=k} x_i$ (average of $x_i$ for class $k$).
                    \item $\hat{\sigma}^2 = \frac{1}{n-K} \sum_{k=1}^K \sum_{i: y_i=k} (x_i - \hat{\mu}_k)^2$ (pooled variance).
                        (Slide L4 p.12 has $\hat{\sigma}^2_k$ for QDA, then $\hat{\sigma}^2$ for LDA is essentially a weighted average if $p=1$, or derived from pooled covariance for $p>1$). The provided slide shows $\hat{\sigma}^2_k$ which is for QDA. For LDA with $p=1$, the formula simplifies to a pooled estimate based on class-wise sums of squares. The formula in the slide for $\hat{\sigma}^2_k$ is for QDA, not LDA's common variance. ISLR p.141 gives common $\hat{\sigma}^2$.
                \end{itemize}
            \item \textbf{LDA for $p>1$ Predictors} (Slide L4 p.15):
                \begin{itemize}
                    \item Assume $X|Y=k \sim N_p(\boldsymbol{\mu}_k, \boldsymbol{\Sigma})$ (multivariate Gaussian with common covariance matrix $\boldsymbol{\Sigma}$).
                    \item Discriminant function:
                        $$ \delta_k(\mathbf{x}) = \mathbf{x}'\boldsymbol{\Sigma}^{-1}\boldsymbol{\mu}_k - \frac{1}{2}\boldsymbol{\mu}_k'\boldsymbol{\Sigma}^{-1}\boldsymbol{\mu}_k + \log(\pi_k) $$
                    \item Decision boundaries are linear (hyperplanes).
                \end{itemize}
            \item \textbf{R Implementation} (\Rpackage{MASS} package, Slides L4 p.13-14, 19):
\begin{lstlisting}[caption={LDA on Default Data (Slides L4 p.13-14 conceptual, ex4.13.R actual)}]
# Conceptual R code from slides for LDA (single predictor 'balance' from Default data)
# x <- Default$balance
# cl <- Default$default # Factor with levels "No", "Yes"
# nk <- table(cl)
# n <- length(cl)
# pi_hat <- nk/n
# mu_hat <- as.matrix(by(x, cl, mean))
# # Pooled variance s2 (simplified for p=1, assuming K=2 classes)
# s2_no <- sum((x[cl=="No"] - mu_hat[1])^2)
# s2_yes <- sum((x[cl=="Yes"] - mu_hat[2])^2)
# s2_pooled <- (s2_no + s2_yes) / (n - 2) # This is (n-K)
# 
# delta_k <- function(x_val, k_idx, mu_vec, s2_val, pi_vec) {
#   # k_idx would be 1 for "No", 2 for "Yes"
#   mu_k <- mu_vec[k_idx]
#   pi_k <- pi_vec[k_idx]
#   return(x_val  mu_k/s2_val - mu_k^2 / (2s2_val) + log(pi_k))
# }
# # To predict: calculate delta1 and delta2 for each x, assign to class with larger delta.
# delta1_vals <- delta_k(x, 1, mu_hat, s2_pooled, pi_hat)
# delta2_vals <- delta_k(x, 2, mu_hat, s2_pooled, pi_hat)
# pred_lda_manual <- ifelse(delta2_vals > delta1_vals, "Yes", "No")
# table(cl, pred_lda_manual)

# Using MASS::lda() (from ex4.13.R for Weekly data)
# library(MASS)
# train_indices <- (Weekly$Year < 2009)
# train_data <- Weekly[train_indices,]
# test_data <- Weekly[!train_indices,]
# lda.fit <- lda(Direction ~ Lag2, data=train_data)
# lda.pred_obj <- predict(lda.fit, newdata=test_data)
# lda.class <- lda.pred_obj$class
# conf_matrix_lda <- table(test_data$Direction, lda.class)
# print(conf_matrix_lda) # Test accuracy was 0.625
# head(lda.pred_obj$posterior) # Shows posterior probabilities
\end{lstlisting}
        \end{itemize}

    \subsubsection{Quadratic Discriminant Analysis (QDA)}
        \begin{itemize}
            \item Similar to LDA, but assumes each class $k$ has its own covariance matrix $\boldsymbol{\Sigma}_k$.
            \item Assumption: $X|Y=k \sim N_p(\boldsymbol{\mu}_k, \boldsymbol{\Sigma}_k)$.
            \item Discriminant function $\delta_k(x)$ becomes quadratic in $x$:
                $$ \delta_k(\mathbf{x}) = -\frac{1}{2}\log|\boldsymbol{\Sigma}_k| - \frac{1}{2}(\mathbf{x}-\boldsymbol{\mu}_k)'\boldsymbol{\Sigma}_k^{-1}(\mathbf{x}-\boldsymbol{\mu}_k) + \log(\pi_k) $$
            \item Decision boundaries are quadratic.
            \item \textbf{LDA vs. QDA Trade-off}:
                \begin{itemize}
                    \item LDA: Estimates $p$ parameters for $\boldsymbol{\mu}_k$ (for each class), and $p(p+1)/2$ for common $\boldsymbol{\Sigma}$. Less flexible.
                    \item QDA: Estimates $p$ parameters for $\boldsymbol{\mu}_k$ and $p(p+1)/2$ for $\boldsymbol{\Sigma}_k$ (for each of $K$ classes). More flexible.
                    \item QDA has higher variance but lower bias (if true decision boundary is non-linear).
                    \item LDA better if $n$ is small or common covariance assumption is reasonable. QDA better if $n$ is large and common covariance is violated.
                \end{itemize}
            \item R Implementation (\Rpackage{MASS} package):
\begin{lstlisting}[caption={QDA on Weekly Data (from ex4.13.R)}]
# library(MASS)
# qda.fit <- qda(Direction ~ Lag2, data=train_data) # Using same train/test as LDA
# qda.pred_obj <- predict(qda.fit, newdata=test_data)
# qda.class <- qda.pred_obj$class
# conf_matrix_qda <- table(test_data$Direction, qda.class)
# print(conf_matrix_qda) # Test accuracy was 0.5865 (predicted "Up" always)
\end{lstlisting}
        \end{itemize}

    \subsubsection{Comparison of Classification Methods (Logistic, LDA, QDA, KNN) }
        \begin{itemize}
            \item \textbf{Logistic Regression vs. LDA}:
                \begin{itemize}
                    \item Both produce linear decision boundaries (if $p>1$).
                    \item LDA assumes Gaussian $f_k(x)$ with common $\boldsymbol{\Sigma}$. LR makes no such assumption about $f_k(x)$, models $P(Y=k|X)$ directly.
                    \item If Gaussian assumption holds, LDA is more efficient (stable estimates with smaller $n$).
                    \item LR is generally more robust if Gaussian assumption is violated.
                    \item In practice, often similar performance.
                    \item LDA parameter estimates can be unstable if classes are well-separated (LR too). (Slide L4 p.16)
                    \item LDA more stable than LR for small $n$ if $X$ is approx. multinormal. (Slide L4 p.16)
                    \item LDA is a natural approach for $K>2$ classes. (Slide L4 p.16)
                \end{itemize}
            \item \textbf{KNN}:
                \begin{itemize}
                    \item Completely non-parametric. Makes no assumptions about decision boundary shape.
                    \item Can outperform parametric methods if boundary is highly non-linear.
                    \item Requires large $n$ for good performance, especially if $p$ is large (curse of dimensionality).
                    \item Optimal $K$ is crucial.
                \end{itemize}
            \item \textbf{QDA}:
                \begin{itemize}
                    \item Compromise between non-parametric KNN and rigid LDA/LR.
                    \item Assumes Gaussian $f_k(x)$ but allows different $\boldsymbol{\Sigma}_k$.
                    \item Good if true boundary is moderately non-linear and Gaussian assumption is reasonable.
                \end{itemize}
        \end{itemize}

    \subsubsection{Evaluating Classification Models (Revisited)}
        \begin{itemize}
            \item \textbf{Threshold Choice} (Slide L4 p.18):
                \begin{itemize}
                    \item Default threshold for binary classification is 0.5 for $P(Y=1|X)$.
                    \item Can be adjusted based on costs of misclassification or to balance sensitivity/specificity.
                    \item Example: For \Robject{Default} data, one might use threshold < 0.5 (e.g., 0.2) to identify more potential defaulters, accepting more false positives.
                \end{itemize}
            \item \textbf{ROC Curve (Receiver Operating Characteristics)} (Slides L4 p.19-21):
                \begin{itemize}
                    \item Plots True Positive Rate (Sensitivity) vs. False Positive Rate (1 - Specificity) for various threshold values.
                    \item A good classifier has ROC curve far from the 45-degree line (random guessing) towards top-left corner.
                    \item \textbf{AUC (Area Under Curve)}: Summary measure of ROC performance.
                        \begin{itemize}
                            \item AUC = 1: Perfect classifier.
                            \item AUC = 0.5: Random guessing.
                            \item AUC > 0.7 generally considered acceptable, >0.8 good, >0.9 excellent.
                        \end{itemize}
                \end{itemize}
            \item R code for ROC (conceptual from Slides L4 p.21, using LDA predictions):
\begin{lstlisting}[caption={Generating ROC Curve Data (Conceptual, Slides L4 p.21)}]
# Assume lda1 fitted on cl ~ x (Default data)
# pr <- predict(lda1)$posterior # Posterior probabilities P(Y=k|X)
# cl_numeric <- as.numeric(Default$default) - 1 # 0 for No, 1 for Yes
# 
# thrange <- seq(0.01, 0.99, by=0.01) # Thresholds to test
# roc_data <- data.frame(FPrate=numeric(length(thrange)), 
#                        TPrate=numeric(length(thrange)))
# 
# for (i in 1:length(thrange)) {
#   th <- thrange[i]
#   pred_class <- ifelse(pr[, "Yes"] > th, 1, 0) # Predict "Yes" if P(Yes|X) > th
#   
#   TP <- sum(pred_class == 1 & cl_numeric == 1)
#   FN <- sum(pred_class == 0 & cl_numeric == 1)
#   FP <- sum(pred_class == 1 & cl_numeric == 0)
#   TN <- sum(pred_class == 0 & cl_numeric == 0)
#   
#   roc_data$TPrate[i] <- TP / (TP + FN) # Sensitivity
#   roc_data$FPrate[i] <- FP / (FP + TN) # 1 - Specificity
# }
# plot(roc_data$FPrate, roc_data$TPrate, type="l", col="red", lwd=2,
#      xlab="False Positive Rate", ylab="True Positive Rate", main="ROC Curve")
# abline(a=0, b=1, lty=2) # Line of no discrimination
# # Packages like 'pROC' or 'ROCR' can do this more easily and calculate AUC.
\end{lstlisting}
        \end{itemize}

    \subsubsection{K-Nearest Neighbors (KNN) Classifier}
        \begin{itemize}
            \item To classify a test observation $x_0$:
                \begin{enumerate}
                    \item Find the $K$ training points closest to $x_0$ (neighborhood $N_0$).
                    \item Estimate conditional probability for class $j$: $P(Y=j|X=x_0) = \frac{1}{K} \sum_{i \in N_0} I(y_i=j)$.
                    \item Assign $x_0$ to the class with the largest probability (majority vote among neighbors).
                \end{enumerate}
            \item Choice of $K$ is critical, often selected by CV.
            \item Small $K$: Flexible boundary, low bias, high variance.
            \item Large $K$: Smoother, less flexible boundary, high bias, low variance.
            \item R Implementation (\Rpackage{class} package):
\begin{lstlisting}[caption={KNN Classification (from ex4.13.R for Weekly data)}]
# library(class)
# train_X_knn <- as.matrix(train_data$Lag2)
# test_X_knn <- as.matrix(test_data$Lag2)
# train_Direction_knn <- train_data$Direction
# 
# set.seed(1) # For reproducibility if there are ties
# knn.pred_k1 <- knn(train=train_X_knn, test=test_X_knn, cl=train_Direction_knn, k=1)
# conf_matrix_knn_k1 <- table(test_data$Direction, knn.pred_k1)
# print(conf_matrix_knn_k1) # Test accuracy 0.5
# 
# # Trying different K values (e.g., K=10)
# # knn.pred_k10 <- knn(train=train_X_knn, test=test_X_knn, cl=train_Direction_knn, k=10)
# # accuracy_k10 <- mean(knn.pred_k10 == test_data$Direction)
\end{lstlisting}
        \end{itemize}
    \subsubsection{Exercises: ISLR 4.13e-i (or old 4.10e-i)}
        (Corresponds to `ex4.13.R` file, second part)
        \begin{itemize}
            \item (e) LDA on \Rcode{Direction ~ Lag2} (train/test split as before):
                \begin{itemize}
                    \item Fit LDA, predict on test set, calculate accuracy.
                    \item \textit{Result from `ex4.13.R`}: Accuracy $\approx 0.625$. Same as logistic regression with only \Rcode{Lag2}.
                \end{itemize}
            \item (f) QDA on \Rcode{Direction ~ Lag2}:
                \begin{itemize}
                    \item Fit QDA, predict, calculate accuracy.
                    \item \textit{Result from `ex4.13.R`}: Accuracy $\approx 0.5865$. QDA predicts "Up" for all test observations, performing worse than LDA/logistic.
                \end{itemize}
            \item (g) KNN with K=1 on \Rcode{Direction ~ Lag2}:
                \begin{itemize}
                    \item Fit KNN (K=1), predict, calculate accuracy.
                    \item \textit{Result from `ex4.13.R`}: Accuracy $\approx 0.50$. Performs poorly.
                \end{itemize}
            \item (h) Compare Logistic, LDA, QDA, KNN(K=1):
                \begin{itemize}
                    \item Logistic regression (with \Rcode{Lag2}) and LDA performed best and identically (accuracy 0.625).
                    \item QDA and KNN(K=1) performed worse.
                \end{itemize}
            \item (i) Experiment with different combinations/transformations/K for KNN:
                \begin{itemize}
                    \item Logistic with \Rcode{Lag2:Lag1} interaction: Accuracy $\approx 0.5865$.
                    \item LDA with \Rcode{Lag2:Lag1} interaction: Accuracy $\approx 0.5769$.
                    \item QDA with \Rcode{Lag2 + sqrt(abs(Lag2))}: Accuracy $\approx 0.5769$.
                    \item KNN with K=10: Accuracy $\approx 0.5769$.
                    \item KNN with K=100: Accuracy $\approx 0.5577$.
                    \item \textit{Conclusion}: Original simpler logistic regression and LDA with only \Rcode{Lag2} were the best performers among these experiments.
                \end{itemize}
        \end{itemize}
\subsection{Lecture 5: Resampling Methods - Cross-Validation }
    \subsubsection{Introduction to Resampling Methods }
        \begin{itemize}
            \item \textbf{Sampling Distribution} : The distribution of a statistic (e.g., sample mean $\bar{X}$) if we were to repeatedly draw samples from the population.
                \begin{itemize}
                    \item Example: Estimating average height $\mu=E(X)$ of Norwegian adults. Each sample of $n=100$ gives one estimate $\hat{\mu} = \bar{x}$. The collection of these $\bar{x}$ values from many hypothetical samples forms the sampling distribution. (Slides L5 p.3-5 show two such sample means: 183.678 and 180.678).
                    \item \textbf{Central Limit Theorem (CLT)} (Slides L5 p.6, 9): For large $n$, the sampling distribution of $\bar{X}$ is approximately Normal: $\bar{X} \sim N(\mu, \sigma^2/n)$.
                    \item We can estimate this sampling distribution using a single sample by plugging in estimates $\hat{\mu}=\bar{x}$ and $\hat{\sigma}=s$: $\bar{X} \sim N(\bar{x}, s^2/n)$. (Slides L5 p.10-11). This allows for constructing confidence intervals (Slide L5 p.12).
                \end{itemize}
            \item \textbf{Why Resampling?} (Slide L5 p.13):
                \begin{itemize}
                    \item Large sample approximations (like CLT) may not hold for small $n$ or for complex statistics where the asymptotic distribution is hard to derive.
                    \item Resampling methods mimic sampling from a population by instead repeatedly sampling from our \textit{observed sample data}.
                    \item Often used to estimate test error $E((Y_i - \hat{f}(X_i))^2)$ or to assess variability of model parameters.
                \end{itemize}
        \end{itemize}

    \subsubsection{Cross-Validation (CV) }
        \begin{itemize}
            \item \textbf{Core Idea}:
                \begin{enumerate}
                    \item Split data into a \textit{training set} and a \textit{test (or validation) set}.
                    \item Fit (train/estimate) the model on the training set.
                    \item Evaluate model performance (e.g., predict) on the test set.
                \end{enumerate}
            \item \textbf{Purpose}:
                \begin{itemize}
                    \item \textit{Model Assessment}: Estimating the test error of a final chosen model.
                    \item \textit{Model Selection}: Choosing the appropriate level of flexibility (e.g., degree of polynomial, value of $K$ in KNN).
                \end{itemize}
            \item \textbf{Types of CV}: Validation Set Approach, Leave-One-Out CV (LOOCV), k-Fold CV.
        \end{itemize}

    \subsubsection{The Validation Set Approach}
        \begin{itemize}
            \item \textbf{Procedure} (Slide L5 p.15):
                \begin{enumerate}
                    \item Randomly split the $n$ observations into a training set (e.g., $m \approx n/2$ observations) and a validation/test set ($n-m$ observations).
                    \item Fit the model using only the training data to get $\hat{f}$.
                    \item Evaluate $\hat{f}$ on the validation set by calculating the test MSE (or other error metric):
                        $$ \text{Test MSE}_{\text{valid}} = \frac{1}{n-m} \sum_{i \in \text{valid\_set}} (y_i - \hat{f}(x_i))^2 $$
                \end{enumerate}
            \item \textbf{R Example (\Rpackage{ISLR} \Robject{Auto} data)} (Slides L5 p.16-21):
                \begin{itemize}
                    \item Predict \Robject{mpg} from \Robject{horsepower}.
\begin{lstlisting}[caption={Validation Set Approach in R (Slides L5 p.16, 18, 21)}]
library(ISLR)
# names(Auto) # mpg, cylinders, displacement, horsepower, ...
n <- nrow(Auto)
set.seed(1) # For reproducibility of the random split
draw <- sample(1:n, size=floor(n/2)) # Indices for training set
train_data <- Auto[draw, ]
test_data <- Auto[-draw, ]

# Model 1: Linear regression
mod1 <- lm(mpg ~ horsepower, data=train_data)
# summary(mod1) # (Slide L5 p.18 shows an example output)

# Model 2: Polynomial regression (degree 2)
mod2 <- lm(mpg ~ horsepower + I(horsepower^2), data=train_data)
# summary(mod2) # (Slide L5 p.20 shows an example output)

# Predicting on test data and calculating Test MSE
pred1_test <- predict(mod1, newdata=test_data)
mse1_test <- mean((test_data$mpg - pred1_test)^2)
# mse1_test -> 27.36073 (from slide L5 p.21, actual value depends on seed)

pred2_test <- predict(mod2, newdata=test_data)
mse2_test <- mean((test_data$mpg - pred2_test)^2)
# mse2_test -> 20.29991 (from slide L5 p.21, actual value depends on seed)
# Polynomial model has lower Test MSE in this specific split.
\end{lstlisting}
                \end{itemize}
            \item \textbf{Drawbacks of Validation Set Approach} (ISLR Ch 5.1.1, Conceptual Ex 5.3b.i):
                \begin{enumerate}
                    \item \textit{High Variability}: The estimated test MSE can be highly variable depending on which observations end up in the training vs. validation set. Different splits can lead to different conclusions about model performance. (The slide L5 p.21 asks "Did you get the exact same numbers?" highlighting this variability).
                    \item \textit{Overestimation of Test Error}: Only a subset of data ($m<n$) is used for training. Models fit on less data tend to perform worse. So, validation set MSE might overestimate the test error of a model fit to the full dataset.
                \end{enumerate}
        \end{itemize}

    \subsubsection{Leave-One-Out Cross-Validation (LOOCV) }
        \begin{itemize}
            \item \textbf{Procedure} (Slide L5 p.22):
                \begin{enumerate}
                    \item For each observation $i=1, \dots, n$:
                        \begin{itemize}
                            \item Hold out observation $(x_i, y_i)$ (this is the "test set" of size 1).
                            \item Fit the model $\hat{f}^{(-i)}$ using the remaining $n-1$ observations (training set).
                            \item Predict $y_i$ using $\hat{f}^{(-i)}(x_i)$ and calculate the squared error: $MSE_i = (y_i - \hat{f}^{(-i)}(x_i))^2$.
                        \end{itemize}
                    \item The LOOCV estimate of test MSE is the average of these $n$ errors:
                        $$ CV_{(n)} = \text{LOOCV MSE} = \frac{1}{n} \sum_{i=1}^{n} MSE_i $$
                \end{enumerate}
            \item \textbf{R Example (\Robject{Auto} data)} (Slide L5 p.23):
\begin{lstlisting}[caption={LOOCV for Auto Data (Conceptual from Slide L5 p.23)}]
# MSE1_loo <- numeric(n) # For linear model
# MSE2_loo <- numeric(n) # For quadratic model
# 
# for(i in 1:n) {
#   train_loo <- Auto[-i, ]
#   test_loo <- Auto[i, ]
#   
#   mod1_loo <- lm(mpg ~ horsepower, data=train_loo)
#   pred1_loo <- predict(mod1_loo, newdata=test_loo)
#   MSE1_loo[i] <- (test_loo$mpg - pred1_loo)^2
#   
#   mod2_loo <- lm(mpg ~ horsepower + I(horsepower^2), data=train_loo)
#   pred2_loo <- predict(mod2_loo, newdata=test_loo)
#   MSE2_loo[i] <- (test_loo$mpg - pred2_loo)^2
# }
# 
# mean_MSE1_loo <- mean(MSE1_loo) # Slide says 24.23
# mean_MSE2_loo <- mean(MSE2_loo) # Slide says 19.25 (Quadratic is better)

# Using boot::cv.glm() for LOOCV with linear models (more efficient)
# library(boot)
# glm.fit.linear <- glm(mpg ~ horsepower, data=Auto)
# cv.err.linear <- cv.glm(Auto, glm.fit.linear) # Default is LOOCV
# print(cv.err.linear$delta[1]) # First element is raw LOOCV MSE
# 
# glm.fit.quad <- glm(mpg ~ poly(horsepower, 2), data=Auto) # poly() is preferred
# cv.err.quad <- cv.glm(Auto, glm.fit.quad)
# print(cv.err.quad$delta[1])
\end{lstlisting}
            \item \textbf{Advantages of LOOCV}:
                \begin{itemize}
                    \item \textit{Less Bias}: Uses $n-1$ observations for training in each fold, so $\hat{f}^{(-i)}$ is very similar to $\hat{f}$ fit on all $n$ observations. LOOCV MSE tends to be an almost unbiased estimate of test error for a model trained on $n$ observations.
                    \item \textit{No Randomness}: Result is always the same, no variability due to random splits.
                \end{itemize}
            \item \textbf{Disadvantages of LOOCV} (ISLR Ch 5.1.2, Conceptual Ex 5.3b.ii):
                \begin{itemize}
                    \item \textit{Computationally Expensive}: Model must be fit $n$ times. Can be very slow for complex models or large $n$.
                        (For OLS linear regression and polynomial regression, a computational shortcut exists: $CV_{(n)} = \frac{1}{n} \sum (\frac{y_i - \hat{y}_i}{1-h_{ii}})^2$, where $h_{ii}$ is leverage of $i$-th obs. So, only need to fit full model once).
                    \item \textit{High Variance}: The $n$ training sets are highly overlapping (share $n-2$ observations). The $MSE_i$ values are highly correlated, so their average ($CV_{(n)}$) can have high variance. It can be a poor estimator of the true test MSE.
                \end{itemize}
        \end{itemize}

    \subsubsection{k-Fold Cross-Validation}
        \begin{itemize}
            \item \textbf{Procedure}:
                \begin{enumerate}
                    \item Randomly divide the $n$ observations into $k$ non-overlapping groups (folds) of approximately equal size ($n/k$).
                    \item For each fold $j=1, \dots, k$:
                        \begin{itemize}
                            \item Hold out fold $j$ as the test set.
                            \item Fit the model $\hat{f}^{(-j)}$ using the other $k-1$ folds as the training set.
                            \item Calculate $MSE_j = \frac{1}{n_j} \sum_{i \in \text{fold }j} (y_i - \hat{f}^{(-j)}(x_i))^2$, where $n_j$ is size of fold $j$.
                        \end{itemize}
                    \item The k-fold CV estimate of test MSE is:
                        $$ CV_{(k)} = \frac{1}{k} \sum_{j=1}^{k} MSE_j $$
                \end{enumerate}
            \item Common choices for $k$: 5 or 10. (Slide L5 p.24 notes $k=n$ is LOOCV).
            \item \textbf{Advantages over LOOCV}:
                \begin{itemize}
                    \item \textit{Computationally Cheaper}: Model fit only $k$ times.
                    \item \textit{Lower Variance}: Training sets for each fold are less overlapping than in LOOCV, leading to less correlated $MSE_j$ values. $CV_{(k)}$ often has lower variance than $CV_{(n)}$.
                \end{itemize}
            \item \textbf{Bias-Variance Trade-off for Choice of $k$}:
                \begin{itemize}
                    \item \textit{Bias}: $CV_{(k)}$ uses training sets of size $n(k-1)/k$, which is smaller than $n-1$ (for LOOCV). So $CV_{(k)}$ might have slightly more bias as an estimator of test error for a model fit on full data, compared to LOOCV.
                    \item \textit{Variance}: $CV_{(k)}$ generally has lower variance than LOOCV.
                    \item $k=5$ or $k=10$ often strike a good balance.
                \end{itemize}
            \item \textbf{CV for Classification Problems}: Use error rate instead of MSE.
                $$ CV_{(k)} = \frac{1}{k} \sum_{j=1}^{k} \text{Err}_j = \frac{1}{k} \sum_{j=1}^{k} \left( \frac{1}{n_j} \sum_{i \in \text{fold }j} I(y_i \neq \hat{y}_i) \right) $$
            \item R Implementation (using \Rpackage{boot}::\Rfunction{cv.glm}):
\begin{lstlisting}[caption={k-Fold CV using boot::cv.glm}]
# library(boot)
# For k-fold CV, specify K argument in cv.glm
# Example: 10-fold CV for linear model on Auto data
# glm.fit.auto <- glm(mpg ~ horsepower, data=Auto)
# cv.err.10fold <- cv.glm(Auto, glm.fit.auto, K=10)
# print(cv.err.10fold$delta[1]) # Raw 10-fold CV MSE

# Example: 10-fold CV for logistic regression on Default data (ISLR Ch5 Ex5)
# glm.fit.default <- glm(default ~ income + balance, data=Default, family=binomial)
# cv.err.default.10fold <- cv.glm(Default, glm.fit.default, K=10) # Cost function needed for error rate
# To get error rate for classification, must define a cost function for cv.glm
# Or implement k-fold CV manually:
# k <- 10
# folds <- sample(cut(seq(1,nrow(Default)),breaks=k,labels=FALSE))
# cv_errors <- numeric(k)
# for(j in 1:k){
#   test_indices <- which(folds==j, arr.ind=TRUE)
#   test_data_fold <- Default[test_indices, ]
#   train_data_fold <- Default[-test_indices, ]
#   fit_fold <- glm(default ~ income + balance, data=train_data_fold, family=binomial)
#   probs_fold <- predict(fit_fold, newdata=test_data_fold, type="response")
#   preds_fold <- ifelse(probs_fold > 0.5, "Yes", "No")
#   cv_errors[j] <- mean(preds_fold != test_data_fold$default)
# }
# mean_cv_error_10fold <- mean(cv_errors)
\end{lstlisting}
        \end{itemize}

    \subsubsection{Pros and Cons of Different CV Approaches }
        \begin{table}[H]
            \centering
            \caption{Comparison of Cross-Validation Methods (Slide L5 p.25, ISLR Ch 5.1.4)}
            \begin{tabular}{@{}llll@{}}
                \toprule
                Property & Validation Set & LOOCV & k-Fold CV \\ \midrule
                Bias of Test Error Estimate & Higher & Lower (Approx. Unbiased) & Intermediate \\
                Variance of Test Error Estimate & Higher & Higher & Lower \\
                Computational Time & Best (1 fit) & Worst ($n$ fits) & Intermediate ($k$ fits) \\ \bottomrule
            \end{tabular}
            \label{tab:cv_comparison}
        \end{table}

        
        \begin{itemize}
            \item The table summarizes that LOOCV is good for low bias in estimating test error but can have high variance and is computationally expensive. k-Fold CV (e.g., $k=5$ or $10$) often provides a better bias-variance trade-off for the estimate of test error itself and is computationally more feasible. The validation set approach is simple but can be unreliable.
        \end{itemize}

    \subsubsection{The Right and Wrong Way to do Cross-Validation }
        \begin{itemize}
            \item \textbf{Crucial Point}: Any variable selection, feature engineering, or parameter tuning step that relies on the outcomes ($y_i$) must be performed \textit{inside} the CV loop, using only the training data of that specific fold.
            \item \textbf{Wrong Way}: Perform variable selection on the entire dataset first, then use CV to estimate the error of the chosen model. This "leaks" information from the test folds into the model selection process, leading to an overly optimistic (too low) estimate of test error.
            \item \textbf{Right Way}: In each CV fold $j$:
                \begin{enumerate}
                    \item Perform variable selection (or other tuning) using \textit{only the training part} of fold $j$.
                    \item Fit the model selected in step 1 using \textit{only the training part} of fold $j$.
                    \item Evaluate this model on the \textit{test part} (hold-out portion) of fold $j$.
                \end{enumerate}
        \end{itemize}
    \subsubsection{Exercise: ISLR 5.5 (Cross-validation for Logistic Regression)}
        
        \begin{itemize}
            \item (a) Fit logistic regression \Rcode{default ~ income + balance} on full dataset.
            \item (b) Validation Set Approach:
                \begin{itemize}
                    \item Split data 50/50.
                    \item Fit model on training.
                    \item Predict on validation, calculate test error rate (misclassification).
                    \item \textit{Result from `ex5.5.R`}: Test error rate $\approx 2.36\%$ to $2.86\%$ depending on split.
                \end{itemize}
            \item (c) Repeat (b) multiple times: Shows variability of validation set error.
            \item (d) Validation Set with \Rcode{student} dummy: Add \Rcode{student} predictor.
                \begin{itemize}
                    \item Fit \Rcode{default ~ income + balance + student} on training.
                    \item Test error rate $\approx 2.64\%$. Does not seem to improve much over model without \Rcode{student} using this single validation split. (More robust comparison would require repeated CV or k-fold CV).
                \end{itemize}
            \item (Extra Task from `ex5.5.R`: LOOCV on smaller subset)
                \begin{itemize}
                    \item Illustrates manual LOOCV loop for logistic regression on 500 observations.
                    \item For logistic regression, \Rfunction{cv.glm} with default settings (or \Rcode{K=nrow(data)}) performs LOOCV but requires a custom cost function to output error rate instead of deviance for binomial models.
\begin{lstlisting}[caption={Manual LOOCV for Classification Error (Conceptual)}]
# n_subset <- 500
# subset_data <- Default[sample(1:nrow(Default), n_subset), ]
# loocv_errors <- numeric(n_subset)
# for(i in 1:n_subset) {
#   train_fold <- subset_data[-i, ]
#   test_fold <- subset_data[i, ]
#   fit_fold <- glm(default ~ income + balance + student, 
#                   data=train_fold, family=binomial)
#   prob_fold <- predict(fit_fold, newdata=test_fold, type="response")
#   pred_fold <- ifelse(prob_fold > 0.5, "Yes", "No")
#   loocv_errors[i] <- (pred_fold != test_fold$default)
# }
# mean_loocv_error <- mean(loocv_errors)
\end{lstlisting}


\subsection{Lecture 6: Resampling Methods - The Bootstrap}
    \subsubsection{Introduction to the Bootstrap}
        \begin{itemize}
            \item \textbf{Purpose}: A powerful and widely applicable tool for quantifying uncertainty associated with a given estimator or statistical learning method.
                \begin{itemize}
                    \item Primarily used to estimate the standard error (SE) of an estimator.
                    \item Can also be used to construct confidence intervals.
                    \item Useful when the true sampling distribution of a statistic is unknown or difficult to derive analytically (e.g., for medians, quantiles, complex model parameters).
                \end{itemize}
            \item \textbf{Core Idea}: Mimic the process of obtaining new sample sets from the population by repeatedly sampling \textit{with replacement} from the original observed dataset.
            \item Each "bootstrap sample" has the same size $n$ as the original dataset.
            \item Some observations from the original dataset may appear multiple times in a bootstrap sample, while others may not appear at all.
        \end{itemize}

    \subsubsection{The Bootstrap Procedure for Estimating Standard Error}
        \begin{enumerate}
            \item Let the original dataset be $Z = \{z_1, z_2, \dots, z_n\}$.
            \item Generate $B$ independent bootstrap samples $Z^{1}, Z^{2}, \dots, Z^{B}$. Each $Z^{b}$ is obtained by drawing $n$ observations from $Z$ with replacement.
            \item For each bootstrap sample $Z^{b}$, compute the statistic of interest, $\hat{\alpha}^{b}$. This could be a sample mean, median, regression coefficient, etc.
            \item The bootstrap estimate of the standard error of $\hat{\alpha}$ (the statistic computed on the original data) is the standard deviation of the $B$ bootstrap estimates:
                $$ SE_B(\hat{\alpha}) = \sqrt{\frac{1}{B-1} \sum_{b=1}^{B} \left(\hat{\alpha}^{b} - \frac{1}{B}\sum_{r=1}^{B}\hat{\alpha}^{r}\right)^2} $$
            \item A large number of bootstrap samples $B$ (e.g., $B=1000$ or more) is typically used.
        \end{enumerate}

    \subsubsection{Example: Estimating SE of Investment Allocation }
        \begin{itemize}
            \item Task: Minimize variance of portfolio $\alpha X + (1-\alpha)Y$.
            \item Optimal $\alpha = \frac{\sigma_Y^2 - \sigma_{XY}}{\sigma_X^2 + \sigma_Y^2 - 2\sigma_{XY}}$.
            \item Suppose we have $n=100$ observations of $(X,Y)$. We can estimate $\hat{\sigma}_X^2, \hat{\sigma}_Y^2, \hat{\sigma}_{XY}$ from this sample, and then $\hat{\alpha}$.
            \item To find $SE(\hat{\alpha})$ using bootstrap:
                \begin{enumerate}
                    \item Draw $B$ bootstrap samples $(X^{b}, Y^{b})$ of size $n=100$ from the original 100 observations.
                    \item For each bootstrap sample $b$, calculate $\hat{\alpha}^{b}$ using the sample variances and covariance from that bootstrap sample.
                    \item Compute $SE_B(\hat{\alpha})$ as the standard deviation of these $B$ values of $\hat{\alpha}^{b}$.
                \end{enumerate}
            \item This provides a measure of how much $\hat{\alpha}$ would vary if we repeatedly sampled 100 observations from the true population.
        \end{itemize}

    \subsubsection{R Implementation of Bootstrap }
        \begin{itemize}
            \item The \Rpackage{boot} package provides the \Rfunction{boot} function for easy bootstrap implementation.
            \item \textbf{Key arguments for \Rfunction{boot}} (ISLR Ch 5 Ex 5.6):
                \begin{itemize}
                    \item \Rcode{data}: The original dataset.
                    \item \Rcode{statistic}: A function that takes two arguments: \Rcode{data} and \Rcode{index}. The \Rcode{index} argument will contain the indices of the observations selected for a particular bootstrap sample. The function should return the statistic(s) of interest computed on \Rcode{data[index,]}.
                    \item \Rcode{R}: The number of bootstrap replicates $B$.
                \end{itemize}
            \item Example: Bootstrap SE for coefficients of logistic regression on \Robject{Default} data (ISLR Ch 5 Ex 5.6).
\begin{lstlisting}[caption={Bootstrap for Logistic Regression Coefficients (ISLR Ch 5 Ex 5.6)}]
library(ISLR) # For Default dataset
library(boot)   # For boot() function

# (a) Fit the logistic regression model once on original data
glm.fit.full <- glm(default ~ income + balance, data = Default, family = binomial)
# summary(glm.fit.full)$coef # Shows original estimates and SEs from asymptotic theory

# (b) Define the statistic function for bootstrap
# This function will be called by boot() for each bootstrap sample.
# It needs to return the coefficients of the logistic regression model.
boot.fn.coeffs <- function(data, index) {
  # Fit model on the bootstrap sample (data[index,])
  fit <- glm(default ~ income + balance, data = data[index, ], family = binomial)
  return(coef(fit)) # Return the vector of estimated coefficients
}

# (c) Run the bootstrap
set.seed(1) # For reproducibility
boot_results <- boot(data = Default, statistic = boot.fn.coeffs, R = 1000) # B=1000 replicates

# Print the bootstrap results
print(boot_results)
# Output includes:
# original: Coefficients from the model fit on the original data
# bias: Difference between mean of bootstrap estimates and original estimate
# std. error: Bootstrap estimate of SE for each coefficient

# (d) Comparison: Bootstrap SEs are usually close to SEs from summary(glm.fit.full),
# but bootstrap is more general and doesn't rely on asymptotic theory.
# E.g., from Ex 5.6 solution:
# Original glm SE for income: 4.99e-06, balance: 2.27e-04
# Bootstrap SE for income:   ~4.5e-06, balance: ~2.3e-04 (approx from R=50 in book example)
\end{lstlisting}
            \item Example: Bootstrap SE for the sample median (ISLR Ch 5 Ex 5.9f, Boston data).
\begin{lstlisting}[caption={Bootstrap SE for Median (ISLR Ch 5 Ex 5.9f)}]
library(MASS) # For Boston dataset
library(boot)
# attach(Boston) # or use Boston$medv

# (e) Calculate median on original data
medv.median.original <- median(Boston$medv) # 21.2

# (f) Define statistic function and run bootstrap
boot.fn.median <- function(data_vector, index) {
  return(median(data_vector[index]))
}
set.seed(1)
boot_median_results <- boot(data = Boston$medv, statistic = boot.fn.median, R = 1000)
print(boot_median_results)
# Bootstrap Statistics :
#     original  bias    std. error
# t1     21.2 -0.0098      0.3874 (from ISLR solution)
# The bootstrap SE for the median is ~0.387.
# detach(Boston)
\end{lstlisting}
        \end{itemize}

    \subsubsection{Bootstrap Confidence Intervals}
        \begin{itemize}
            \item Standard $(1-\alpha)\%$ Normal-based CI: $\hat{\alpha} \pm z_{1-\alpha/2} \cdot SE_B(\hat{\alpha})$. Relies on $\hat{\alpha}$ being approx. Normal and $SE_B(\hat{\alpha})$ being a good estimate.
            \item \textbf{Percentile Confidence Interval}: A more direct method.
                \begin{enumerate}
                    \item Obtain $B$ bootstrap estimates $\hat{\alpha}^{1}, \dots, \hat{\alpha}^{B}$.
                    \item Sort them: $\hat{\alpha}^{(1)} \le \hat{\alpha}^{(2)} \le \dots \le \hat{\alpha}^{(B)}$.
                    \item A $(1-\alpha)\%$ percentile CI is $[\hat{\alpha}^{(\lfloor B\alpha/2 \rfloor)}, \hat{\alpha}^{(\lceil B(1-\alpha/2) \rceil)}]$.
                    \item E.g., for a 95\% CI ($\alpha=0.05$) with $B=1000$, use the 25th and 975th sorted bootstrap estimates.
                \end{enumerate}
            \item Other types: Bias-Corrected and Accelerated (BCa) intervals (often better but more complex).
            \item R: \Rfunction{boot.ci} function from \Rpackage{boot} package can compute various CIs.
\begin{lstlisting}[caption={Bootstrap Confidence Intervals using boot.ci}]
# Assuming boot_results from logistic regression example above
# conf_intervals <- boot.ci(boot_results, type = c("norm", "perc", "bca"), index = 2) # For 'income' coeff
# print(conf_intervals)
# index=1 for intercept, index=2 for income, index=3 for balance

# For median example:
# boot.ci(boot_median_results, type=c("norm", "perc", "bca"))
\end{lstlisting}
        \end{itemize}

    \subsubsection{When is Bootstrap Useful?}
        \begin{itemize}
            \item When it's hard to derive SEs or CIs analytically (e.g., for medians, quantiles, ratios of parameters, complex model outputs).
            \item For small sample sizes where asymptotic theory for SEs might not hold well.
            \item To check assumptions of simpler SE formulas.
        \end{itemize}
    \subsubsection{Limitations of Bootstrap}
        \begin{itemize}
            \item \textbf{Assumption}: The empirical distribution (from the original sample) is a good approximation of the true population distribution. This may not hold if the sample is very small or not representative.
            \item \textbf{Extreme Quantiles/Parameters}: Bootstrap may perform poorly for estimating properties related to the tails of a distribution if the original sample doesn't capture those tails well.
            \item \textbf{Computational Cost}: Can be intensive if $B$ is large and the statistic computation is complex.
            \item \textbf{Not a substitute for more data}: Bootstrap quantifies uncertainty based on the current sample; it doesn't improve the point estimate itself in the way more data would.
            \item \textbf{Dependence}: Standard bootstrap assumes i.i.d. observations. Modifications exist for time series (e.g., block bootstrap) or dependent data, but are more complex.
        \end{itemize}
    \subsubsection{Conceptual Exercise Insights }
        \begin{itemize}
            \item \textbf{(a,b)} Probability a specific observation $j$ is not in a bootstrap sample: $(1 - 1/n)$.
            \item \textbf{(c)} Probability observation $j$ is not in a bootstrap sample of size $n$: $(1 - 1/n)^n$.
            \item \textbf{(d,e,f,g)} As $n \to \infty$, $(1 - 1/n)^n \to e^{-1} \approx 0.368$. So, probability observation $j$ is in the bootstrap sample approaches $1 - e^{-1} \approx 0.632$.
                \begin{itemize}
                    \item This means, on average, about 63.2\% of original observations are present in any given bootstrap sample (these are the "in-bag" samples). The remaining ~36.8% are "out-of-bag" (OOB).
                \end{itemize}
\begin{lstlisting}[caption={Probability an Observation is in a Bootstrap Sample}]
pr_in_bootstrap <- function(n) {
  return(1 - (1 - 1/n)^n)
}
n_values <- c(5, 100, 10000)
sapply(n_values, pr_in_bootstrap)
# For n=5:  0.67232
# For n=100: 0.63397
# For n=10000:0.63214
# Converges to 1 - exp(-1)
# 1 - exp(-1) # approx 0.63212
\end{lstlisting}
        \end{itemize}
    \subsubsection{Exercise: ISLR 5.9 (Bootstrap for \Robject{Boston} data)}
        (Corresponds to `ch5-applied.R` Exercise 9)
        \begin{itemize}
            \item (a) Estimate $\hat{\mu} = \text{mean}(\Robject{medv})$.
            \item (b) Estimate $SE(\hat{\mu})$ using formula $s/\sqrt{n}$.
\begin{lstlisting}[caption={Mean and SE of Mean for Boston$medv}]
# library(MASS)
# data(Boston)
# mu_hat_medv <- mean(Boston$medv) # 22.53281
# se_mu_hat_formula <- sd(Boston$medv) / sqrt(nrow(Boston)) # 0.4088611
\end{lstlisting}
            \item (c) Estimate $SE(\hat{\mu})$ using bootstrap.
\begin{lstlisting}[caption={Bootstrap SE of Mean for Boston$medv}]
# library(boot)
# boot.fn_mean <- function(data_vector, index) {
#   return(mean(data_vector[index]))
# }
# set.seed(1)
# boot_mean_results <- boot(Boston$medv, boot.fn_mean, R=1000)
# print(boot_mean_results) # Bootstrap SE is ~0.4119 (close to formula)
\end{lstlisting}
            \item (d) Compare SEs and construct 95\% CI for $\mu$.
                \begin{itemize}
                    \item Bootstrap SE is similar to formula-based SE.
                    \item CI from \Rfunction{t.test(Boston\$medv)}: e.g., [21.73, 23.34].
                    \item Bootstrap CI (e.g., percentile or BCa from \Rfunction{boot.ci}) should be similar.
                \end{itemize}
            \item (e) Estimate $\hat{\mu}_{med} = \text{median}(\Robject{medv})$. (Result: 21.2)
            \item (f) Estimate $SE(\hat{\mu}_{med})$ using bootstrap. (Result SE $\approx 0.38$, shown in earlier R example).
            \item (g) Estimate $\hat{\mu}_{0.1} = \text{10th percentile of } \Robject{medv}$. (Result: 12.75)
            \item (h) Estimate $SE(\hat{\mu}_{0.1})$ using bootstrap.
\begin{lstlisting}[caption={Bootstrap SE for 10th Percentile of Boston$medv}]
# boot.fn_q10 <- function(data_vector, index) {
#   return(quantile(data_vector[index], probs=0.1))
# }
# set.seed(1)
# boot_q10_results <- boot(Boston$medv, boot.fn_q10, R=1000)
# print(boot_q10_results) # Bootstrap SE is ~0.5113
\end{lstlisting}
        \end{itemize}

\subsection{Lecture 7: Linear Model Selection Methods }
    \subsubsection{Introduction to Model Selection}
        \begin{itemize}
            \item Model: $Y = \beta_0 + \beta_1 X_1 + \dots + \beta_p X_p + \epsilon$.
            \item \textbf{Conflicting Goals}:
                \begin{itemize}
                    \item \textit{Prediction Accuracy}: We want a model that predicts well on new, unseen data.
                    \item \textit{Model Interpretability}: We want a simpler model that is easier to understand and explain.
                \end{itemize}
            \item \textbf{Challenge with many predictors ($p$ large relative to $n$)}:
                \begin{itemize}
                    \item Potentially more information about $Y \implies$ better predictions.
                    \item Estimating many parameters $\beta_j \implies$ high estimation uncertainty (variance) $\implies$ poor predictions.
                    \item This leads to a trade-off:
                        \begin{itemize}
                            \item \textit{Prediction Bias}: Due to an incorrectly specified model (e.g., omitting important variables).
                            \item \textit{Prediction Variance}: Due to estimating too many parameters from limited data.
                        \end{itemize}
                \end{itemize}
            \item \textbf{Goal of Model Selection}: Find a model that optimally balances this bias-variance trade-off to achieve good prediction accuracy and/or interpretability.
        \end{itemize}

    \subsubsection{Three Types of Model Selection Techniques }
        \begin{enumerate}
            \item \textbf{Subset Selection}: Identify a subset of the $p$ predictors that we believe are most relevant to the response. Then fit a model using OLS on this reduced set.
                \begin{itemize}
                    \item Examples: Best Subset Selection, Forward Stepwise, Backward Stepwise.
                \end{itemize}
            \item \textbf{Shrinkage (Regularization)}: Fit a model involving all $p$ predictors, but the estimated coefficients are constrained or "shrunk" towards zero relative to the OLS estimates. This reduces variance.
                \begin{itemize}
                    \item Examples: Ridge Regression, Lasso. (Covered in Lecture 8)
                \end{itemize}
            \item \textbf{Dimension Reduction}: Project the $p$ predictors into an $M$-dimensional subspace, where $M < p$. Then fit a linear regression model using these $M$ projections as predictors.
                \begin{itemize}
                    \item Examples: Principal Components Regression (PCR), Partial Least Squares (PLS).
                \end{itemize}
        \end{enumerate}

    \subsubsection{Best Subset Selection }
        \begin{itemize}
            \item \textbf{Algorithm} (Slide L6 p.4):
                \begin{enumerate}
                    \item Let $\mathcal{M}_0$ be the null model (intercept only, no predictors): $Y = \beta_0 + \epsilon$.
                    \item For $k = 1, 2, \dots, p$:
                        \begin{itemize}
                            \item Fit all $\binom{p}{k}$ models that contain exactly $k$ predictors.
                            \item Pick the best among these $\binom{p}{k}$ models based on some criterion (e.g., highest $R^2$ or lowest RSS for a fixed $k$). Call this model $\mathcal{M}_k$.
                        \end{itemize}
                    \item Select the single best model from $\mathcal{M}_0, \mathcal{M}_1, \dots, \mathcal{M}_p$ using a criterion that accounts for model complexity/overfitting (e.g., Adjusted $R^2$, $C_p$, BIC, AIC, or cross-validated test MSE). Cannot use raw $R^2$ or RSS for this final step as they always improve with more predictors.
                \end{enumerate}
            \item \textbf{Computational Cost}: Involves fitting $2^p$ models. If $p=50$, this is $>10^{15}$ models, which is computationally infeasible. Generally feasible for $p \le 30-40$.
            \item \textbf{R Example (\Rpackage{ISLR} \Robject{Auto} data)} (Slides L6 p.5-11, using $p=6$ predictors):
                \begin{itemize}
                    \item Predictors: cylinders, displacement, horsepower, weight, acceleration, age (derived from year).
                    \item \textit{Full Model} (Slide L6 p.6): \Rcode{lm(mpg ~ ., data=Auto_modified)}. Adjusted $R^2 = 0.8063$.
                    \item \textit{Step 1: Fit $\mathcal{M}_0$} (Slide L6 p.7): \Rcode{M0=lm(y~1)}. RSS for $\mathcal{M}_0$ is TSS.
                    \item \textit{Step 2: Find best model $\mathcal{M}_k$ for each $k$} (Slide L6 p.8 for $k=2$):
                        For $k=2$, the model with \Rcode{weight} and \Rcode{age} has the lowest RSS (4568.95) among all 2-predictor models.
                    \item \textit{Step 3: Select overall best model} (Slide L6 p.9): Using Adjusted $R^2$, the model with \Rcode{weight} and \Rcode{age} ($\mathcal{M}_2$) has the highest Adj $R^2 = 0.8072$.
\begin{lstlisting}[caption={Best Subset Selection with leaps::regsubsets (Slides L6 p.10-11)}]
# require(ISLR)
# Auto$age <- 83 - Auto$year # Create age variable
# Auto_subset <- Auto[, !(names(Auto) %in% c("name", "origin", "year"))] # Select relevant columns

# require(leaps)
# regfit.full <- regsubsets(mpg ~ ., data=Auto_subset, nvmax = 6) # nvmax = p
# reg.summary <- summary(regfit.full)
# print(reg.summary)
# names(reg.summary) # Shows "which", "rsq", "rss", "adjr2", "cp", "bic"

# Results from summary (Slide L6 p.10)
# cbind(reg.summary$which[,-1], adjR2=round(reg.summary$adjr2,4))
# Shows which variables are in the best model of each size, and its AdjR2.
# ##   cylinders displacement horsepower weight acceleration age  adjR2
# ## 1     FALSE        FALSE      FALSE   TRUE        FALSE FALSE 0.6918 (weight)
# ## 2     FALSE        FALSE      FALSE   TRUE        FALSE  TRUE 0.8072 (weight, age)
# ## 3     FALSE        FALSE      FALSE   TRUE         TRUE  TRUE 0.8071 (weight, acc, age)
# ## ...

# plot(regfit.full, scale="adjr2", col=gray.colors(10)) # Visual (Slide L6 p.11)
# The plot shows that AdjR2 peaks for the model with 'weight' and 'age'.
\end{lstlisting}
                \end{itemize}
            \item \textbf{Example with 3 predictors ($X_1, X_2, X_3$)} (Slides L6 p.18-19):
                \begin{itemize}
                    \item $\mathcal{M}_0: Y = \beta_0 + \epsilon$
                    \item $\mathcal{M}_1$: Best of $\{X_1\}, \{X_2\}, \{X_3\}$ (based on $R^2$)
                    \item $\mathcal{M}_2$: Best of $\{X_1,X_2\}, \{X_1,X_3\}, \{X_2,X_3\}$ (based on $R^2$)
                    \item $\mathcal{M}_3: Y = \beta_0 + \beta_1 X_1 + \beta_2 X_2 + \beta_3 X_3 + \epsilon$
                    \item Choose among $\mathcal{M}_0, \mathcal{M}_1, \mathcal{M}_2, \mathcal{M}_3$ using Adj $R^2$, $C_p$, BIC, or CV test MSE.
                    \item Total models estimated: $2^3 = 8$. (Slide L6 p.20)
                \end{itemize}
        \end{itemize}

    \subsubsection{Stepwise Selection Methods }
        \begin{itemize}
            \item Computationally cheaper alternatives to best subset selection, especially for large $p$.
            \item \textbf{Forward Stepwise Selection} (Slides L6 p.12):
                \begin{enumerate}
                    \item Start with null model $\mathcal{M}_0$ (intercept only).
                    \item For $k=0, \dots, p-1$:
                        \begin{itemize}
                            \item Consider all $p-k$ models that add one additional predictor to $\mathcal{M}_k$.
                            \item Choose the best among these $p-k$ models (e.g., one with lowest RSS or highest $R^2$). Call this $\mathcal{M}_{k+1}$.
                        \end{itemize}
                    \item Select the best model among $\mathcal{M}_0, \dots, \mathcal{M}_p$ using $C_p$, BIC, Adj $R^2$, or CV.
                    \item Total models: $1 + \sum_{k=0}^{p-1} (p-k) = 1 + p(p+1)/2$. For $p=50$, this is $1+1275=1276$ models.
                \end{enumerate}
            \item \textbf{R Example (Forward Stepwise for \Robject{Auto} data)} (Slides L6 p.13-14):
\begin{lstlisting}[caption={Forward Stepwise with leaps::regsubsets (Slides L6 p.13)}]
# regfit.fwd <- regsubsets(mpg ~ ., data=Auto_subset, nvmax = 6, method="forward")
# summary.fwd <- summary(regfit.fwd)
# cbind(summary.fwd$which[,-1], adjR2=round(summary.fwd$adjr2,4))
# # Results are identical to Best Subset for this Auto example
# plot(regfit.fwd, scale="adjr2", col=gray.colors(10))
# Number of models estimated: 1 (null) + 6+5+4+3+2+1 = 22. Slide says 21, likely not counting M0 in sum.
\end{lstlisting}
            \item \textbf{Backward Stepwise Selection} (Slides L6 p.15):
                \begin{enumerate}
                    \item Start with full model $\mathcal{M}_p$ (all $p$ predictors).
                    \item For $k=p, p-1, \dots, 1$:
                        \begin{itemize}
                            \item Consider all $k$ models that remove one predictor from $\mathcal{M}_k$.
                            \item Choose the best among these $k$ models (e.g., lowest RSS or highest $R^2$). Call this $\mathcal{M}_{k-1}$.
                        \end{itemize}
                    \item Select the best model among $\mathcal{M}_0, \dots, \mathcal{M}_p$ using $C_p$, BIC, Adj $R^2$, or CV.
                    \item Total models: $1 + \sum_{k=1}^{p} k = 1 + p(p+1)/2$. Requires $n>p$ for initial full model.
                \end{enumerate}
            \item \textbf{R Example (Backward Stepwise for \Robject{Auto} data)} (Slides L6 p.16-17):
\begin{lstlisting}[caption={Backward Stepwise with leaps::regsubsets (Slides L6 p.16)}]
# regfit.bwd <- regsubsets(mpg ~ ., data=Auto_subset, nvmax = 6, method="backward")
# summary.bwd <- summary(regfit.bwd)
# cbind(summary.bwd$which[,-1], adjR2=round(summary.bwd$adjr2,4))
# # Results are identical to Best Subset and Forward for this Auto example
# plot(regfit.bwd, scale="adjr2", col=gray.colors(10))
\end{lstlisting}
            \item \textbf{Hybrid Approaches}: Combine forward and backward steps (e.g., add predictors then remove some if they become non-significant).
            \item Stepwise methods are not guaranteed to find the true best model out of all $2^p$ possibilities but are often good approximations and computationally efficient.
        \end{itemize}

    \subsubsection{Choosing the Optimal Model (Criteria)}
        \begin{itemize}
            \item After generating a sequence of models $\mathcal{M}_0, \dots, \mathcal{M}_p$ (either from best subset or stepwise), we need to select one.
            \item Training error (RSS, $R^2$) is not suitable as it always improves with more variables.
            \item We need to estimate test error. Two approaches:
                \begin{enumerate}
                    \item \textbf{Adjusting Training Error for Model Size}: Add a penalty for number of predictors ($d$).
                        \begin{itemize}
                            \item \textbf{Mallow's $C_p$} (Slide L6 p.21): For OLS with estimated $\hat{\sigma}^2$ from full model.
                                $$ C_p = \frac{1}{n}(RSS_d + 2d\hat{\sigma}^2) $$
                                Choose model with smallest $C_p$. If $C_p \approx d$, model has low bias.
                            \item \textbf{Akaike Information Criterion (AIC)} (Slide L6 p.21): Proportional to
                                $$ AIC \propto \frac{1}{n\hat{\sigma}^2}(RSS_d + 2d\hat{\sigma}^2) $$
                                (Derived from maximizing log-likelihood minus a penalty $2d$). Choose model with smallest AIC. For linear models with Gaussian errors, $C_p$ and AIC are proportional.
                            \item \textbf{Bayesian Information Criterion (BIC)} (Slide L6 p.21):
                                $$ BIC = \frac{1}{n}(RSS_d + \log(n)d\hat{\sigma}^2) $$
                                (For linear models with Gaussian errors, assuming $\hat{\sigma}^2$ is error variance). BIC penalizes model size more heavily than AIC/Cp for $n>7$. Tends to select smaller models. Choose model with smallest BIC.
                            \item \textbf{Adjusted $R^2$} (Slide L6 p.21):
                                $$ R^2_{adj} = 1 - \frac{RSS_d/(n-d-1)}{TSS/(n-1)} $$
                                Penalizes for adding useless predictors. Choose model with largest Adj $R^2$.
                        \end{itemize}
                    \item \textbf{Direct Estimation of Test Error using Cross-Validation}:
                        \begin{itemize}
                            \item For each model size $k=0, \dots, p$, get the best $k$-variable model $\mathcal{M}_k$.
                            \item Compute its CV test error (e.g., 5-fold or 10-fold CV).
                            \item Select $k$ that minimizes CV test error. This is often considered the most direct and reliable approach if computationally feasible.
                            \item "One-standard-error rule": Choose the simplest model whose CV error is within one standard error of the minimum CV error.
                        \end{itemize}
                \end{itemize}
        \end{itemize}
    \subsubsection{Exercise: ISLR 6.8 
    }
        (Refers to `ex6.8.R`)
        \begin{itemize}
            \item (a) Generate data: $X \sim N(0,1)$, $\epsilon \sim N(0,1)$, $Y = \beta_0 + \beta_1 X + \dots + \beta_p X^p + \epsilon$.
                For this exercise, data is simulated for $p=100$ predictors, $n=1000$ obs, where some $\beta_j$ are zero. $Y = X\beta + \epsilon$.
            \item (b) Split into train (100 obs) and test (900 obs).
            \item (c) Best subset selection on training data, plot training MSE vs. subset size. Training MSE will decrease monotonically.
            \item (d) Plot test MSE vs. subset size. Will show a U-shape, identifying optimal number of predictors for test performance.
            \item (e) Find best model size based on test MSE.
            \item (f) Compare coefficients of best model to true $\beta$s.
            \item (g) Plot error $\sqrt{\sum(\hat{\beta}_j - \beta_j)^2}$ vs. subset size. This measures how well coefficients are estimated. The minimum may not align with minimum test MSE.
            \item \textit{Key R functions for ISLR 6.8}: \Rfunction{rnorm}, \Rfunction{matrix}, \Rfunction{sample}, \Rpackage{leaps}::\Rfunction{regsubsets}, \Rfunction{coef}, \Rfunction{predict.regsubsets} (custom function usually needed for `regsubsets` as it doesn't have a standard `predict` method for new data directly).
\begin{lstlisting}[caption={Conceptual predict function for regsubsets (ISLR Ch6 Lab)}]
# predict.regsubsets <- function(object, newdata, id, ...) {
#   form <- as.formula(object$call[[2]]) # Get formula from regsubsets object
#   mat <- model.matrix(form, newdata)   # Create model matrix for new data
#   coefi <- coef(object, id=id)         # Get coefficients for model of size 'id'
#   xvars <- names(coefi)
#   pred <- mat[, xvars] %% coefi        # Matrix multiplication
#   return(pred)
# }
\end{lstlisting}
        \end{itemize}
\subsection{Lecture 8: Regularization (Shrinkage) Methods and PCR }
    \subsubsection{Introduction to Shrinkage Methods }
        \begin{itemize}
            \item Recall: Subset selection methods select a subset of predictors and fit OLS.
            \item \textbf{Shrinkage Methods}: Use all $p$ predictors, but constrain or "shrink" the estimated coefficients $\hat{\beta}_j$ towards zero (or each other).
            \item \textbf{Goal}: Reduce variance at the cost of a small increase in bias, potentially leading to better prediction accuracy (lower test MSE).
            \item \textbf{OLS objective}: Minimize $RSS(\boldsymbol{\beta}) = \sum_{i=1}^{n} (y_i - \mathbf{x}_i'\boldsymbol{\beta})^2$.
            \item \textbf{Shrinkage methods add a penalty term} to the RSS:
                \begin{itemize}
                    \item Ridge Regression: Minimize $RSS(\boldsymbol{\beta}) + \lambda \sum_{j=1}^{p} \beta_j^2$.
                    \item LASSO: Minimize $RSS(\boldsymbol{\beta}) + \lambda \sum_{j=1}^{p} |\beta_j|$.
                \end{itemize}
            \item The penalty term discourages large coefficient values.
        \end{itemize}

    \subsubsection{Ridge Regression }
        \begin{itemize}
            \item \textbf{Objective Function} (Slide L7 p.3):
                $$ \min_{\boldsymbol{\beta}} \left\{ \sum_{i=1}^{n} \left(y_i - \beta_0 - \sum_{j=1}^{p} \beta_j x_{ij}\right)^2 + \lambda \sum_{j=1}^{p} \beta_j^2 \right\} $$
                The term $\lambda \sum_{j=1}^{p} \beta_j^2$ is the $L_2$ penalty or shrinkage penalty.
            \item \textbf{Tuning Parameter $\lambda \ge 0$} (Slide L7 p.3):
                \begin{itemize}
                    \item Controls the amount of shrinkage.
                    \item $\lambda = 0$: Penalty has no effect. Ridge estimates = OLS estimates.
                    \item $\lambda \to \infty$: Shrinkage penalty dominates. Coefficients $\hat{\beta}_j^{\text{ridge}} \to 0$ (for $j=1, \dots, p$). Model approaches intercept-only model.
                    \item $\beta_0$ (intercept) is typically not penalized.
                \end{itemize}
            \item \textbf{Equivalent Formulation} (Constrained optimization, Slide L7 p.4):
                $$ \min_{\boldsymbol{\beta}} \left\{ \sum_{i=1}^{n} \left(y_i - \beta_0 - \sum_{j=1}^{p} \beta_j x_{ij}\right)^2 \right\} \quad \text{subject to} \quad \sum_{j=1}^{p} \beta_j^2 \le s $$
                There is a one-to-one correspondence between $\lambda$ and $s$.
            \item \textbf{Standardization of Predictors} (Slides L7 p.4-5):
                \begin{itemize}
                    \item OLS estimates are scale equivariant (if $X_j$ is multiplied by $c$, $\hat{\beta}_j^{OLS}$ is divided by $c$).
                    \item Ridge regression estimates are \textit{not} scale equivariant because the penalty $\lambda \sum \beta_j^2$ treats all $\beta_j$ equally, regardless of the scale of $X_j$.
                    \item \textbf{Crucial}: Standardize predictors to have mean 0 and standard deviation 1 before fitting Ridge:
                        $$ \tilde{x}_{ij} = \frac{x_{ij} - \bar{x}_j}{s_j} $$
                        (Slide L7 p.5 shows $\tilde{x}_{ij} = x_{ij} / \sqrt{\frac{1}{n}\sum(x_{ij}-\bar{x})^2}$, which is scaling to have RMS of 1, effectively standardizing if mean is 0. Standard practice is to scale to unit variance.)
                \end{itemize}
            \item \textbf{Effect on Bias and Variance} (Slide L7 p.3):
                \begin{itemize}
                    \item Ridge regression reduces variance compared to OLS, especially when predictors are collinear.
                    \item It introduces some bias (coefficients are shrunk from OLS values).
                    \item Can lead to lower Test MSE if the reduction in variance outweighs the increase in squared bias.
                \end{itemize}
            \item \textbf{R Implementation (\Rpackage{glmnet} package)} (Slides L7 p.9-12):
                \begin{itemize}
                    \item \Rfunction{glmnet} function. Set \Rcode{alpha=0} for Ridge.
                    \item Requires predictors \Robject{X} as a matrix and response \Robject{y} as a vector.
                    \item \Rfunction{cv.glmnet} performs cross-validation to select optimal $\lambda$.
\begin{lstlisting}[caption={Ridge Regression Example (Simulated Data, Slides L7 p.6-12)}]
# Simulated data from slides L7 p.6
# set.seed(1234567)
# n <- 100
# p_vars <- 10
# X_sim <- matrix(0, n, p_vars)
# for(j in 1:p_vars) X_sim[,j] <- rnorm(n, sd=j) # Predictors with increasing variance
# beta_sim <- 2:(p_vars+1)
# e_sim <- rnorm(n, sd=200)
# y_sim <- 1 + X_sim %% beta_sim + e_sim

# OLS fit (Slide L7 p.8)
# ols.fit <- lm(y_sim ~ X_sim)
# summary(ols.fit) # Shows some coeffs not significant, high RSE

library(glmnet)
# Ridge with small lambda (Slide L7 p.9)
# ridge.fit_small_lambda <- glmnet(X_sim, y_sim, alpha=0, lambda=0.01)
# cbind(coef(ols.fit), coef(ridge.fit_small_lambda)) # Coeffs very similar

# Ridge with large lambda (Slide L7 p.10)
# ridge.fit_large_lambda <- glmnet(X_sim, y_sim, alpha=0, lambda=1000000)
# cbind(coef(ols.fit), coef(ridge.fit_large_lambda)) # Ridge coeffs shrunk close to 0

# Choosing lambda by LOOCV (Slide L7 p.11)
# cv.ridge <- cv.glmnet(X_sim, y_sim, alpha=0, nfolds=n) # nfolds=n for LOOCV
# lambda_min_ridge <- cv.ridge$lambda.min # e.g., 28.27952 from slide
# ridgemin.fit <- glmnet(X_sim, y_sim, alpha=0, lambda=lambda_min_ridge)
# cbind(coef(ols.fit), coef(ridgemin.fit)) # Shows shrunk coefficients

# Test predictions (Slide L7 p.12)
# # Assume xy_df created, split into train_df, test_df, Xtrain, ytrain, Xtest, ytest
# olstrain.fit <- lm(y ~ ., data=train_df) # Using original formula style with data.frame
# predols_test <- predict(olstrain.fit, newdata=test_df)
# ols_MSE_test <- mean((ytest - predols_test)^2) # e.g., 53559.83

# ridgetrain.fit <- glmnet(Xtrain, ytrain, alpha=0, lambda=lambda_min_ridge)
# predridge_test <- predict(ridgetrain.fit, newx=Xtest)
# ridge_MSE_test <- mean((ytest - predridge_test)^2) # e.g., 52444.16 (better than OLS)
\end{lstlisting}
                \end{itemize}
        \end{itemize}

    \subsubsection{The Lasso (Least Absolute Shrinkage and Selection Operator)}
        \begin{itemize}
            \item \textbf{Objective Function} (Slide L7 p.13):
                $$ \min_{\boldsymbol{\beta}} \left\{ \sum_{i=1}^{n} \left(y_i - \beta_0 - \sum_{j=1}^{p} \beta_j x_{ij}\right)^2 + \lambda \sum_{j=1}^{p} |\beta_j| \right\} $$
                The term $\lambda \sum_{j=1}^{p} |\beta_j|$ is the $L_1$ penalty.
            \item \textbf{Key Property}: Unlike Ridge, the $L_1$ penalty can force some coefficient estimates to be \textit{exactly zero} if $\lambda$ is large enough.
            \item This means Lasso performs \textbf{variable selection}.
            \item \textbf{Tuning Parameter $\lambda \ge 0$} (Slide L7 p.13):
                \begin{itemize}
                    \item $\lambda = 0$: Lasso estimates = OLS estimates.
                    \item $\lambda \to \infty$: All coefficients $\hat{\beta}_j^{\text{lasso}} \to 0$.
                    \item As $\lambda$ increases, more coefficients are set to zero.
                \end{itemize}
            \item \textbf{Equivalent Formulation} (Constrained optimization):
                $$ \min_{\boldsymbol{\beta}} \left\{ \sum_{i=1}^{n} \left(y_i - \beta_0 - \sum_{j=1}^{p} \beta_j x_{ij}\right)^2 \right\} \quad \text{subject to} \quad \sum_{j=1}^{p} |\beta_j| \le s $$
            \item Like Ridge, predictors should be standardized before applying Lasso.
            \item \textbf{R Implementation (\Rpackage{glmnet} package)} (Slides L7 p.14-15):
                \begin{itemize}
                    \item Use \Rfunction{glmnet} with \Rcode{alpha=1}.
                    \item \Rfunction{cv.glmnet} for choosing $\lambda$.
\begin{lstlisting}[caption={Lasso Example (Simulated Data, Slides L7 p.14-15)}]
# Using X_sim, y_sim from Ridge example
# cv.lasso <- cv.glmnet(X_sim, y_sim, alpha=1, nfolds=n) # nfolds=n for LOOCV
# lambda_min_lasso <- cv.lasso$lambda.min # e.g., 13.12619 from slide
# lassomin.fit <- glmnet(X_sim, y_sim, alpha=1, lambda=lambda_min_lasso)
# cbind(coef(ols.fit), coef(ridgemin.fit), coef(lassomin.fit))
# # Lasso sets X1, X3, X4 to zero (coefficients are '.') in slide example

# Test predictions (Slide L7 p.15)
# # Using Xtrain, ytrain, Xtest, ytest from Ridge example
# lassotrain.fit <- glmnet(Xtrain, ytrain, alpha=1, lambda=lambda_min_lasso)
# predlasso_test <- predict(lassotrain.fit, newx=Xtest)
# lasso_MSE_test <- mean((ytest - predlasso_test)^2) # e.g., 51258.12 (best of OLS, Ridge, Lasso)
\end{lstlisting}
                \end{itemize}
        \end{itemize}

    \subsubsection{Comparing Lasso and Ridge }
        \begin{itemize}
            \item \textbf{Variable Selection}: Lasso can produce sparse models (some $\beta_j=0$), Ridge cannot (all $\beta_j$ are non-zero unless $\lambda=\infty$).
            \item \textbf{Performance}:
                \begin{itemize}
                    \item If many predictors have small/moderate effects, Ridge might perform better (keeps all variables).
                    \item If a smaller number of predictors have substantial effects and others are negligible, Lasso might perform better (by setting negligible ones to zero).
                \end{itemize}
            \item \textbf{Geometric Interpretation} (ISLR Fig 6.7):
                \begin{itemize}
                    \item Ridge constraint $\sum \beta_j^2 \le s$ is a circle (2D) or hypersphere. OLS solution contours (ellipses) are less likely to hit an axis exactly.
                    \item Lasso constraint $\sum |\beta_j| \le s$ is a diamond (2D) or rhomboid. OLS solution contours are more likely to hit a corner, making a coefficient zero.
                \end{itemize}
            \item \textbf{Elastic Net}: A compromise between Ridge and Lasso, penalty $\alpha \sum |\beta_j| + (1-\alpha) \sum \beta_j^2$. (\Rfunction{glmnet} can fit this by setting $0 < \alpha < 1$).
        \end{itemize}

    \subsubsection{Selecting the Tuning Parameter $\lambda$ }
        \begin{itemize}
            \item Cross-validation (typically k-fold, e.g., 10-fold) is the standard method.
            \item For a grid of $\lambda$ values:
                \begin{enumerate}
                    \item For each $\lambda$, perform k-fold CV.
                    \item Calculate average test MSE (or other error metric) across folds for that $\lambda$.
                    \item Choose $\lambda$ that results in the lowest average CV test error (\Rcode{lambda.min} from \Rfunction{cv.glmnet}).
                    \item Often, \Rcode{lambda.1se} is also considered: largest $\lambda$ such that error is within one standard error of the minimum. This gives a more parsimonious model with similar performance.
                \end{itemize}

        \end{itemize}

    \subsubsection{Dimension Reduction Methods}
        \begin{itemize}
            \item Transform predictors $X_1, \dots, X_p$ into a new set of $M < p$ predictors $Z_1, \dots, Z_M$. Then fit OLS using $Z_m$.
            \item $Z_m = \sum_{j=1}^p \phi_{jm} X_j$ (linear combinations). (Slide L7 p.16)
            \item Two main methods: Principal Components Regression (PCR) and Partial Least Squares (PLS).
        \end{itemize}

    \subsubsection{Principal Components Analysis (PCA) and Regression (PCR) }
        \begin{itemize}
            \item \textbf{Principal Component Analysis (PCA)} (Slides L7 p.17-18):
                \begin{itemize}
                    \item A technique for dimension reduction that finds directions (principal components, PCs) in the data that capture the most variance.
                    \item $Z_1$ (first PC): Linear combination of $X_j$s that has maximal variance.
                    \item $Z_2$ (second PC): Linear combination of $X_j$s, uncorrelated with $Z_1$, that has maximal variance among all such combinations.
                    \item And so on, up to $p$ PCs.
                    \item The $\phi_{jm}$ (loadings) define the components.
                    \item Predictors should be standardized before PCA if on different scales.
                \end{itemize}
            \item \textbf{Principal Components Regression (PCR)}:
                \begin{enumerate}
                    \item Perform PCA on the $p$ predictors to get $Z_1, \dots, Z_p$.
                    \item Select the first $M < p$ principal components. $M$ is a tuning parameter chosen by CV.
                    \item Fit OLS regression of $Y$ on $Z_1, \dots, Z_M$.
                \end{enumerate}
            \item PCA is an unsupervised method (does not use $Y$ to find components). PCR becomes supervised when $Y$ is regressed on $Z_m$.
            \item PCR can reduce variance by using fewer components, especially if early PCs capture most of the signal related to $Y$.
            \item If most variation in predictors can be seen in a few PCs, we reduce overfitting. (Slide L7 p.18)
            \item \textbf{R Implementation (\Rpackage{pls} package)} (Slides L7 p.19-21):
\begin{lstlisting}[caption={PCR Example (Auto Data, Slides L7 p.19, 21)}]
library(pls)
library(ISLR)
# Auto2 <- Auto[,!(names(Auto) %in% c("name","origin","year"))] # From slide L6 p.6
# Auto2$age <- 83 - Auto$year # From slide L6 p.6
# set.seed(1) # Assuming a seed for train/test split
# n_auto <- nrow(Auto2)
# train_idx_auto <- sample(1:n_auto, floor(n_auto/2))
# train_auto <- Auto2[train_idx_auto,]
# test_auto <- Auto2[-train_idx_auto,]

# Fit PCR on training data, choose M using CV (built-in)
# pcr.fit <- pcr(mpg ~ ., data=train_auto, scale=TRUE, validation="CV")
# summary(pcr.fit) # Shows % variance explained by X and mpg for different M
# validationplot(pcr.fit, val.type="MSEP") # Helps choose M

# Example output from summary(pcr.fit) on full Auto2 (Slide L7 p.19):
# TRAINING: % variance explained
#        1 comps 2 comps 3 comps 4 comps 5 comps 6 comps
# X        99.76   99.96  100.00  100.00  100.00  100.00
# mpg      69.35   70.09   70.75   80.79   80.88   80.93
# First PC explains 99.76% of variance in X's, and 69.35% of variance in mpg.

# Training/test evaluation (Slide L7 p.21 - assuming M=4 chosen from CV)
# regall.fit <- lm(mpg ~ ., data=train_auto)
# pcr.fit_M4 <- pcr(mpg ~ ., ncomp=4, data=train_auto, scale=TRUE)
#
# predall_test <- predict(regall.fit, newdata=test_auto)
# MSEall_test <- mean((test_auto$mpg - predall_test)^2) # e.g., 11.47
#
# predpcr_test <- predict(pcr.fit_M4, newdata=test_auto)
# MSEpcr_test <- mean((test_auto$mpg - predpcr_test)^2) # e.g., 15.60
# In this specific split on slide L7, OLS performed better than PCR with M=4.
# Choice of M from CV on training data is crucial.
\end{lstlisting}
        \end{itemize}
    \subsubsection{Partial Least Squares (PLS) }
        \begin{itemize}
            \item Similar to PCR, but finds components $Z_m$ in a supervised way.
            \item $Z_1 = \sum \phi_{j1} X_j$ where $\phi_{j1}$ are coefficients of simple linear regression of $Y$ on each $X_j$. $Z_1$ is the linear combination most correlated with $Y$.
            \item $Z_2$ is found by first regressing $Y$ on $Z_1$ and taking residuals. Then $Z_2$ is the linear combination of $X_j$s (after orthogonalizing them w.r.t $Z_1$) that best explains these residuals.
            \item Can sometimes outperform PCR, especially if predictors are highly collinear or if the response is strongly related to directions with lower variance in $X$.
            \item Number of components $M$ chosen by CV.
            \item R: \Rfunction{plsr} in \Rpackage{pls} package.
        \end{itemize}

    \subsubsection{Exercise: ISLR 6.11 }
        (Refers to `ex6.11.R`)
        \begin{itemize}
            \item Goal: Compare Best Subset, Lasso, Ridge, PCR for predicting per capita crime rate (\Robject{crim}).
            \item Setup: Split \Robject{Boston} data into training/test. Prepare \Robject{X} matrix and \Robject{y} vector.
            \item (a) Best Subset Selection:
                \begin{itemize}
                    \item Use k-fold CV (e.g., 10-fold) to select optimal number of predictors.
                    \item Need a `predict` method for `regsubsets` (often custom written).
                    \item Fit best model of chosen size on full training set, evaluate on test set.
                    \item \textit{From `ex6.11.R`}: 10-fold CV suggests 9 predictors. Test RMSE $\approx 6.59$.
                \end{itemize}
            \item (a) Lasso:
                \begin{itemize}
                    \item Use \Rfunction{cv.glmnet} on training data to find optimal $\lambda$.
                    \item Fit Lasso model with optimal $\lambda$ on training data.
                    \item Predict on test data, calculate test RMSE.
                    \item \textit{From `ex6.11.R`}: Test RMSE (using $\lambda_{1se}$) $\approx 7.405$.
                \end{itemize}
            \item (a) Ridge Regression:
                \begin{itemize}
                    \item Use \Rfunction{cv.glmnet} (\Rcode{alpha=0}) on training data for $\lambda$.
                    \item Fit Ridge model with optimal $\lambda$ on training data.
                    \item Predict on test data, calculate test RMSE.
                    \item \textit{From `ex6.11.R`}: Test RMSE (using $\lambda_{1se}$) $\approx 7.457$.
                \end{itemize}
            \item (a) PCR:
                \begin{itemize}
                    \item Use \Rfunction{pcr} with \Rcode{validation="CV"} on training data to choose number of components $M$.
                    \item Fit PCR model with chosen $M$ on training data.
                    \item Predict on test data, calculate test RMSE.
                    \item \textit{From `ex6.11.R`}: CV suggests $M=13$ components (full model). Test RMSE $\approx 6.546$.
                \end{itemize}
            \item (b) Conclusion: Compare test RMSEs. Best subset and PCR (with many components) performed best in this example.
            \item (c) Are models parsimonious?: Discuss which methods yield simpler models (Lasso, Best Subset if few variables chosen).
        \end{itemize}
\subsection{Lecture 9: Non-linear Models - Polynomials, Splines, Local Regression }
    \subsubsection{Introduction to Non-linear Models }
        \begin{itemize}
            \item Linear models assume a linear relationship between predictors and response, which is often too simplistic.
            \item This lecture explores methods to model non-linear relationships:
                \begin{itemize}
                    \item Polynomial Regression
                    \item Step Functions (Piecewise Constant Regression)
                    \item Regression Splines (Piecewise Polynomials)
                    \item Local Regression
                    \item Smoothing Splines (covered briefly, more detail in GAMs)
                \end{itemize}
            \item Generalized Additive Models (GAMs) provide a framework to use these for multiple predictors (covered in Lecture 10).
        \end{itemize}

    \subsubsection{Polynomial Regression }
        \begin{itemize}
            \item Extends linear model by adding polynomial terms of predictors:
                $$ y_i = \beta_0 + \beta_1 x_i + \beta_2 x_i^2 + \dots + \beta_d x_i^d + \epsilon_i $$
            \item This is still a linear model in terms of coefficients $\beta_j$, so it can be fit using OLS.
            \item Choice of degree $d$ is critical. Too low $d \implies$ underfitting. Too high $d \implies$ overfitting (wiggly fit).
            \item $d$ often chosen by cross-validation or hypothesis tests (e.g., ANOVA to compare nested models).
            \item R: \Rfunction{lm(y ~ poly(x, degree=d))} or \Rfunction{lm(y ~ x + I(x\textasciicircum{}2) + ...)}. Using \Rfunction{poly} creates orthogonal polynomials, which can be numerically more stable.
        \end{itemize}

    \subsubsection{Step Functions (Piecewise Constant Regression) }
        \begin{itemize}
            \item Divide the range of $X$ into $K+1$ disjoint regions using $K$ cutpoints (knots) $c_1 < c_2 < \dots < c_K$.
            \item Fit a constant (mean of $Y$) in each region:
                $$ y_i = \beta_0 + \beta_1 I(x_i < c_1) + \beta_2 I(c_1 \le x_i < c_2) + \dots + \beta_{K+1} I(x_i \ge c_K) + \epsilon_i $$
                (Slide L8 p.3 shows a slightly different formulation with 5 coefficients for 4 knots, creating 5 regions).
            \item This is achieved by creating dummy variables for each interval.
            \item Choice of number and location of knots is important. Often chosen by CV or domain knowledge.
            \item R Example (Simulated data, Slides L8 p.3-7):
\begin{lstlisting}[caption={Piecewise Constant Regression (Step Function) in R}]
# Simulating data (Slides L8 p.3)
# set.seed(1); n=100; x=runif(n)
# c_knots=seq(0.10,0.90,length=4) # c1, c2, c3, c4
# b_coeffs=c(1,2,3,-1,1) # Beta for intercept and each interval
# y_step = b_coeffs[1] + 
#          b_coeffs[2](x < c_knots[1]) + 
#          b_coeffs[3]((x >= c_knots[1]) & x < c_knots[2]) +
#          b_coeffs[4]((x >= c_knots[2]) & x < c_knots[3]) + # Error in slide formula for b[4]
#          b_coeffs[5]((x >= c_knots[3]) & x < c_knots[4]) + # Error in slide formula for b[5]
#          0.1rnorm(n) # Should be 5 intervals for 5 beta coeffs after intercept if b[1] is overall mean
                     # Or one more interval for x >= c_knots[4] if b[1] is first interval
# A more direct way in R is using cut()
# fit_step <- lm(y ~ cut(x, breaks=c(min(x)-.1, c_knots, max(x)+.1)))
# summary(fit_step)
# The coefficients will be differences from the intercept (mean of first interval).

# Example with Wage data (Slides L8 p.8)
# library(ISLR)
# attach(Wage)
# table(cut(age, 4)) # Divides age into 4 equally spaced intervals
# fit_wage_step <- lm(wage ~ cut(age, 4), data=Wage)
# coef(summary(fit_wage_step))
# # Shows average wage for first age group, and differences for others.
# detach(Wage)
\end{lstlisting}
            \item Drawback: Fit is not continuous, can miss trends at boundaries.
        \end{itemize}

    \subsubsection{Basis Functions }
        \begin{itemize}
            \item General idea: Transform $X$ using a set of known basis functions $b_1(X), b_2(X), \dots, b_K(X)$.
            \item Then fit linear model: $y_i = \beta_0 + \beta_1 b_1(x_i) + \dots + \beta_K b_K(x_i) + \epsilon_i$.
            \item Examples:
                \begin{itemize}
                    \item Polynomial regression: $b_j(x) = x^j$.
                    \item Step functions: $b_j(x) = I(c_j \le x < c_{j+1})$.
                    \item Periodic functions: $b_j(x) = \sin(\alpha_j x)$ or $\cos(\alpha_j x)$.
                \end{itemize}
        \end{itemize}

    \subsubsection{Regression Splines (Piecewise Polynomials) }
        \begin{itemize}
            \item Fit separate low-degree polynomials in different regions defined by knots $\xi_1, \dots, \xi_K$.
            \item \textbf{Constraints for Smoothness} (Slide L8 p.11): To avoid discontinuities, constraints are imposed at the knots.
                \begin{itemize}
                    \item Function must be continuous at knots.
                    \item First derivative continuous (smooth curve).
                    \item Second derivative continuous (smoother curve).
                \end{itemize}
            \item A degree-$d$ spline with $K$ knots has $d+1+K$ degrees of freedom (parameters).
            \item \textbf{Cubic Spline} (degree 3) is common: Continuous, with continuous 1st and 2nd derivatives. Smooth and flexible.
            \item \textbf{Cubic Spline Basis Representation} (Slide L8 p.12):
                A cubic spline with $K$ knots $\xi_1, \dots, \xi_K$ can be modeled using a basis:
                $$ y_i = \beta_0 + \beta_1 x_i + \beta_2 x_i^2 + \beta_3 x_i^3 + \sum_{k=1}^K \beta_{k+3} h(x_i, \xi_k) + \epsilon_i $$
                where $h(x, \xi) = (x-\xi)_+^3 = \begin{cases} (x-\xi)^3 & \text{if } x > \xi \\ 0 & \text{otherwise} \end{cases}$.
                This model has $K+4$ parameters (degrees of freedom = $K+4$).
            \item \textbf{Choosing Knots and Number of Knots (or Degrees of Freedom)}:
                \begin{itemize}
                    \item More knots $\implies$ more flexible, can overfit.
                    \item Knots often placed at uniform quantiles of $X$ or based on domain knowledge.
                    \item Number of knots (or equivalently, degrees of freedom `df`) chosen by cross-validation.
                \end{itemize}
            \item \textbf{Natural Splines} (ISLR Ch 7.4.3): Cubic splines constrained to be linear beyond boundary knots. Reduces variance at boundaries. Uses $K$ df for $K$ knots.
            \item \textbf{R Implementation (\Rpackage{splines} package)} (Slide L8 p.13, using \Rfunction{bs}):
\begin{lstlisting}[caption={Regression Spline with Wage Data (Slide L8 p.13)}]
# library(ISLR)
# library(splines) # For bs() function
# attach(Wage)
# # Fit a cubic spline with knots at ages 25, 40, 60
# fit.spline <- lm(wage ~ bs(age, knots=c(25,40,60)), data=Wage)
# summary(fit.spline) # Shows coefficients for basis functions
# 
# # Plotting the fit
# agelims <- range(age)
# age.grid <- seq(from=agelims[1], to=agelims[2])
# pred.spline <- predict(fit.spline, newdata=list(age=age.grid))
# plot(age, wage, col="lightgrey")
# lines(age.grid, pred.spline, col="black", lwd=2)
# # Add vertical lines for knots
# abline(v=c(25,40,60), lty=2, col="red")
# detach(Wage)
# # Alternatively, specify degrees of freedom (df) instead of knots
# # fit.spline_df <- lm(wage ~ bs(age, df=6), data=Wage) # df=6 for 3 knots cubic spline (df=K+3+1-1=K+3 usually, or K+degree)
# # bs() uses df = K + degree (for cubic, df=K+3). Intercept is separate.
# # So, 3 knots with bs() is df=3+3=6.
\end{lstlisting}
        \end{itemize}

    \subsubsection{Smoothing Splines }
        \begin{itemize}
            \item Goal: Find a function $g(x)$ that fits the data well ($RSS = \sum (y_i - g(x_i))^2$ is small) but is also smooth.
            \item \textbf{Objective Function} (Slide L9 p.3):
                $$ \min_{g} \left\{ \sum_{i=1}^{n} (y_i - g(x_i))^2 + \lambda \int (g''(t))^2 dt \right\} $$
                \begin{itemize}
                    \item First term: Measures closeness to data (RSS).
                    \item Second term: Roughness penalty. $\int (g''(t))^2 dt$ is large if $g(x)$ is "wiggly".
                    \item $\lambda \ge 0$ is a tuning parameter:
                        \begin{itemize}
                            \item $\lambda = 0$: $g(x)$ interpolates data (can be very rough, zero RSS if all $x_i$ unique). (Slide L9 p.2 figure)
                            \item $\lambda \to \infty$: $g''(t)$ must be zero, meaning $g(x)$ is a linear function (OLS line).
                        \end{itemize}
                \end{itemize}
            \item \textbf{Solution}: The function $g(x)$ that minimizes this is a \textit{natural cubic spline} with knots at every unique value of $x_i$. (Slide L9 p.3)
            \item Seems complex (many knots), but $\lambda$ controls effective degrees of freedom ($df_{\lambda}$).
            \item $\lambda$ is chosen by (generalized) cross-validation, often LOOCV.
            \item R: \Rfunction{smooth.spline(x, y, cv=TRUE)} (Slide L9 p.5). \Rcode{cv=TRUE} chooses $\lambda$ by LOOCV.
\begin{lstlisting}[caption={Smoothing Spline with Wage Data (Slide L9 p.5)}]
# library(ISLR)
# attach(Wage)
# fit.smooth <- smooth.spline(age, wage, cv=TRUE) # Chooses lambda by LOOCV
# print(fit.smooth)
# # fit.smooth$df gives effective degrees of freedom, e.g., 6.79
# 
# plot(age, wage, col="lightgrey")
# lines(fit.smooth, col="red", lwd=2) # Red line is CV-chosen smoothing spline
# # Can also specify df to get a specific lambda
# lines(smooth.spline(age, wage, df=16), col="blue", lwd=2) # More wiggly (higher df)
# detach(Wage)
\end{lstlisting}
        \end{itemize}

    \subsubsection{Local Regression (LOESS/LOWESS)}
        \begin{itemize}
            \item Fits separate simple models (e.g., linear or quadratic polynomials) in local neighborhoods of $x_0$.
            \item \textbf{Algorithm at a target point $x_0$}:
                \begin{enumerate}
                    \item Gather $s \cdot n$ training points closest to $x_0$ (the neighborhood). $s$ is the span (e.g., $0.2$ to $0.5$).
                    \item Assign weights to points in this neighborhood, with points closer to $x_0$ getting higher weights (e.g., tricube weight function). Points outside neighborhood get weight 0.
                    \item Fit a weighted least squares regression of $y$ on $x$ using only points in the neighborhood with these weights.
                    \item The fitted value $\hat{y}_0$ at $x_0$ is the prediction.
                \end{enumerate}
            \item This is repeated for all $x_0$ where a prediction is desired.
            \item Span $s$ is the main tuning parameter, controls smoothness. Small $s \implies$ local, wiggly fit. Large $s \implies$ global, smoother fit.
            \item R: \Rfunction{loess(y ~ x, span=..., data=...)}.
        \end{itemize}

    \subsubsection{Kernel Estimators (Nadaraya-Watson)}
        \begin{itemize}
            \item Model $y = f(x) + \epsilon$. (Slides L9 p.6)
            \item \textbf{Nadaraya-Watson Estimator}: A type of local averaging.
                $$ \hat{f}(x_0) = \sum_{i=1}^n w_i(x_0) y_i $$
                where weights $w_i(x_0) = \frac{K\left(\frac{x_i - x_0}{h}\right)}{\sum_{j=1}^n K\left(\frac{x_j - x_0}{h}\right)}$ sum to 1.
            \item $K(\cdot)$ is a \textbf{kernel function}, a non-negative function that integrates to one and is symmetric about zero (e.g., Gaussian kernel, Epanechnikov kernel, boxcar kernel).
                \begin{itemize}
                    \item Gaussian kernel (using standard normal density $\phi$): $K(u) = \phi(u)$. (Slide L9 p.6 shows weights with $\phi$).
                    \item The plot on Slide L9 p.7 shows how $K((x-x_0)/h)$ gives higher weights to $x_i$ close to $x_0$.
                \end{itemize}
            \item $h$ is the \textbf{bandwidth}, a tuning parameter controlling smoothness. Small $h \implies$ local, wiggly. Large $h \implies$ global, smooth. Chosen by CV.
            \item R Example (Slides L8 p.19, implementing N-W with Gaussian kernel):
\begin{lstlisting}[caption={Nadaraya-Watson with Gaussian Kernel (Slides L8 p.19)}]
# K_gauss <- function(x0_val, x_vec, h_val=0.2) {
#   u <- (x_vec - x0_val) / h_val
#   weights_numerator <- dnorm(u) # Using standard normal density
#   return(weights_numerator / sum(weights_numerator))
# }
# 
# nw_estimator <- function(x0_val, x_train, y_train, h_val=0.2) {
#   weights <- K_gauss(x0_val, x_train, h_val)
#   return(sum(weights  y_train))
# }
# 
# # Example data from slides L8 p.16
# # set.seed(1); x_sim_nw <- rnorm(100, sd=20); y_sim_nw <- x_sim_nw2 + rnorm(100, sd=50)
# # plot(x_sim_nw, y_sim_nw)
# # ypred_knn_nw <- sapply(as.matrix(x_sim_nw), knn, x=x_sim_nw, y=y_sim_nw) # Using knn from slide L8 p.17
# # lines(sort(x_sim_nw), ypred_knn_nw[order(x_sim_nw)], col="blue")
# # ypred_nw_h5 <- sapply(as.matrix(x_sim_nw), nw_estimator, 
# #                       x_train=x_sim_nw, y_train=y_sim_nw, h_val=5)
# # lines(sort(x_sim_nw), ypred_nw_h5[order(x_sim_nw)], col="red")
\end{lstlisting}
            \item \textbf{Local Linear Regression} (Slides L8 p.28): Can be seen as an extension of Nadaraya-Watson. Instead of local constant (average), fit a local linear model using weighted least squares, with weights from a kernel.
                $$ \min_{\beta_0, \beta_1} \sum_{i=1}^n K\left(\frac{x_i-x_0}{h}\right) (y_i - \beta_0 - \beta_1(x_i-x_0))^2 $$
                Then $\hat{f}(x_0) = \hat{\beta}_0$. (Slide formula has $\beta_1 x_i$, often $(x_i-x_0)$ is used for stability). LOESS is a form of local polynomial regression.
        \end{itemize}

    \subsubsection{Exercises: ISLR 7.6, 7.9, (7.10, 7.11 for GAMs)}
        \begin{itemize}
            \item \textbf{ISLR 7.6 (\Rpackage{Wage} data)}: (Corresponds to `ex7.6.R`)
                \begin{itemize}
                    \item (a) Polynomial regression for \Rcode{wage ~ age}: Use CV to choose degree. ANOVA for comparison. Plot results.
                        \textit{Result from `ex7.6.R`}: CV suggests degree 3 or 4 (varies with seed/method). ANOVA often suggests degree 3 or 4 as significant improvements over lower degrees.
                    \item (b) Step function for \Rcode{wage ~ age}: Use CV to choose number of cuts. Plot.
                        \textit{Result from `ex7.6.R`}: CV often suggests around 7-9 cuts.
                \end{itemize}
            \item \textbf{ISLR 7.9 (\Rpackage{Boston} data)}: (Corresponds to `ex7.9.R`)
                \begin{itemize}
                    \item (a) Polynomial regression for \Rcode{nox ~ dis}: Plot polynomial fits of various degrees.
                    \item (b) Report training RSS for different degrees. (Decreases monotonically).
                    \item (c) Use CV to select best degree. Plot CV error. (Often degree 3 or 4 is chosen).
                    \item (d) Regression spline with \Rfunction{bs} for \Rcode{nox ~ dis}: Specify knots (e.g., at quartiles of `dis`). Report fit. Plot.
                    \item (e) Regression spline with varying df: Fit for df from 3 to 16. Report training RSS. (Decreases).
                    \item (f) Use CV to select best df for spline. Plot CV error.
                \end{itemize}
        \end{itemize}


\subsection{Lecture 10: Generalized Additive Models (GAMs)}
    \subsubsection{Introduction to GAMs }
        \begin{itemize}
            \item Extends linear models to allow non-linear functions for \textit{each} predictor, while maintaining an additive structure.
            \item Model form:
                $$ y_i = \beta_0 + f_1(x_{i1}) + f_2(x_{i2}) + \dots + f_p(x_{ip}) + \epsilon_i $$
                or, for expected value: $E(Y|X_1, \dots, X_p) = \beta_0 + \sum_{j=1}^p f_j(X_j)$.
            \item Each $f_j(x_{ij})$ can be a smooth non-linear function (e.g., spline, local polynomial) or a linear term.
            \item \textbf{Additive}: Allows examining the effect of each $X_j$ on $Y$ individually, holding other predictors constant, by plotting $\hat{f}_j(x_j)$.
            \item Avoids the "curse of dimensionality" faced by fitting a general non-linear function $f(x_1, \dots, x_p)$ directly in high dimensions.
        \end{itemize}

    \subsubsection{Fitting GAMs: Backfitting Algorithm }
        \begin{itemize}
            \item An iterative procedure to fit GAMs.
            \item \textbf{Example with two predictors ($y = f_1(x_1) + f_2(x_2) + \epsilon$)} (Slide L9 p.9-10):
                \begin{enumerate}
                    \item Initialize: e.g., $\hat{\beta}_0 = \bar{y}$, $\hat{f}_2(x_{i2}) = 0$ for all $i$.
                    \item Iterate until convergence:
                        \begin{itemize}
                            \item Estimate $\hat{f}_1$ by fitting a model to $y_i - \hat{\beta}_0 - \hat{f}_2(x_{i2})$ using $x_{i1}$ (e.g., smoothing spline on $x_1$).
                            \item Estimate $\hat{f}_2$ by fitting a model to $y_i - \hat{\beta}_0 - \hat{f}_1(x_{i1})$ using $x_{i2}$ (e.g., smoothing spline on $x_2$).
                            \item (Often, residuals are centered).
                        \end{itemize}
                    \item This means iteratively fitting each $f_j$ on "partial residuals" $y_i - \hat{\beta}_0 - \sum_{k \neq j} \hat{f}_k(x_{ik})$.
                \end{enumerate}
            \item R: The \Rpackage{gam} package (or \Rpackage{mgcv}) handles this.
        \end{itemize}

    \subsubsection{GAMs for Regression }
        \begin{itemize}
            \item R Example (\Rpackage{ISLR} \Robject{Wage} data, using \Rpackage{gam} package):
\begin{lstlisting}[caption={GAM for Regression with Wage Data (Slide L9 p.11-12)}]
# library(ISLR)
# library(gam) # Older gam package. mgcv is more modern.
# attach(Wage)

# GAM with smoothing splines for year and age (df specified)
# gam.fit1 <- gam(wage ~ s(year, df=3) + s(age, df=3), data=Wage) # Slide uses df=3 for both
# plot.Gam(gam.fit1, terms="s(age, 3)", col="red", se=TRUE) # Slide L9 p.11, example with one term
# par(mfrow=c(1,3)) # To plot multiple terms (Slide L9 p.12)

# GAM including a linear term for education and smoothing splines for year & age
gam.fit2 <- gam(wage ~ s(year, df=4) + s(age, df=5) + education, data=Wage)
# Note: df might be different in slide examples, often chosen by CV or set.
# plot.Gam(gam.fit2, se=TRUE, col="red") # Plots all terms
# par(mfrow=c(1,3)); plot(gam.fit2, se=TRUE, col="red") # Example from ISLR book for Fig 7.12
# (Slide L9 p.12 shows plots for s(year), s(age), education. s() implies smoothing spline)
# The plots show the estimated non-linear effect of year and age, and linear effect of education.
# detach(Wage)
\end{lstlisting}
            \item For factor predictors like `education`, \Rfunction{gam} fits a separate constant for each level (similar to linear regression with dummy variables).
        \end{itemize}

    \subsubsection{GAMs for Classification}
        \begin{itemize}
            \item Extends logistic regression by allowing non-linear functions for predictors.
            \item Model for binary response $Y \in \{0,1\}$ (Slide L9 p.13):
                $$ \log\left(\frac{P(Y=1|X)}{1-P(Y=1|X)}\right) = \beta_0 + f_1(x_1) + \dots + f_p(x_p) $$
            \item Each $f_j$ can be a smooth function.
            \item Fitted using backfitting with weighted least squares for each step (as logistic regression is fit with IRLS).
            \item R Example (\Robject{Wage} data, predict if \Rcode{wage > 250}, Slide L9 p.14):
\begin{lstlisting}[caption={GAM for Classification (Slide L9 p.14)}]
# library(ISLR)
# library(gam)
# attach(Wage)
# gam.log.fit <- gam(I(wage > 250) ~ s(year, df=4) + s(age, df=5) + education,
#                    data=Wage, family=binomial)
# # par(mfrow=c(1,3))
# # plot.Gam(gam.log.fit, se=TRUE, col="red")
# # Shows effect of each predictor on the log-odds of wage > 250.
# detach(Wage)
\end{lstlisting}
        \end{itemize}

    \subsubsection{Pros and Cons of GAMs}
        \begin{itemize}
            \item \textbf{Pros}:
                \begin{itemize}
                    \item Automatically model non-linear relationships without manual specification of transformations.
                    \item Additive nature allows examining effect of each predictor individually.
                    \item More flexible than linear models, but more interpretable than fully non-parametric methods like KNN in high dimensions.
                    \item Can improve predictive accuracy if true relationships are non-linear.
                \end{itemize}
            \item \textbf{Cons}:
                \begin{itemize}
                    \item Primarily additive, so important interactions can be missed unless explicitly included (e.g., `s(x1,x2)` or `te(x1,x2)` in \Rpackage{mgcv}).
                    \item With many predictors, interpretation of many individual non-linear plots can still be complex.
                \end{itemize}
        \end{itemize}
    \subsubsection{Exercises: ISLR 7.10, 7.11}
        \begin{itemize}
            \item \textbf{ISLR 7.10 (\Rpackage{College} data)}: (Corresponds to `ex7.10.R`)
                \begin{itemize}
                    \item (a) Forward stepwise selection for \Rcode{Outstate ~ .}. Identify best model by BIC/Cp/AdjR2. (E.g., 6 variables: Private, Room.Board, Terminal, perc.alumni, Expend, Grad.Rate).
                    \item (b) Fit GAM with selected variables, using smoothing splines (\Rfunction{s}) for continuous predictors. Plot terms.
                    \item (c) Evaluate GAM on test set. Compare MSE with linear model using same predictors.
                    \item (d) Check for non-linear evidence using Anova for GAM terms (`summary(gam.fit)` in \Rpackage{gam} provides this for non-parametric effects).
                    \item \textit{Result from `ex7.10.R`}: GAM Test MSE can be slightly better than OLS. Anova often shows evidence of non-linearity for some predictors like `Expend`.
                \end{itemize}
            \item \textbf{ISLR 7.11 (Backfitting Algorithm)}: (Corresponds to `ex7.11.R`)
                \begin{itemize}
                    \item (a) Simulate data: $Y = \beta_0 + \beta_1 X_1 + \beta_2 X_2 + \epsilon$.
                    \item (b, c, d, e) Implement backfitting algorithm manually:
                        Initialize $\hat{\beta}_1$. Iterate:
                        1. $a_i = y_i - \hat{\beta}_1 x_{i1}$. Regress $a_i$ on $X_2$ to get $\hat{\beta}_{2, \text{iter}}$.
                        2. $a_i = y_i - \hat{\beta}_{2, \text{iter}} x_{i2}$. Regress $a_i$ on $X_1$ to get $\hat{\beta}_{1, \text{iter}+1}$.
                        Repeat until convergence. Plot $\hat{\beta}_0, \hat{\beta}_1, \hat{\beta}_2$ vs. iteration.
                    \item (f) Compare with OLS coefficients from \Rfunction{lm(Y ~ X1 + X2)}. They should match.
                    \item (g) How many iterations? Typically very few for linear case.
                    \item \textit{`ex7.11.R` shows code for this manual backfitting.}
                \end{itemize}
        \end{itemize}

\subsection{Lecture 11: Tree-Based Methods }
    \subsubsection{Introduction to Decision Trees }
        \begin{itemize}
            \item \textbf{Core Idea}: Segment the predictor space into a number of simple, non-overlapping regions. For every observation that falls into a region $R_m$, make the same prediction.
            \item \textbf{Regression Trees}: Prediction is the mean of response values for training observations in $R_m$.
            \item \textbf{Classification Trees}: Prediction is the most common class (mode) for training observations in $R_m$.
            \item Example (Slides L10 p.2-3, \Rpackage{ISLR} \Robject{Hitters} data, predict \Robject{logSalary} from \Robject{Years} and \Robject{Hits}):
                \begin{itemize}
                    \item First split: \Rcode{Years < 4.5}.
                    \item If \Rcode{Years < 4.5}, predict mean logSalary for this group (e.g., 5.11).
                    \item If \Rcode{Years >= 4.5}, then split further by \Rcode{Hits < 117.5}.
                        \begin{itemize}
                            \item If \Rcode{Hits < 117.5}, predict mean (e.g., 6.00).
                            \item If \Rcode{Hits >= 117.5}, predict mean (e.g., 6.74).
                        \end{itemize}
                    \item These splits define regions $R_1, R_2, R_3$.
                \end{itemize}
        \end{itemize}
    \subsubsection{Building Regression Trees }
        \begin{itemize}
            \item \textbf{Stratifying Predictor Space}:
                \begin{enumerate}
                    \item Divide predictor space ($X_1, \dots, X_p$) into $J$ non-overlapping rectangular regions $R_1, \dots, R_J$.
                    \item For any observation in region $R_j$, predict $\hat{y}_{R_j} = \text{mean}(y_i | x_i \in R_j)$.
                    \item Goal: Find regions that minimize RSS:
                        $$ RSS = \sum_{j=1}^{J} \sum_{i \in R_j} (y_i - \hat{y}_{R_j})^2 $$
                \end{enumerate}
            \item \textbf{Recursive Binary Splitting} (Greedy Algorithm, Slides L10 p.7-8, 14):
                \begin{itemize}
                    \item Computationally infeasible to consider all possible partitions.
                    \item Start with all data in one region.
                    \item Iteratively split existing regions. At each step, choose the predictor $X_j$ and cutpoint $s$ that lead to the greatest possible reduction in RSS when splitting a region into $\{X|X_j < s\}$ and $\{X|X_j \ge s\}$.
                    \item This is a top-down, greedy approach (best split at current step, may not lead to global optimum).
                    \item The two new regions are then split further using the same principle.
                    \item Continue until a stopping criterion is met (e.g., minimum number of observations per region/node, say 5 or 10 (Slide L10 p.14 \Rcode{mincut=10})).
                \end{itemize}
            \item R Example (Manual splitting logic for \Robject{Hitters} data, Slides L10 p.9-12):
                \begin{itemize}
                    \item Demonstrates finding the best split for \Rcode{Hits} by minimizing RSS over possible cutpoints $s$.
                    \item Then for \Rcode{Years}.
                    \item Compares RSS reduction from best \Rcode{Hits} split vs. best \Rcode{Years} split. Chooses \Rcode{Years < 4.5} as the first split because it gives lower total RSS.
                \end{itemize}
            \item R Implementation with \Rpackage{tree} package (Slides L10 p.15-16):
\begin{lstlisting}[caption={Fitting a Regression Tree with \Rpackage{tree} (Slides L10 p.15)}]
# library(ISLR)
# data(Hitters)
# Hitters <- Hitters[complete.cases(Hitters$Salary),] # Remove NAs for Salary
# Hitters$logSalary <- log(Hitters$Salary)

# library(tree)
# Fit tree to predict logSalary using Years and Hits
# tree.fit1 <- tree(logSalary ~ Years + Hits, data=Hitters,
#                  control = tree.control(nobs=nrow(Hitters), mincut=10)) # mincut is like minleaf
# summary(tree.fit1)
# plot(tree.fit1)
# text(tree.fit1, pretty=0, digits=3) # digits for precision of node labels

# Fit tree with all variables (Slide L10 p.16, more complex tree)
# tree.fit.allvars <- tree(logSalary ~ . - Salary, data=Hitters,
#                          control = tree.control(nobs=nrow(Hitters), mincut=10))
# plot(tree.fit.allvars)
# text(tree.fit.allvars, pretty=0)
\end{lstlisting}
        \end{itemize}

    \subsubsection{Tree Pruning }
        \begin{itemize}
            \item \textbf{Problem}: Large trees grown by recursive binary splitting often overfit the training data, leading to poor test performance.
            \item \textbf{Strategy}:
                \begin{enumerate}
                    \item Grow a large initial tree $T_0$ (e.g., by having a small `mincut` or `minsplit`).
                    \item Prune this tree back to get a sequence of smaller subtrees.
                \end{enumerate}
            \item \textbf{Cost Complexity Pruning (Weakest Link Pruning)} (Slide L10 p.17):
                \begin{itemize}
                    \item For each possible subtree $T \subseteq T_0$, consider a penalized RSS:
                        $$ \sum_{m=1}^{|T|} \sum_{i: x_i \in R_m} (y_i - \hat{y}_{R_m})^2 + \alpha |T| $$
                        where $|T|$ is the number of terminal nodes in subtree $T$.
                    \item $\alpha \ge 0$ is a non-negative tuning parameter.
                        \begin{itemize}
                            \item $\alpha=0$: $T_0$ is chosen (no penalty for complexity).
                            \item As $\alpha$ increases, the penalty for more terminal nodes increases, favoring smaller trees.
                        \end{itemize}
                    \item Goal: For a given $\alpha$, find the subtree $T_\alpha$ that minimizes this criterion.
                \end{itemize}
            \item \textbf{Selecting Optimal $\alpha$ (and thus best subtree)}:
                \begin{itemize}
                    \item Use k-fold cross-validation.
                    \item For each $\alpha$ in a range, get $T_\alpha$.
                    \item Compute CV error for each $T_\alpha$.
                    \item Choose $\alpha$ (and corresponding $T_\alpha$) that minimizes CV error.
                \end{itemize}
            \item R Implementation (\Rpackage{tree} package, Slides L10 p.18-19):
\begin{lstlisting}[caption={Pruning a Regression Tree (Slides L10 p.18-19)}]
# Assuming tree.fit1 from before (logSalary ~ Years + Hits)
# cv.results <- cv.tree(tree.fit1, FUN=prune.tree) # FUN=prune.tree is for regression
# # For classification, FUN=prune.misclass can be used.
# names(cv.results) # "size", "dev", "k", "method"
# # 'size' is number of terminal nodes.
# # 'dev' is cross-validated deviance (RSS for regression, or deviance for classification).
# # 'k' is effectively alpha (cost-complexity parameter).
# 
# plot(cv.results$size, cv.results$dev, type="b", 
#      xlab="Number of Terminal Nodes (Tree Size)", ylab="CV Deviance (RSS)")
# # Identify the size that minimizes CV deviance.
# best.size <- cv.results$size[which.min(cv.results$dev)] # e.g., 4 from slide
# 
# # Prune the original tree to this best size
# pruned.tree <- prune.tree(tree.fit1, best=best.size)
# plot(pruned.tree)
# text(pruned.tree, pretty=0)
\end{lstlisting}
        \end{itemize}

    \subsubsection{Classification Trees }
        \begin{itemize}
            \item Similar to regression trees, but $Y$ is qualitative.
            \item \textbf{Prediction in a region $R_m$}: Predict the most common class among training observations in $R_m$.
            \item \textbf{Splitting Criteria} (Instead of RSS, Slide L10 p.21):
                \begin{itemize}
                    \item Let $\hat{p}_{mk}$ be the proportion of training observations in region $R_m$ that belong to class $k$.
                    \item \textbf{Classification Error Rate}: $E_m = 1 - \max_k (\hat{p}_{mk})$. Proportion of misclassified obs in region $R_m$ if we predict majority class.
                        \begin{itemize}
                            \item Not sensitive enough for tree-growing (can lead to same error for different splits).
                            \item Often used for pruning.
                        \end{itemize}
                    \item \textbf{Gini Index}: $G_m = \sum_{k=1}^K \hat{p}_{mk} (1 - \hat{p}_{mk})$.
                        \begin{itemize}
                            \item Measure of total variance across $K$ classes, or node purity.
                            \item Small Gini $\implies$ node contains mostly observations from one class.
                        \end{itemize}
                    \item \textbf{Cross-Entropy (Deviance)}: $D_m = -\sum_{k=1}^K \hat{p}_{mk} \log(\hat{p}_{mk})$.
                        \begin{itemize}
                            \item Similar to Gini index. Small value $\implies$ node is pure.
                        \end{itemize}
                    \item Gini or Cross-Entropy are typically used for growing the tree because they are more sensitive to changes in node probabilities than classification error rate.
                \end{itemize}
            \item Recursive binary splitting proceeds by choosing splits that achieve the largest reduction in Gini index or cross-entropy.
            \item Pruning is done similarly to regression trees, using CV with error rate, Gini, or deviance.
            \item R Example (\Rpackage{ISLR} \Robject{Carseats} data, Slides L10 p.22-25):
\begin{lstlisting}[caption={Classification Tree for Carseats Data (Slides L10 p.22-25)}]
# library(ISLR)
# data(Carseats)
# Carseats$High <- as.factor(ifelse(Carseats$Sales <= 8, "No", "Yes"))
# 
# library(tree)
# tree.carseats <- tree(High ~ . - Sales, data=Carseats) # Exclude original Sales
# summary(tree.carseats)
# plot(tree.carseats)
# text(tree.carseats, pretty=0) # Shows classes and proportions at nodes
# 
# # Pruning for classification tree
# cv.carseats <- cv.tree(tree.carseats, FUN=prune.misclass) # Use misclass error for pruning
# plot(cv.carseats$size, cv.carseats$k, type="b", # Note: Slide plots size vs deviance (k=-alpha)
#      xlab="Tree Size", ylab="Alpha (Cost-Complexity)") 
# # Or plot deviance: plot(cv.carseats$size, cv.carseats$dev, type="b")
# 
# # From slide L10 p.24, deviance plot suggests size 4 might be optimal
# best.size.class <- 4 # Example from slide
# pruned.carseats <- prune.misclass(tree.carseats, best=best.size.class)
# plot(pruned.carseats)
# text(pruned.carseats, pretty=0)
\end{lstlisting}
        \end{itemize}
    \subsubsection{Pros and Cons of Decision Trees}
        \begin{itemize}
            \item \textbf{Pros (+)}:
                \begin{itemize}
                    \item Simple to understand and interpret.
                    \item Easy to visualize (graphical representation).
                    \item Can handle categorical predictors without dummy variables (natively in some packages like \Rpackage{tree}).
                    \item Can capture non-linear relationships and interactions.
                    \item Considered to mimic human decision-making.
                \end{itemize}
            \item \textbf{Cons (-)}:
                \begin{itemize}
                    \item Predictive accuracy is often not as good as other methods.
                    \item High variance: Small changes in training data can lead to very different tree structures.
                    \item Greedy approach of recursive binary splitting is not guaranteed to find the globally optimal tree.
                \end{itemize}
        \end{itemize}

    \subsubsection{Exercise: ISLR 8.8a-c}
        (Corresponds to `ex8.8.R` file, first part)
        \begin{itemize}
            \item (a) Split data into training and test sets.
            \item (b) Fit a regression tree to training data. Plot. Interpret. Compute test MSE.
\begin{lstlisting}[caption={ISLR 8.8b - Basic Regression Tree}]
# library(ISLR); library(tree)
# data(Carseats)
# set.seed(123) # Or seed from problem
# train_idx <- sample(1:nrow(Carseats), nrow(Carseats)/2)
# train_data <- Carseats[train_idx,]
# test_data <- Carseats[-train_idx,]
# 
# tree.sales <- tree(Sales ~ ., data=train_data)
# summary(tree.sales) # Variables used, number of nodes, deviance
# plot(tree.sales); text(tree.sales, pretty=0)
# # Interpretation: e.g., ShelveLoc=Good and Price < 109.5 -> high sales
# 
# pred.sales.test <- predict(tree.sales, newdata=test_data)
# test_mse_tree <- mean((test_data$Sales - pred.sales.test)^2)
# # test_mse_tree around 4-5 usually
\end{lstlisting}
            \item (c) Use cross-validation to determine optimal tree complexity (pruning). Does pruning improve test MSE?
\begin{lstlisting}[caption={ISLR 8.8c - Pruning Regression Tree}]
# cv.sales <- cv.tree(tree.sales, FUN=prune.tree)
# plot(cv.sales$size, cv.sales$dev, type="b", xlab="Tree Size", ylab="CV Deviance")
# # Find optimal size (e.g., cv.sales$size[which.min(cv.sales$dev)])
# best_size_cv <- cv.sales$size[which.min(cv.sales$dev)] # e.g., 15 in ex8.8.R
# 
# pruned.sales <- prune.tree(tree.sales, best=best_size_cv)
# plot(pruned.sales); text(pruned.sales, pretty=0)
# 
# pred.pruned.test <- predict(pruned.sales, newdata=test_data)
# test_mse_pruned <- mean((test_data$Sales - pred.pruned.test)^2)
# # Compare test_mse_pruned with test_mse_tree.
# # In ex8.8.R, pruning didn't help or slightly worsened MSE for that particular seed.
\end{lstlisting}
        \end{itemize}


\subsection{Lecture 12: Bagging, Random Forests, Boosting }
    \subsubsection{Bagging (Bootstrap Aggregation) }
        \begin{itemize}
            \item \textbf{Motivation} (Slide L11 p.3): Decision trees suffer from high variance. Averaging a set of observations reduces variance: $\text{Var}(\bar{X}) = \sigma^2/n < \sigma^2 = \text{Var}(X)$.
            \item \textbf{Idea} (Slide L11 p.4): Reduce prediction error variance by fitting many trees on different training sets and averaging their predictions. Since we only have one training set, we use bootstrap.
            \item \textbf{Bagging Algorithm for Regression Trees} (Slides L11 p.5, 12-13):
                \begin{enumerate}
                    \item Generate $B$ bootstrap samples from the original training data (sample with replacement).
                    \item For each bootstrap sample $b=1, \dots, B$:
                        \begin{itemize}
                            \item Grow a regression tree $\hat{f}^{b}(x)$ on this bootstrap sample. Trees are typically grown deep (unpruned).
                        \end{itemize}
                    \item To predict for a new point $x_0$: Average the predictions from all $B$ trees:
                        $$ \hat{f}_{\text{bag}}(x_0) = \frac{1}{B} \sum_{b=1}^{B} \hat{f}^{b}(x_0) $$
                \end{enumerate}
            \item For classification, the prediction is the majority vote among the $B$ trees. (Slide L11 p.13)
            \item Bagging for KNN (Slides L11 p.5, 8-10): Illustrates the concept. KNN is relatively stable, so bagging offers less improvement compared to unstable methods like trees. Test MSE improved slightly from 0.0607 to 0.0564 in the example.
            \item \textbf{Out-of-Bag (OOB) Error Estimation} (ISLR Ch 8.2.2, Slide L11 p.20):
                \begin{itemize}
                    \item Computationally efficient way to estimate test error without explicit CV.
                    \item Each bootstrap sample uses ~2/3 of original observations. The remaining ~1/3 are "out-of-bag" (OOB) for that tree.
                    \item For each observation $i$:
                        \begin{itemize}
                            \item Predict $y_i$ using only the trees for which observation $i$ was OOB.
                            \item Average these predictions (or take majority vote).
                        \end{itemize}
                    \item Calculate overall OOB MSE or error rate. Provides a valid estimate of test error.
                \end{itemize}
            \item R Example (Bagging for \Robject{Carseats} Sales, Slides L11 p.14-15):
\begin{lstlisting}[caption={Bagging with randomForest Package (Slide L11 p.15)}]
# library(ISLR); library(randomForest)
# data(Carseats)
# set.seed(1) # Assuming a seed from lecture example for Carseats train/test
# train_idx_cs <- sample(1:nrow(Carseats), nrow(Carseats)/2)
# train_cs <- Carseats[train_idx_cs,]
# test_cs <- Carseats[-train_idx_cs,]

# For bagging, set mtry = number of predictors (e.g., 10 if Sales is excluded)
# ncol(train_cs)-1 if Sales is the first column and no other exclusions.
# For Carseats, Sales is col 1. Other 10 vars are predictors.
# bagging.carseats <- randomForest(Sales ~ ., data=train_cs, mtry=10, importance=TRUE)
# # Default ntree=500
# pred.bag <- predict(bagging.carseats, newdata=test_cs)
# mse.bag <- mean((test_cs$Sales - pred.bag)^2) 
# # Slide L11 p.15 shows mse.bag = 3.113909
# # Compare to single tree mse1 = 4.881557 (Slide L11 p.14) - Bagging improves.
\end{lstlisting}
        \end{itemize}

    \subsubsection{Random Forests }
        \begin{itemize}
            \item \textbf{Improvement over Bagging}: Bagged trees can be highly correlated if there are a few very strong predictors that are always selected at top splits. Averaging correlated quantities doesn't reduce variance as much.
            \item \textbf{Random Forest Algorithm} (Slide L11 p.17):
                \begin{enumerate}
                    \item Generate $B$ bootstrap samples.
                    \item For each bootstrap sample, grow a tree:
                        \begin{itemize}
                            \item At \textit{each split} in the tree, randomly select a subset of $m$ predictors out of the total $p$ predictors.
                            \item Choose the best split among these $m$ predictors only.
                        \end{itemize}
                    \item Average predictions (regression) or take majority vote (classification).
                \end{enumerate}
            \item \textbf{Choice of $m$}:
                \begin{itemize}
                    \item Typically $m \approx \sqrt{p}$ for classification.
                    \item Typically $m \approx p/3$ for regression.
                    \item If $m=p$, then Random Forest = Bagging.
                \end{itemize}
            \item By restricting choice of predictors at each split, Random Forests decorrelate the trees, leading to greater variance reduction when averaged.
            \item R Example (\Robject{Carseats} Sales, Slides L11 p.18):
\begin{lstlisting}[caption={Random Forest with randomForest Package (Slide L11 p.18)}]
# Assuming train_cs, test_cs from Bagging example
# For Carseats, p=10 predictors. mtry=p/3 approx 3.
# rf.carseats <- randomForest(Sales ~ ., data=train_cs, mtry=3, importance=TRUE)
# pred.rf <- predict(rf.carseats, newdata=test_cs)
# mse.rf <- mean((test_cs$Sales - pred.rf)^2)
# # Slide L11 p.18 shows mserf = 3.568976. This is worse than bagging in this specific example.
# # Often RF is better or similar. Results can vary with seed and mtry choice.
\end{lstlisting}
        \end{itemize}

    \subsubsection{Variable Importance Measures (for Bagging/RF)}
        \begin{itemize}
            \item Trees make it hard to see overall variable importance when many trees are averaged.
            \item \textbf{Mean Decrease in Impurity (e.g., RSS or Gini)} (Slide L11 p.22):
                \begin{itemize}
                    \item For each tree, record total reduction in RSS (regression) or Gini index (classification) due to splits on each predictor.
                    \item Average this reduction over all $B$ trees for each predictor.
                    \item Larger value $\implies$ more important.
                    \item R: \Rfunction{importance(rf.object)} gives this (\Rcode{IncNodePurity} column), \Rfunction{varImpPlot(rf.object)}.
                \end{itemize}
            \item \textbf{Permutation Importance (Mean Decrease in Accuracy/MSE)}:
                \begin{itemize}
                    \item For each tree, calculate OOB error.
                    \item Then, for each predictor $X_j$:
                        \begin{itemize}
                            \item Randomly permute the values of $X_j$ in the OOB data for that tree.
                            \item Recalculate OOB error with permuted $X_j$.
                        \end{itemize}
                    \item The average increase in OOB error across all trees due to permuting $X_j$ is its importance.
                    \item R: \Rfunction{importance(rf.object, type=1)} or \Rcode{importance=TRUE} in \Rfunction{randomForest} and check \Rcode{\%IncMSE} (regression) or \Rcode{MeanDecreaseAccuracy} (classification).
                \end{itemize}
            \item R Example (\Robject{Carseats} Bagging, Slides L11 p.23):
\begin{lstlisting}[caption={Variable Importance Plot (Slide L11 p.23)}]
# Assuming bagging.carseats from before
# varImpPlot(bagging.carseats) 
# # Shows ShelveLoc and Price as most important by IncNodePurity (Gini/RSS decrease)
\end{lstlisting}
        \end{itemize}

    \subsubsection{Boosting }
        \begin{itemize}
            \item \textbf{Core Idea}: Sequentially fit trees. Each new tree is fit to the \textit{residuals} of the previous model, thus focusing on observations the model currently mispredicts.
            \item This is different from Bagging/RF where trees are grown independently.
            \item \textbf{Boosting Algorithm for Regression Trees} (Slide L11 p.25):
                \begin{enumerate}
                    \item Initialize $\hat{f}(x) = 0$ and residuals $r_i = y_i$ for all $i$.
                    \item For $b = 1, \dots, B$ (number of trees):
                        \begin{enumerate}
                            \item Fit a tree $\hat{f}^b$ with $d$ splits (often small, e.g., $d=1$ for stumps) to the training data $(X, r)$ (i.e., predict current residuals).
                            \item Update the overall prediction: $\hat{f}(x) \leftarrow \hat{f}(x) + \lambda \hat{f}^b(x)$.
                            \item Update residuals: $r_i \leftarrow r_i - \lambda \hat{f}^b(x_i)$.
                        \end{enumerate}
                    \item The final boosted model is $\hat{f}(x) = \sum_{b=1}^B \lambda \hat{f}^b(x)$. (Slide L11 p.25 has $\lambda$ multiplying sum, which is equivalent if $\lambda$ is constant).
                \end{itemize}
            \item \textbf{Tuning Parameters for Boosting}:
                \begin{itemize}
                    \item \textbf{Number of Trees $B$}: Unlike Bagging/RF, boosting can overfit if $B$ is too large. Chosen by CV.
                    \item \textbf{Shrinkage Parameter $\lambda$} (learning rate): Small positive number (e.g., 0.01, 0.1). Controls rate at which boosting learns. Smaller $\lambda$ requires larger $B$.
                    \item \textbf{Interaction Depth $d$} (or tree depth): Number of splits in each tree. $d=1$ gives an additive model. $d>1$ allows for interactions.
                \end{itemize}
            \item R Example (Simulated data, Slides L11 p.26-31, \Robject{Hitters} data, Slides L11 p.32-37):
\begin{lstlisting}[caption={Boosting with gbm Package (Hitters data, Slides L11 p.36)}]
# library(ISLR); library(gbm)
# data(Hitters)
# Hitters <- Hitters[complete.cases(Hitters$Salary),]
# Hitters$logSalary <- log(Hitters$Salary)
# set.seed(1234) # From slide
# itrain_hit <- sample(1:nrow(Hitters), floor(nrow(Hitters)/2))
# traindata_hit <- Hitters[itrain_hit,]
# testdata_hit <- Hitters[-itrain_hit,]
# ytest_hit <- testdata_hit$logSalary

# Define formula (slide uses a subset of variables)
# formula1_hit <- logSalary ~ CRuns+CHits+HmRun+CWalks+CRBI+Hits+RBI 

# Fit boosted model
# boost.hitters <- gbm(formula1_hit, data=traindata_hit, distribution="gaussian",
#                      n.trees=1000, interaction.depth=4, shrinkage=0.01)
# summary(boost.hitters) # Shows variable importance

# Predict and calculate Test MSE
# pred.gbm <- predict(boost.hitters, newdata=testdata_hit, n.trees=1000)
# mse.gbm <- mean((ytest_hit - pred.gbm)^2) 
# # mse.gbm = 0.3591336 from slide L11 p.36
# # mse.simple_tree = 0.3851602 from slide L11 p.35 - Boosting improves.
\end{lstlisting}
        \end{itemize}

    \subsubsection{Exercises: ISLR 8.8d-e, 8.11}
        \begin{itemize}
            \item \textbf{ISLR 8.8d-e (\Rpackage{Carseats} data)}: (Corresponds to `ex8.8.R` file, latter part)
                \begin{itemize}
                    \item (d) Bagging: Fit bagged trees for \Rcode{Sales}. Compute test MSE. Variable importance.
                        \textit{From `ex8.8.R`}: Bagging with \Rcode{mtry=10} (all predictors) gives Test MSE $\approx 2.5-2.6$. Price, ShelveLoc, Age are important.
                    \item (e) Random Forest: Fit RF for \Rcode{Sales}. Vary \Rcode{mtry}. Compute test MSE. Variable importance.
                        \textit{From `ex8.8.R`}: RF with \Rcode{mtry=5} (approx $p/2$) gives Test MSE $\approx 2.8-2.9$. Sometimes bagging performed slightly better for this dataset/seed.
                \end{itemize}
            \item \textbf{ISLR 8.11 (\Rpackage{Caravan} data)}: (Corresponds to `ch8-applied.R` Exercise 11)
                \begin{itemize}
                    \item Predict \Rcode{Purchase}. Highly imbalanced data.
                    \item (b) Boosting: Fit boosted trees. Variable importance.
                    \item (c) Evaluate Boosting: Predict on test set. Use threshold (e.g., 0.2). Compare with logistic regression. Focus on fraction of correctly predicted "Yes" among those predicted "Yes".
                    \item \textit{From `ch8-applied.R` Ex 11}: Boosting (threshold 0.2) correctly identifies ~20% of those it predicts will purchase. Logistic regression (threshold 0.2) correctly identifies ~14%. Boosting performs better for this specific task.
                \end{itemize}
        \end{itemize}

\subsection{Lecture 13: Support Vector Machines (SVMs) }
    \subsubsection{Introduction to Support Vector Machines}
        \begin{itemize}
            \item SVMs are a powerful class of supervised learning algorithms used for classification and regression (though primarily classification is covered).
            \item We focus on binary classification (two classes).
            \item Progression of concepts:
                \begin{enumerate}
                    \item \textbf{Maximal Margin Classifier}: Assumes classes are perfectly separable by a linear boundary.
                    \item \textbf{Support Vector Classifier (SVC)}: Extends maximal margin classifier to handle non-separable classes by allowing some misclassifications (soft margin). Still linear boundaries.
                    \item \textbf{Support Vector Machine (SVM)}: Extends SVC to accommodate non-linear boundaries using kernels.
                \end{enumerate}
        \end{itemize}

    \subsubsection{Hyperplanes }
        \begin{itemize}
            \item A hyperplane in a $p$-dimensional space is a flat affine subspace of dimension $p-1$.
            \item Equation: $\beta_0 + \beta_1 X_1 + \beta_2 X_2 + \dots + \beta_p X_p = 0$.
                \begin{itemize}
                    \item If $p=2$: Line, $X_2 = -\frac{\beta_0}{\beta_2} - \frac{\beta_1}{\beta_2} X_1$.
                    \item If $p=3$: Plane.
                \end{itemize}
            \item A hyperplane divides the $p$-dimensional space into two half-spaces.
        \end{itemize}

    \subsubsection{Classification by Separating Hyperplanes}
        \begin{itemize}
            \item Data: $n$ training observations $(\mathbf{x}_i, y_i)$, where $\mathbf{x}_i \in \mathbb{R}^p$ and $y_i \in \{-1, 1\}$.
            \item Decision rule for a test observation $\mathbf{x}^$:
                \begin{itemize}
                    \item Classify as $+1$ if $f(\mathbf{x}^) = \beta_0 + \beta_1 x_1^ + \dots + \beta_p x_p^ > 0$.
                    \item Classify as $-1$ if $f(\mathbf{x}^) < 0$.
                \end{itemize}
                (Slide L12 p.6-7 illustrates this for $p=2$).
            \item For a separating hyperplane, all training observations satisfy $y_i f(\mathbf{x}_i) > 0$.
            \item The magnitude $|f(\mathbf{x}^)|$ indicates confidence in the classification (larger magnitude $\implies$ further from hyperplane).
        \end{itemize}

    \subsubsection{The Maximal Margin Classifier (MMC)}
        \begin{itemize}
            \item If classes are linearly separable, there can be infinitely many separating hyperplanes.
            \item \textbf{MMC}: The separating hyperplane that is farthest from the closest training observations from either class. (Slide L12 p.8).
            \item \textbf{Margin ($M$)}: The perpendicular distance from the hyperplane to the nearest training observation. MMC maximizes $M$.
            \item \textbf{Support Vectors}: Training observations that lie exactly on the margin (equidistant from the hyperplane). They "support" the hyperplane; if moved, the hyperplane would change.
            \item \textbf{Optimization Problem} (Slide L12 p.9):
                $$ \max_{\beta_0, \dots, \beta_p, M} M $$
                Subject to:
                \begin{align}
                    \sum_{j=1}^p \beta_j^2 &= 1 \quad \text{(Normalization constraint)} \\
                    y_i (\beta_0 + \beta_1 x_{i1} + \dots + \beta_p x_{ip}) &\ge M \quad \forall i=1, \dots, n
                \end{align}
                The term $y_i f(\mathbf{x}_i)$ is the functional margin. With $\sum \beta_j^2 = 1$, $M$ becomes the geometric margin.
            \item Slide L12 p.10 shows the maximal margin hyperplane (black) and the margin boundaries (red/blue).
            \item \textbf{Limitation}: Only exists if classes are perfectly linearly separable. Sensitive to outliers. (Slide L12 p.11 shows a non-separable case).
        \end{itemize}

    \subsubsection{Support Vector Classifier (SVC) / Soft Margin Classifier }
        \begin{itemize}
            \item Handles non-linearly separable data or cases where a wider margin with a few errors is preferred.
            \item Allows some observations to be on the wrong side of the margin, or even on the wrong side of the hyperplane.
            \item \textbf{Optimization Problem} (Slide L12 p.12):
                $$ \max_{\beta_0, \dots, \beta_p, \epsilon_1, \dots, \epsilon_n, M} M $$
                Subject to:
                \begin{align}
                    \sum_{j=1}^p \beta_j^2 &= 1 \\
                    y_i (\beta_0 + \beta_1 x_{i1} + \dots + \beta_p x_{ip}) &\ge M(1 - \epsilon_i) \quad \forall i \\
                    \epsilon_i &\ge 0, \quad \sum_{i=1}^n \epsilon_i \le C
                \end{align}
            \item \textbf{Slack Variables $\epsilon_i$} (Slide L12 p.13):
                \begin{itemize}
                    \item $\epsilon_i = 0$: Observation $i$ is on the correct side of its margin.
                    \item $0 < \epsilon_i \le 1$: Observation $i$ is on the correct side of the hyperplane but on the wrong side of its margin (margin violation).
                    \item $\epsilon_i > 1$: Observation $i$ is on the wrong side of the hyperplane (misclassified).
                \end{itemize}
            \item \textbf{Tuning Parameter $C$} (Cost/Budget for violations, Slide L12 p.14):
                \begin{itemize}
                    \item Controls the trade-off between margin width and number/severity of margin violations.
                    \item Small $C$: Wide margin, more violations allowed (high bias, low variance). Similar to larger $s$ in $\sum \beta_j^2 \le s$.
                    \item Large $C$: Narrow margin, fewer violations allowed (low bias, high variance). Approaches MMC if data is separable.
                    \item $C$ is chosen by cross-validation.
                \end{itemize}
            \item Only observations on the margin or violating the margin (support vectors) affect the solution. (Slide L12 p.13)
            \item Still produces a linear decision boundary.
        \end{itemize}

    \subsubsection{Support Vector Machines (SVMs) - Non-linear Boundaries }
        \begin{itemize}
            \item Extends SVC to handle non-linear decision boundaries by using \textbf{kernels}.
            \item \textbf{The Kernel Trick}: Instead of explicitly mapping data to a higher-dimensional space to find a linear separator, kernels compute inner products in that higher-dimensional space directly using original data.
            \item The decision function becomes (Slide L12 p.19, simplified from dual form):
                $$ f(\mathbf{x}^) = \beta_0 + \sum_{i \in \mathcal{S}} \alpha_i K(\mathbf{x}^, \mathbf{x}_i) $$
                where $\mathcal{S}$ is the set of support vectors, $\alpha_i$ are parameters, and $K(\cdot, \cdot)$ is a kernel function.
            \item \textbf{Common Kernels} (Slide L12 p.19):
                \begin{itemize}
                    \item \textbf{Linear Kernel}: $K(\mathbf{x}_i, \mathbf{x}_{i'}) = \mathbf{x}_i' \mathbf{x}_{i'}$. Recovers the SVC.
                    \item \textbf{Polynomial Kernel} of degree $d$: $K(\mathbf{x}_i, \mathbf{x}_{i'}) = (1 + \gamma \mathbf{x}_i' \mathbf{x}_{i'} + r)^d$. (Slide has $1 + \sum x_{ij}x_{i'j}$). Tuning parameters: $d$, $\gamma$, cost $C$.
                    \item \textbf{Radial Basis Function (RBF) Kernel / Gaussian Kernel}:
                        $$ K(\mathbf{x}_i, \mathbf{x}_{i'}) = \exp\left(-\gamma \sum_{j=1}^p (x_{ij} - x_{i'j})^2\right) = \exp(-\gamma \|\mathbf{x}_i - \mathbf{x}_{i'}\|^2) $$
                        Tuning parameters: $\gamma > 0$, cost $C$.
                        \begin{itemize}
                            \item If $\gamma$ is very large, fit is very local (wiggly, high variance).
                            \item If $\gamma$ is very small, fit is very smooth (like linear, high bias).
                        \end{itemize}
                \end{itemize}
        \end{itemize}

    \subsubsection{R Implementation of SVMs}
        \begin{itemize}
            \item Main function: \Rfunction{svm} from \Rpackage{e1071}.
            \item Key arguments:
                \begin{itemize}
                    \item \Rcode{formula} or \Rcode{x, y} input.
                    \item \Rcode{data}: Training data.
                    \item \Rcode{kernel}: "linear", "polynomial", "radial" (default), "sigmoid".
                    \item \Rcode{cost}: Parameter $C$ (SVC/SVM).
                    \item \Rcode{degree}: For polynomial kernel.
                    \item \Rcode{gamma}: For polynomial and radial kernels.
                    \item \Rcode{scale = TRUE/FALSE}: Whether to standardize predictors. (Default TRUE). (Slide L12 p.17 shows FALSE for manual data).
                \end{itemize}
            \item \textbf{Tuning Parameters} using \Rfunction{tune} from \Rpackage{e1071}:
\begin{lstlisting}[caption={Tuning SVM with e1071::tune (ISLR Ch9 Lab, Ex 9.7)}]
# library(e1071)
# set.seed(1) # For reproducibility
# # Example data from ISLR Ch9 Lab / Slides L12 p.15
# x_sim_svm <- matrix(rnorm(202), ncol=2)
# y_sim_svm <- c(rep(-1,10), rep(1,10))
# x_sim_svm[y_sim_svm==1,] <- x_sim_svm[y_sim_svm==1,] + 1 # Create separable classes
# dat_sim_svm <- data.frame(x=x_sim_svm, y=as.factor(y_sim_svm))

# # Tune linear SVM for cost
# tune.out.linear <- tune(svm, y ~ ., data=dat_sim_svm, kernel="linear",
#                         ranges=list(cost=c(0.001, 0.01, 0.1, 1, 5, 10, 100)))
# summary(tune.out.linear) # Shows best cost and CV error
# bestmod.linear <- tune.out.linear$best.model
# summary(bestmod.linear) # Shows parameters of best model, number of SVs

# # Plotting the SVM decision boundary (Slides L12 p.17)
# # svmfit.linear <- svm(y ~ ., data=dat_sim_svm, kernel="linear", cost=10, scale=FALSE)
# # plot(svmfit.linear, dat_sim_svm) # For 2D data

# # Example with radial kernel (Slides L12 p.21)
# # svmfit.radial <- svm(y ~ ., data=dat_sim_svm, kernel="radial", cost=10, gamma=1, scale=FALSE)
# # plot(svmfit.radial, dat_sim_svm)
\end{lstlisting}
            \item \textbf{Support Vectors}: Can be accessed via \Rcode{svmfit\$index}.
        \end{itemize}

    \subsubsection{SVMs with More than Two Classes }
        \begin{itemize}
            \item \textbf{One-Versus-One (OVO)}: Construct $\binom{K}{2}$ SVMs, each comparing a pair of classes. Classify a new observation by majority vote among these classifiers.
            \item \textbf{One-Versus-All (OVA)}: Fit $K$ SVMs, each separating one class from the rest. Classify to the class whose SVM gives largest $|f(\mathbf{x})|$ (or highest probability if calibrated).
            \item \Rpackage{e1071} typically uses OVO for multi-class SVM.
        \end{itemize}

    \subsubsection{Relationship to Logistic Regression }
        \begin{itemize}
            \item SVC optimization problem has similarities to logistic regression, particularly when $L_2$ penalty is used with logistic regression.
            \item Loss function for SVC (hinge loss) vs. logistic loss. Hinge loss is zero for points correctly classified beyond the margin.
            \item In practice, performance can be similar, especially with linear boundaries. SVMs can be more robust to outliers when classes are well-separated due to focus on support vectors.
        \end{itemize}

    \subsubsection{Exercise: ISLR 9.7 }
        (Corresponds to `ex9.7.R`)
        \begin{itemize}
            \item (a) Create binary variable \Rcode{high}: 1 if \Rcode{mpg > median(mpg)}, 0 otherwise.
            \item (b) SVC (linear kernel): Use \Rfunction{tune} to find best \Rcode{cost}.
                \textit{From `ex9.7.R`}: Best cost is often 1.
            \item (c) SVM with polynomial and radial kernels: Use \Rfunction{tune} for \Rcode{cost}, \Rcode{degree} (poly), \Rcode{gamma} (radial).
                \textit{From `ex9.7.R`}:
                \begin{itemize}
                    \item Polynomial: E.g., cost=10, degree=2.
                    \item Radial: E.g., cost=10, gamma=0.01.
                \end{itemize}
            \item (d) Plot results and compare predictions (e.g., training confusion matrices).
                \begin{itemize}
                    \item Radial kernel often gives good separation on training data, but generalization to test data is key.
                    \item Linear SVM with optimal cost usually performs reasonably well.
                \end{itemize}
\begin{lstlisting}[caption={ISLR 9.7 - Tuning SVM (Example for Radial)}]
# library(ISLR); library(e1071)
# Auto$high <- as.factor(ifelse(Auto$mpg > median(Auto$mpg), 1, 0))
# Auto_svm <- Auto[, !(names(Auto) %in% c("mpg", "name"))] # Exclude original mpg and name

# set.seed(1)
# tune.radial <- tune(svm, high ~ ., data=Auto_svm, kernel="radial",
#                     ranges=list(cost=c(0.1, 1, 10, 100), gamma=c(0.01, 0.1, 1, 5)))
# summary(tune.radial)
# best.svm.radial <- tune.radial$best.model
# plot(best.svm.radial, Auto_svm, horsepower ~ weight) # Example plot
\end{lstlisting}
        \end{itemize}

\subsection{Lecture 14: Unsupervised Learning }
    \subsubsection{Introduction to Unsupervised Learning}
        \begin{itemize}
            \item \textbf{Key Characteristic}: We only observe predictors $\mathbf{X}$, no response variable $y$.
            \item \textbf{Goals}:
                \begin{itemize}
                    \item Exploratory / descriptive data analysis.
                    \item Detect unusual/remarkable patterns or subgroups in observations.
                    \item Generate hypotheses about the data structure.
                    \item Can be seen as "Advanced descriptive analysis."
                \end{itemize}
            \item \textbf{Main Techniques Covered}:
                \begin{itemize}
                    \item Principal Component Analysis (PCA) for dimension reduction and visualization.
                    \item Clustering (K-Means, Hierarchical) for finding subgroups.
                \end{itemize}
        \end{itemize}

    \subsubsection{Principal Component Analysis (PCA) }
        \begin{itemize}
            \item \textbf{Motivation} (Slides L13 p.3-4):
                \begin{itemize}
                    \item Visualizing high-dimensional data ($p > 2$) is difficult. A pairs plot for $p=10$ yields $\binom{10}{2}=45$ scatterplots.
                    \item PCA aims to summarize $p$ variables into a smaller set of $M < p$ new variables (principal components) that capture as much of the original variability as possible.
                \end{itemize}
            \item \textbf{What are Principal Components?}
                \begin{itemize}
                    \item Principal components (PCs) are linear combinations of the original predictors:
                        $$ Z_m = \phi_{1m}X_1 + \phi_{2m}X_2 + \dots + \phi_{pm}X_p = \sum_{j=1}^p \phi_{jm}X_j $$
                    \item The coefficients $\phi_{jm}$ are called \textbf{loadings}. The loading vector for PC $m$ is $\boldsymbol{\phi}_m = (\phi_{1m}, \dots, \phi_{pm})'$.
                    \item Loadings define the direction in feature space along which the data varies the most (for $Z_1$), or varies most subject to being uncorrelated with previous PCs (for $Z_2, Z_3, \dots$).
                \end{itemize}
            \item \textbf{First Principal Component ($Z_1$)} (Slide L13 p.5):
                \begin{itemize}
                    \item $Z_1$ is the normalized linear combination of $X_j$'s that has the largest variance:
                        $$ \text{maximize } \text{Var}(Z_1) = \text{Var}\left(\sum_{j=1}^p \phi_{j1}X_j\right) $$
                        subject to the normalization constraint $\sum_{j=1}^p \phi_{j1}^2 = 1$.
                    \item The loadings $\phi_{j1}$ define the first principal component direction.
                \end{itemize}
            \item \textbf{Second Principal Component ($Z_2$)} (Slide L13 p.6):
                \begin{itemize}
                    \item $Z_2$ is the normalized linear combination of $X_j$'s that has the largest variance, subject to being uncorrelated with $Z_1$ ($\text{Cov}(Z_1, Z_2)=0$).
                        $$ \text{maximize } \text{Var}(Z_2) \quad \text{subject to} \quad \sum_{j=1}^p \phi_{j2}^2 = 1 \quad \text{and} \quad \text{Cov}(Z_1, Z_2)=0 $$
                \end{itemize}
            \item Subsequent PCs are defined similarly (maximize variance, uncorrelated with all previous PCs). Up to $\min(n-1, p)$ PCs can be constructed.
            \item \textbf{Geometric Interpretation}: PCs represent new axes in the feature space. $Z_1$ is the direction of greatest spread, $Z_2$ is the next direction of greatest spread orthogonal to $Z_1$, etc.
            \item \textbf{Scaling of Variables}: It is crucial to scale variables to have standard deviation one (and often mean zero) before performing PCA if they are measured on different scales. Otherwise, variables with larger variance will dominate the PCs. (Slide L13 p.13 uses \Rcode{scale=TRUE} with \Rfunction{prcomp}).
            \item \textbf{Proportion of Variance Explained (PVE)}:
                \begin{itemize}
                    \item The variance explained by PC $m$ is $\text{Var}(Z_m)$. Total variance is $\sum_{j=1}^p \text{Var}(X_j)$ (if scaled, this is $p$).
                    \item $PVE_m = \frac{\text{Var}(Z_m)}{\sum_{j=1}^p \text{Var}(X_j)}$.
                    \item Cumulative PVE helps decide how many PCs to retain.
                \end{itemize}
            \item \textbf{Scree Plot}: Plot of PVE for each PC (or eigenvalues $\text{Var}(Z_m)$). Look for an "elbow" to decide number of components to keep.
            \item \textbf{R Implementation} (\Rfunction{prcomp} or \Rfunction{princomp}):
                \begin{itemize}
                    \item \Rfunction{prcomp} is generally preferred (uses SVD).
                    \item \Robject{rotation} element contains loadings $\phi_{jm}$.
                    \item \Robject{x} element contains the principal component scores $z_{im}$.
                    \item \Robject{sdev} contains standard deviations of PCs (square for variances).
                \end{itemize}
            \item R Example (Simulated $p=3$ data, Slides L13 p.7-10):
\begin{lstlisting}[caption={PCA on Simulated 3D Data (Slides L13 p.9-10)}]
# Assume 'x' is the n x 3 matrix of simulated data
# pca_sim <- princomp(x) # Or prcomp(x, scale.=TRUE, center.=TRUE)
# summary(pca_sim)
# Output from slide L13 p.9:
# Importance of components:
#                            Comp.1    Comp.2      Comp.3
# Standard deviation     1.5940343 0.6333490 0.031345120
# Proportion of Variance 0.8633688 0.1362973 0.000333842
# Cumulative Proportion  0.8633688 0.9996662 1.000000000
# Almost all variance (99.97%) captured by first two PCs.

# plot(pca_sim$scores[,1:2]) # Plot observations in PC1-PC2 space (Slide L13 p.10)
\end{lstlisting}
            \item R Example (\Rpackage{ISLR} \Robject{USArrests} data, Slides L13 p.11-16):
\begin{lstlisting}[caption={PCA on USArrests Data (Slides L13 p.13-16)}]
# library(ISLR)
# data(USArrests)
# # summary(USArrests) # Variables have different scales/means
# # apply(USArrests, 2, mean)
# # apply(USArrests, 2, var)

# pca.usarrests <- prcomp(USArrests, scale=TRUE, center=TRUE)
# summary(pca.usarrests)
# # Importance of components:
# #                            PC1    PC2     PC3     PC4
# # Standard deviation     1.5749 0.9949 0.59713 0.41645
# # Proportion of Variance 0.6201 0.2474 0.08914 0.04336
# # Cumulative Proportion  0.6201 0.8675 0.95664 1.00000
# # First two PCs explain ~87% of variance.

# pca.usarrests$rotation # Loadings (Slide L13 p.15)
# # PC1: High negative loadings for Murder, Assault, Rape (crime factor), negative for UrbanPop.
# #      States with low PC1 scores have high crime rates.
# # PC2: High positive loading for UrbanPop, small negative for Assault. (urbanization factor)
# #      States with high PC2 scores are highly urbanized.

# # Biplot (Slide L13 p.14 shows state names, Slide L13 p.16 shows arrows for variables)
# biplot(pca.usarrests, scale=0) # scale=0 ensures arrows represent loadings
\end{lstlisting}
        \end{itemize}

    \subsubsection{Clustering Methods }
        \begin{itemize}
            \item Goal: Find homogeneous subgroups (clusters) among observations. Observations within a cluster should be similar; observations in different clusters should be dissimilar.
            \item "Similarity" is context-dependent, often based on Euclidean distance between observations in feature space.
            \item Examples (Slide L13 p.17): Classifying plants based on features, identifying consumer subgroups based on shopping patterns.
        \end{itemize}

    \subsubsection{K-Means Clustering}
        \begin{itemize}
            \item \textbf{Objective}:
                \begin{itemize}
                    \item Partition $n$ observations into a pre-specified number of $K$ clusters $C_1, \dots, C_K$.
                    \item Clusters must be non-overlapping ($C_k \cap C_{k'} = \emptyset$) and cover all observations ($\cup C_k = \{1, \dots, n\}$).
                    \item Minimize within-cluster variation (WCV). A common measure is sum of squared Euclidean distances between observations in a cluster and the cluster centroid, or sum of pairwise squared Euclidean distances within a cluster.
                        $$ \min_{C_1, \dots, C_K} \sum_{k=1}^K W(C_k) $$
                        where $W(C_k) = \frac{1}{|C_k|} \sum_{i,i' \in C_k} \sum_{j=1}^p (x_{ij} - x_{i'j})^2$.
                        This is equivalent to minimizing $2 \sum_{k=1}^K \sum_{i \in C_k} \sum_{j=1}^p (x_{ij} - \bar{x}_{kj})^2$, where $\bar{x}_{kj}$ is the mean of feature $j$ for cluster $k$ (centroid). (ISLR Ch10 Exercise 10.1).
                \end{itemize}
            \item \textbf{Algorithm} (Greedy, iterative, Slide L13 p.20):
                \begin{enumerate}
                    \item Randomly assign each observation to one of $K$ clusters.
                    \item Iterate until cluster assignments stop changing:
                        \begin{enumerate}
                            \item For each cluster $k$, compute its centroid $\boldsymbol{\mu}_k$ (vector of feature means for observations in $C_k$).
                            \item Reassign each observation to the cluster whose centroid is closest (e.g., by Euclidean distance).
                        \end{enumerate}
                \end{enumerate}
            \item \textbf{Sensitivity to Initialization}: Final clustering can depend on initial random assignment. Run algorithm multiple times with different random starts (argument \Rcode{nstart} in R's \Rfunction{kmeans}) and choose the solution with smallest total WCV.
            \item \textbf{Choosing K}: No single best method.
                \begin{itemize}
                    \item Elbow method: Plot total WCV (or within-cluster sum of squares, \Rcode{tot.withinss}) vs. $K$. Look for an "elbow" point where adding more clusters gives diminishing returns.
                    \item Silhouette analysis.
                    \item Domain knowledge or practical considerations.
                \end{itemize}
            \item R Example (Simulated 2D data, 3 clusters, Slides L13 p.21-27):
\begin{lstlisting}[caption={K-Means Clustering in R (Slides L13 p.22, 26-27)}]
# Assume 'x_cluster_sim' is an n x 2 matrix of simulated data with 3 true clusters
# K_val <- 3

# Manual iteration steps (conceptual from slides L13 p.22-25)
# 1. Random initial assignment:
#    cluster_colors <- c("black", "red", "blue")
#    initial_assignment <- sample(1:K_val, size=nrow(x_cluster_sim), replace=TRUE)
#    plot(x_cluster_sim, col=cluster_colors[initial_assignment])
#    centroids <- matrix(0, K_val, ncol(x_cluster_sim))
#    for(k_idx in 1:K_val) {
#      centroids[k_idx,] <- colMeans(x_cluster_sim[initial_assignment==k_idx,])
#      points(centroids[k_idx,1], centroids[k_idx,2], pch=17, col=cluster_colors[k_idx], cex=2)
#    }
# 2. Reassign based on distance to centroids, recompute centroids ... (iterate)

# Using kmeans() function (Slide L13 p.26-27)
# set.seed(123) # For reproducibility of kmeans if nstart=1
# km.out <- kmeans(x_cluster_sim, centers=K_val, nstart=20) # nstart=20 runs 20 random initializations
# names(km.out) # "cluster", "centers", "totss", "withinss", "tot.withinss", "betweenss", "size"
#
# plot(x_cluster_sim, col=cluster_colors[km.out$cluster], main="K-Means Clustering Result (K=3)")
# points(km.out$centers, col=cluster_colors, pch=17, cex=2, lwd=2)
\end{lstlisting}
        \end{itemize}

    \subsubsection{Hierarchical Clustering }
        \begin{itemize}
            \item Does not require pre-specifying $K$.
            \item Produces a tree-based representation (dendrogram) of the observations.
            \item \textbf{Agglomerative (Bottom-Up) Approach} (Algorithm on Slide L13 p.32):
                \begin{enumerate}
                    \item Start with each of $n$ observations as its own cluster.
                    \item Compute all $n(n-1)/2$ pairwise dissimilarities (e.g., Euclidean distances).
                    \item For $i = n, n-1, \dots, 2$:
                        \begin{enumerate}
                            \item Fuse the two closest (most similar) clusters. The distance at which they fuse becomes the height in the dendrogram.
                            \item Compute new pairwise inter-cluster dissimilarities among the $i-1$ remaining clusters.
                        \end{enumerate}
                \end{enumerate}
            \item \textbf{Linkage Methods} (How to define dissimilarity between clusters, Slide L13 p.33):
                \begin{itemize}
                    \item \textit{Complete}: Maximal inter-cluster dissimilarity (largest distance between an obs in cluster A and an obs in cluster B).
                    \item \textit{Single}: Minimal inter-cluster dissimilarity (smallest distance). Can result in "chaining."
                    \item \textit{Average}: Mean inter-cluster dissimilarity.
                    \item \textit{Centroid}: Dissimilarity between centroids of two clusters. Can lead to inversions in dendrogram (later fusions at lower height).
                    \item \textit{Ward's method}: Minimizes increase in total within-cluster variance upon merging.
                \end{itemize}
            \item \textbf{Dendrogram} (Slide L13 p.30, USArrests example):
                \begin{itemize}
                    \item Leaves are individual observations.
                    \item Fusions are represented by horizontal lines; height indicates dissimilarity at fusion.
                    \item Cut dendrogram at a certain height to obtain $K$ clusters. (Slide L13 p.31)
                \end{itemize}
            \item Choice of dissimilarity measure (Euclidean, correlation-based, etc.) and linkage method can significantly affect results.
            \item R Example (\Robject{USArrests} data, Slide L13 p.29-30):
\begin{lstlisting}[caption={Hierarchical Clustering in R (Slides L13 p.30)}]
# data(USArrests)
# # Hierarchical clustering with complete linkage and Euclidean distance
# hc.complete <- hclust(dist(USArrests), method="complete")
# plot(hc.complete, main="Complete Linkage", xlab="", sub="", cex=0.9)
# 
# # Cutting the tree to get specific number of clusters
# cutree(hc.complete, k=3) # Get 3 clusters
# 
# # Scaling data is often recommended before calculating distances
# hc.complete.scaled <- hclust(dist(scale(USArrests)), method="complete")
# plot(hc.complete.scaled, main="Complete Linkage (Scaled Data)")
\end{lstlisting}
        \end{itemize}
    \subsubsection{Practical Issues in Clustering }
        \begin{itemize}
            \item \textbf{Scaling Variables}: Important if variables are on different scales, as distance measures will be dominated by variables with larger magnitudes/variances. Standardize to mean 0, std dev 1.
            \item \textbf{Choice of K (for K-Means) or Cut Height (for Hierarchical)}: Often subjective or guided by external criteria/domain knowledge.
            \item \textbf{Interpreting Clusters}: Examine feature means/distributions within each cluster.
            \item Clustering is exploratory; results should be interpreted with caution and domain expertise.
        \end{itemize}
    \subsubsection{Exercises: ISLR 12.8, 12.9 (}
        \begin{itemize}
            \item \textbf{ISLR 12.8 (old 10.8, PCA on \Robject{USArrests})} (Corresponds to `ch10-lab1.R`, `ch10ex8.R`):
                \begin{itemize}
                    \item (a) Calculate PVE for each PC.
                    \item (b) Relate PVE to sum of squared PC scores and sum of squared scaled original data. (PVE for PC $m$ is $\frac{\sum_{i=1}^n z_{im}^2}{\sum_{j=1}^p \sum_{i=1}^n \tilde{x}_{ij}^2}$, where $z_{im}$ are scores for PC $m$, and $\tilde{x}_{ij}$ are scaled original data.)
                \end{itemize}
            \item \textbf{ISLR 12.9 (old 10.9, Hierarchical Clustering on \Robject{USArrests})} (Corresponds to `ch10-lab2.R`, `ch10ex9.R`):
                \begin{itemize}
                    \item (a) Hierarchical clustering with complete linkage, Euclidean distance. Plot dendrogram.
                    \item (b) Cut tree to get 3 clusters. Identify states in each.
                    \item (c) Repeat with scaled data. Plot.
                    \item (d) Compare results. Scaling can change cluster assignments significantly. Discuss if scaling is appropriate (generally yes if variables have different units/scales).
                \end{itemize}
        \end{itemize}


\section{Previous Exams}
\subsection{Exam Spring 2021 }
\vspace{0.5em}
Overall Datasets: \Robject{Smarket} (\Rpackage{ISLR}), 
\Robject{dataCar(\Rpackage{insuranceData})}
\vspace{0.5em}

\textbf{\Large Task 1: Analysis of \Robject{Smarket} Data}
\vspace{0.5em}

    \subsubsection{1a: Bootstrap histogram for volatility}
        \paragraph{Question}
        \textit{Use the bootstrap to create a histogram of the sampling distribution for the volatility (i.e., the standard deviation of the returns) of the S\&P stock index. Observations of the returns are in the variable \Robject{Today}. Start by setting the seed to 1 (\Rcode{set.seed(1)}). Use 1000 bootstrap replicates.}
        \paragraph{Base R Solution}
\begin{lstlisting}[]
library(ISLR)
library(boot)
data(Smarket)

set.seed(1)
calculate_sd_Smarket <- function(data_vector, index) {
  return(sd(data_vector[index]))
}
boot_volatility_Smarket <- boot(data = Smarket$Today, statistic = calculate_sd_Smarket, R = 1000)
hist(boot_volatility_Smarket$t, main = "Bootstrap Dist. of Volatility (Smarket$Today)",
     xlab = "Estimated Volatility (Std Dev)", col = "lightblue", breaks = "Scott")
\end{lstlisting}
        \paragraph{Tidyverse Solution (Data Prep + Base R boot)}
\begin{lstlisting}[]
library(ISLR); library(boot); library(dplyr); library(ggplot2)
data(Smarket)
today_returns_Smarket <- Smarket %>% pull(Today)
set.seed(1)
# calculate_sd_Smarket function is the same
boot_volatility_tidy_Smarket <- boot(data = today_returns_Smarket, statistic = calculate_sd_Smarket, R = 1000)
ggplot(data.frame(volatility = boot_volatility_tidy_Smarket$t), aes(x = volatility)) +
  geom_histogram(aes(y=..density..), fill = "lightblue", color = "black", bins=30) +
  geom_density(alpha=.2, fill="#FF6666") +
  labs(title="Bootstrap Distribution of Volatility (Smarket$Today)",
       x="Estimated Volatility", y="Density")
\end{lstlisting}

    \subsubsection{1b: 95\% CI for volatility (normal assumption)}
        \paragraph{Question}
        \textit{Compute a 95\% confidence interval based on the bootstrap, assuming that the volatility is normally distributed.}
        \paragraph{Base R Solution (using \Rpackage{boot})}
\begin{lstlisting}[]
# Assuming boot_volatility_Smarket from 1a
ci_normal_Smarket <- boot.ci(boot_volatility_Smarket, type = "norm")
print(ci_normal_Smarket)
# Interpretation: Based on the normal approximation, we are 95% confident the true volatility is between [lower] and [upper].
\end{lstlisting}

    \subsubsection{1c: 95\% CI for volatility (percentile method)}
        \paragraph{Question}
        \textit{Compute a 95\% confidence interval based on the bootstrap, not making the normality assumption but use percentiles from the bootstrapped volatilities.}
        \paragraph{Base R Solution (using \Rpackage{boot})}
\begin{lstlisting}[]
# Assuming boot_volatility_Smarket from 1a
ci_percentile_Smarket <- boot.ci(boot_volatility_Smarket, type = "perc")
print(ci_percentile_Smarket)
# Interpretation: Using percentiles, we are 95% confident the true volatility is between [lower] and [upper].
\end{lstlisting}

    \subsubsection{1d: Regression of Squared Returns}
        \paragraph{Question}
        \textit{Squared returns, \Rcode{Today\textasciicircum{}2}, can be used as a proxy for volatility. Squared returns are often autocorrelated, indicating that volatility can be predicted. Run a regression of the square of \Robject{Today} on the square of \Robject{Lag1} and interpret the estimate of the regression parameter for the square of \Robject{Lag1}.}
        \paragraph{Base R Solution}
\begin{lstlisting}[]
model_sq_returns_Smarket <- lm(I(Today^2) ~ I(Lag1^2), data = Smarket)
summary_model_sq_Smarket <- summary(model_sq_returns_Smarket)
print(summary_model_sq_Smarket)
# Interpretation of I(Lag1^2) coefficient:
# A one-unit increase in Lag1^2 (yesterday's squared return) is associated
# with an estimated [coefficient_value, e.g., 0.165] change in Today^2 (today's squared return).
# This relationship is statistically significant (p-value = [p_val]).
\end{lstlisting}

    \subsubsection{1e: Bootstrap SE of regression coefficient}
        \paragraph{Question}
        \textit{Compute the bootstrapped standard error of the regression coefficient in front of the square of \Robject{Lag1} in the model in d) and compare it with the standard error from the regression output.}
        \paragraph{Base R Solution}
\begin{lstlisting}[]
coef_sq_lag1_Smarket_bs <- function(data, index) {
  subset_data <- data[index, ]
  fit <- lm(I(Today^2) ~ I(Lag1^2), data = subset_data)
  return(coef(fit)[2]) 
}
set.seed(1) % \textit{(Or seed from marking scheme for this specific subtask if different, e.g., if a variant was used.)} %
boot_coef_Smarket <- boot(data = Smarket, statistic = coef_sq_lag1_Smarket_bs, R = 1000)
bootstrap_se_coef_Smarket <- sd(boot_coef_Smarket$t)
cat("Bootstrap SE for I(Lag1^2) coef (Smarket):", bootstrap_se_coef_Smarket, "\n")

lm_se_coef_Smarket <- summary(model_sq_returns_Smarket)$coefficients["I(Lag1^2)", "Std. Error"]
cat("LM Output SE for I(Lag1^2) coef (Smarket):", lm_se_coef_Smarket, "\n")
# Comparison: The bootstrap SE ([value]) is [larger/smaller/similar] than the lm output SE ([value]).
# Marking scheme: Bootstrap SE (0.0488) was almost twice lm SE (0.0279).
\end{lstlisting}

\vspace{1em}
\textbf{\Large Task 2: Analysis of \Robject{dataCar} Data - Predicting \Robject{claimcst0}}
\vspace{0.5em}

    \subsubsection{2a: Data Filtering and Variable Removal}
        \paragraph{Question}
        \textit{Start by first selecting the observations where there is an occurrence of claim, \Rcode{clm!=0}. Also, remove the variable \Rcode{X\_OBSTAT\_}. Work with this new data set for the rest of Task 2.}
        \paragraph{Base R Solution}
\begin{lstlisting}[]
library(insuranceData)
data(dataCar)
xdata_task2_car <- dataCar[dataCar$clm != 0, ]
xdata_task2_car <- xdata_task2_car[, names(xdata_task2_car) != "X_OBSTAT_"]
# print(paste("Dimensions after filtering:", paste(dim(xdata_task2_car), collapse="x")))
\end{lstlisting}
        \paragraph{Tidyverse Solution}
\begin{lstlisting}[]
library(insuranceData); library(dplyr)
data(dataCar)
xdata_task2_car_tidy <- dataCar %>%
  filter(clm != 0) %>%
  select(-any_of("X_OBSTAT_")) 
# glimpse(xdata_task2_car_tidy)
\end{lstlisting}

    \subsubsection{2b: Descriptive statistics and variable formatting}
        \paragraph{Question}
        \textit{Use descriptive statistics, the description of the data in the help-function and common sense to remove variables that you think will not be helpful. Motivate your choices thoroughly. Also, make sure that all variables are on the right format.}
        \paragraph{Base R / Tidyverse Solution (Conceptual - based on marking scheme)}
\begin{lstlisting}[]
# Assuming xdata_task2_car from 2a
# Marking scheme solution: Remove clm (col 3 in their xdata after X_OBSTAT_ removal)
# Original columns: veh_value, exposure, clm, numclaims, claimcst0, veh_body, veh_age, gender, area, agecat
# After X_OBSTAT_ removal and clm filtering, clm column is now constant 1.
# If X_OBSTAT_ was, say, col 11 as in marking scheme:
# data(dataCar); xdata_task2_car <- dataCar[dataCar$clm !=0,]; xdata_task2_car <- xdata_task2_car[,-11]
# Original column indices matter for marking scheme solution.
# Let's assume xdata_task2_car is as per previous step (X_OBSTAT_ removed).
# `clm` column is now all 1s.

# Removing 'clm' as it's constant after filtering:
xdata_task2_car_processed <- xdata_task2_car[, names(xdata_task2_car) != "clm"]

# Recoding 'veh_age' and 'agecat' to factors as they represent categories
xdata_task2_car_processed$veh_age <- as.factor(xdata_task2_car_processed$veh_age)
xdata_task2_car_processed$agecat <- as.factor(xdata_task2_car_processed$agecat)
# 'numclaims' is kept as numeric. 'exposure' also kept.

# Motivation for removals/changes:
# - 'clm': After filtering for clm!=0, this variable is constant and uninformative for predicting claimcst0 among those who claimed.
# - 'veh_age', 'agecat': These are described as age categories (e.g., "youngest"), making factor representation appropriate.
# - 'X_OBSTAT_': Likely an ID, not predictive.
# - Other variables (e.g., 'exposure', 'numclaims') are kept as potential predictors for claim amount.
# print(sapply(xdata_task2_car_processed, class))
\end{lstlisting}

    \subsubsection{2c: Cross-validation for linear models}
        \paragraph{Question}
        \textit{Use a cross validation approach to evaluate predictions for \Rcode{claimcst0} using: 1) Full LM, 2) All simple LMs, 3) Intercept-only LM. Motivate CV method. Seed 123 (Marking guide used 543, this exam used 123 in task description, so using 123).}
        \paragraph{Base R Solution (5-fold CV)}
\begin{lstlisting}[]
# Assuming xdata_task2_car_processed from 2b
set.seed(123) 
K_cv <- 5
n_car_proc <- nrow(xdata_task2_car_processed)
folds_cv_car <- sample(cut(seq(1, n_car_proc), breaks=K_cv, labels=FALSE))

response_car <- "claimcst0"
predictors_all_car <- names(xdata_task2_car_processed)[names(xdata_task2_car_processed) != response_car]

# Matrix to store MSEs
mse_results_car <- matrix(NA, nrow=K_cv, ncol=2 + length(predictors_all_car))
colnames(mse_results_car) <- c("FullLM", "InterceptOnly", predictors_all_car)

for (k_idx_cv in 1:K_cv) {
  test_idx_cv <- which(folds_cv_car == k_idx_cv)
  train_df_cv <- xdata_task2_car_processed[-test_idx_cv, ]
  test_df_cv  <- xdata_task2_car_processed[test_idx_cv, ]

  # Full LM
  lm_full_cv_car <- lm(paste(response_car, "~ ."), data=train_df_cv)
  pred_full_cv_car <- predict(lm_full_cv_car, newdata=test_df_cv)
  mse_results_car[k_idx_cv, "FullLM"] <- mean((test_df_cv[[response_car]] - pred_full_cv_car)^2, na.rm=TRUE)

  # Intercept-only LM
  lm_int_cv_car <- lm(paste(response_car, "~ 1"), data=train_df_cv)
  pred_int_cv_car <- predict(lm_int_cv_car, newdata=test_df_cv)
  mse_results_car[k_idx_cv, "InterceptOnly"] <- mean((test_df_cv[[response_car]] - pred_int_cv_car)^2, na.rm=TRUE)

  # Simple LMs
  for (pred_name_car in predictors_all_car) {
    # Ensure factor levels are consistent or handle potential errors if a level is missing in a fold
    # This might require more careful data handling in a real scenario for factors
    if (is.factor(train_df_cv[[pred_name_car]]) && length(levels(droplevels(train_df_cv[[pred_name_car]]))) < 2) next 
    
    formula_simple_cv_car <- as.formula(paste(response_car, "~", pred_name_car))
    tryCatch({ # In case of errors with factors in small folds
        lm_simple_cv_car <- lm(formula_simple_cv_car, data=train_df_cv)
        pred_simple_cv_car <- predict(lm_simple_cv_car, newdata=test_df_cv)
        mse_results_car[k_idx_cv, pred_name_car] <- mean((test_df_cv[[response_car]] - pred_simple_cv_car)^2, na.rm=TRUE)
    }, error = function(e) { print(paste("Error with", pred_name_car, "in fold", k_idx_cv))})
  }
}
cv_avg_mses_car <- colMeans(mse_results_car, na.rm=TRUE)
print("Average CV Test MSEs for different models:")
print(sort(cv_avg_mses_car))
# Motivation for K=5: Balances bias (not too small training folds) and variance (multiple estimates)
# of the test error estimate, and is computationally feasible for fitting many models.
\end{lstlisting}

    \subsubsection{2d: Generalized Additive Model (GAM)}
        \paragraph{Question}
        \textit{Use a GAM with smoothing splines for the numerical variables. Include categorical variables in the model as well. Use \Rfunction{s}-function in \Rpackage{gam}-library, with argument \Rcode{df=4}, without motivation. Plot the result and comment on the relationship between predictors and dependent variable.}
        \paragraph{Base R Solution (using \Rpackage{gam})}
\begin{lstlisting}[]
library(gam)
# Assuming xdata_task2_car_processed from 2b
num_preds_car_gam <- names(xdata_task2_car_processed)[sapply(xdata_task2_car_processed, is.numeric) & 
                                              names(xdata_task2_car_processed) != "claimcst0"]
cat_preds_car_gam <- names(xdata_task2_car_processed)[sapply(xdata_task2_car_processed, is.factor)]

gam_terms_str_car <- c()
for (pred in num_preds_car_gam) { gam_terms_str_car <- c(gam_terms_str_car, paste0("s(", pred, ", df=4)")) }
for (pred in cat_preds_car_gam) { gam_terms_str_car <- c(gam_terms_str_car, pred) }
gam_formula_full_car_str <- paste("claimcst0 ~", paste(gam_terms_str_car, collapse=" + "))
gam_formula_full_car <- as.formula(gam_formula_full_car_str)

# Fit GAM on the processed dataset (xdata_task2_car_processed) for plotting
gam_model_car_full <- gam(gam_formula_full_car, data=xdata_task2_car_processed)
# summary(gam_model_car_full)

# Plot partial effects (adjust mfrow based on number of terms)
# num_plots <- length(num_preds_car_gam) + length(cat_preds_car_gam)
# par(mfrow=c(ceiling(num_plots/3), 3)) 
# plot(gam_model_car_full, se=TRUE, col="blue", ask=TRUE) 
# Comment: For each s() term: is it linear, U-shaped, increasing/decreasing?
# Example from marking guide: numclaims linear, veh_value non-linear for small values, exposure non-linear.
\end{lstlisting}

    \subsubsection{2e: Specify appropriate GAM and compute testMSE}
        \paragraph{Question}
        \textit{Use your conclusions in 2d to specify an appropriate GAM-model. Compute the testMSE for this GAM-model.}
        \paragraph{Base R Solution (Integrating with CV loop from 2c)}
\begin{lstlisting}[]
# Based on marking scheme plots/conclusions: s(veh_value), s(exposure), linear numclaims, and all factors.
# Assuming 'numclaims' is numeric, others like 'veh_body', 'veh_age', 'gender', 'area', 'agecat' are factors.
gam_formula_refined_car_str <- "claimcst0 ~ s(veh_value, df=4) + s(exposure, df=4) + numclaims + veh_body + veh_age + gender + area + agecat"
gam_formula_refined_car <- as.formula(gam_formula_refined_car_str)

# To compute CV testMSE, add this to the CV loop from 2c:
# (Inside the loop for k_idx_cv in 1:K_cv)
# Ensure factors have same levels in train_df_cv and test_df_cv, or handle carefully.
#   tryCatch({
#     gam_refined_cv_car <- gam(gam_formula_refined_car, data=train_df_cv)
#     pred_gam_refined_cv_car <- predict(gam_refined_cv_car, newdata=test_df_cv)
#     # Add a new column to mse_matrix_car if not already defined:
#     # mse_matrix_car[k_idx_cv, "RefinedGAM"] <- mean((test_df_cv[[response_car]] - pred_gam_refined_cv_car)^2, na.rm=TRUE)
#   }, error = function(e) { print(paste("Error with GAM in fold", k_idx_cv, ":", e$message))})
# } %% End of loop %
# # After loop: 
# # cv_avg_mses_car <- colMeans(mse_matrix_car, na.rm=TRUE)
# # print(cv_avg_mses_car["RefinedGAM"])
# # Marking guide: Full GAM had testMSE ~12,243,092. Full linear reg ~12,277,335. GAM slightly better.
\end{lstlisting}

\vspace{1em}
\textbf{\Large Task 3: Analysis of \Robject{dataCar} Data (Original) - Predicting \Robject{clm}}
\vspace{0.5em}
    \textit{Note: For this task, use the original \Rcode{dataCar}, not the version from Task 2.}

    \subsubsection{3a: Data prep, descriptive stats for \Rcode{clm}}
        \paragraph{Question}
        \textit{Start by removing variables \Rcode{X\_OBSTAT\_}, \Rcode{numclaims} and \Rcode{claimcst0}. Compute and interpret appropriate cross-tables of \Rcode{clm} and the categorical variables. Also compute and interpret well-chosen descriptive statistics (such as central tendencies and measures of variation) for the numerical variables for each of the two categories of \Rcode{clm}. Are some of the categories of the categorical variables more interesting than others? Based on this, do you see any promising predictors for \Rcode{clm}? Mention a drawback with such pairwise comparisons.}
        \paragraph{Base R Solution}
\begin{lstlisting}[]
data(dataCar) # Reload original
xdata_task3_orig_car <- dataCar[, !(names(dataCar) %in% c("X_OBSTAT_", "numclaims", "claimcst0"))]
xdata_task3_orig_car$clm <- as.factor(xdata_task3_orig_car$clm) # Target variable

# Ensure other categoricals are factors
xdata_task3_orig_car$veh_body <- as.factor(xdata_task3_orig_car$veh_body)
xdata_task3_orig_car$veh_age <- as.factor(xdata_task3_orig_car$veh_age)
xdata_task3_orig_car$gender <- as.factor(xdata_task3_orig_car$gender)
xdata_task3_orig_car$area <- as.factor(xdata_task3_orig_car$area)
xdata_task3_orig_car$agecat <- as.factor(xdata_task3_orig_car$agecat)

# cat_predictors_task3 <- names(xdata_task3_orig_car)[sapply(xdata_task3_orig_car, is.factor) & names(xdata_task3_orig_car)!="clm"]
# num_predictors_task3 <- names(xdata_task3_orig_car)[sapply(xdata_task3_orig_car, is.numeric)]

# for (pred in cat_predictors_task3) {
#   cat("\nCross-table for clm vs", pred, ":\n")
#   tbl <- table(Claim=xdata_task3_orig_car$clm, Predictor=xdata_task3_orig_car[[pred]])
#   print(tbl)
#   print(round(prop.table(tbl, margin = 2)100,1)) # Column percentages (P(Claim | Predictor Level))
# }
# # Interpretation: Look for levels of categorical predictors with notably higher/lower claim rates.
# # E.g., "BUS and MCARA vehicle types show higher claim rates (19% and 11% vs. overall 6.8%)."

# for (pred_num in num_predictors_task3) {
#   cat("\nSummary of", pred_num, "by clm status:\n")
#   print(by(xdata_task3_orig_car[[pred_num]], xdata_task3_orig_car$clm, summary))
#   # boxplot(as.formula(paste(pred_num, "~ clm")), data=xdata_task3_orig_car, main=paste(pred_num, "by Claim"))
# }
# # Interpretation: "Policies with claims have, on average, higher veh_value and exposure."
# # Promising predictors: veh_body, agecat, veh_value, exposure (from marking guide).
# # Drawback of pairwise: Does not account for confounding or interaction effects between predictors.
\end{lstlisting}

    \subsubsection{3b: Logistic regression for \Rcode{clm}}
        \paragraph{Question}
        \textit{Based on your hypotheses from 3a, use logistic regression based on all or a subset of the predictors to predict \Rcode{clm}. Interpret the estimated coefficients.}
        \paragraph{Base R Solution}
\begin{lstlisting}[]
# Assuming xdata_task3_orig_car from 3a.
# Predictors from marking guide: veh_body, agecat, veh_value, exposure
logit_model_clm_car_task3 <- glm(clm ~ veh_body + agecat + veh_value + exposure, 
                                 data=xdata_task3_orig_car, family=binomial)
summary(logit_model_clm_car_task3)
# Interpretation (odds ratios):
# coefs <- coef(logit_model_clm_car_task3)
# odds_ratios <- exp(coefs)
# print(odds_ratios)
# Example: For agecat2, odds of claim are exp(coef_agecat2) (e.g., 0.81) times odds for agecat1 (baseline),
# holding other variables constant.
# For veh_value (continuous), a one-unit increase (10000 dollars) multiplies odds of claim by exp(0.06) ~ 1.06.
\end{lstlisting}

    \subsubsection{3c: Validation set for logistic regression, thresholding}
        \paragraph{Question}
        \textit{Modify model if you believe it to be sensible. Use the validation set approach with a 50/50 training/test split to evaluate the predictions from the logistic regression. Choose a threshold for when an estimated probability should lead to a prediction of a claim (imagine cost FN > FP). Compute a confusion matrix and argue for your choice of threshold. Set the seed to 1234 (Marking guide seed was different, here 1234).}
        \paragraph{Base R Solution}
\begin{lstlisting}[]
# Using xdata_task3_orig_car from 3a
set.seed(1234) 
n_task3 <- nrow(xdata_task3_orig_car)
train_idx_task3 <- sample(1:n_task3, floor(n_task3/2))
train_data_task3 <- xdata_task3_orig_car[train_idx_task3,]
test_data_task3  <- xdata_task3_orig_car[-train_idx_task3,]

logit_train_task3 <- glm(clm ~ veh_body + agecat + veh_value + exposure, 
                         data=train_data_task3, family=binomial)
probs_test_task3 <- predict(logit_train_task3, newdata=test_data_task3, type="response")

# Threshold choice (e.g., 0.07 as in marking scheme to catch more actual claims)
# Motivation: Overall claim rate is low (~6.8%). A standard 0.5 threshold will predict
# very few claims. To reduce False Negatives (costly if missing an actual claim),
# lower the threshold. This increases True Positives but also False Positives.
chosen_threshold_task3 <- 0.07 
pred_class_task3_logistic <- ifelse(probs_test_task3 > chosen_threshold_task3, 
                                   levels(train_data_task3$clm)[2], # "1" or "Yes"
                                   levels(train_data_task3$clm)[1]) # "0" or "No"
pred_class_task3_logistic <- factor(pred_class_task3_logistic, levels=levels(train_data_task3$clm))

conf_matrix_task3_logistic <- table(Actual = test_data_task3$clm, Predicted = pred_class_task3_logistic)
print(conf_matrix_task3_logistic)
# print(prop.table(conf_matrix_task3_logistic, margin=1)) # Row percentages (Sensitivity, Specificity)
# Interpretation: "With a threshold of 0.07, we correctly identify [Sensitivity]% of actual claims,
# while incorrectly classifying [1-Specificity]% of non-claims as claims. This trade-off was
# chosen to prioritize detecting claims."
\end{lstlisting}

    \subsubsection{3d: Boosted trees for \Rcode{clm}}
        \paragraph{Question}
        \textit{Use boosted trees to predict \Rcode{clm} and evaluate the predictions with the same approach as in 3c.}
        \paragraph{Base R Solution (using \Rpackage{gbm})}
\begin{lstlisting}[]
library(gbm)
# Using train_data_task3, test_data_task3, chosen_threshold_task3 from 3c
# GBM requires numeric 0/1 response for bernoulli
train_data_task3$clm_numeric <- as.numeric(train_data_task3$clm) - 1
test_data_task3$clm_numeric  <- as.numeric(test_data_task3$clm) - 1

formula_gbm_task3 <- clm_numeric ~ veh_body + agecat + veh_value + exposure

set.seed(1234) # For reproducibility
boost_model_task3 <- gbm(formula_gbm_task3,
                         data=train_data_task3,
                         distribution="bernoulli",
                         n.trees=100, # Marking scheme suggests 100 trees
                         interaction.depth=4, # Example, tune this
                         shrinkage=0.01)      # Example, tune this

probs_boost_task3_test <- predict(boost_model_task3, newdata=test_data_task3, 
                                  n.trees=100, type="response")
pred_class_boost_task3 <- ifelse(probs_boost_task3_test > chosen_threshold_task3, 1, 0)

conf_matrix_boost_task3 <- table(Actual = test_data_task3$clm_numeric, Predicted = pred_class_boost_task3)
print(conf_matrix_boost_task3)
# print(prop.table(conf_matrix_boost_task3, margin=1))
# Compare with logistic regression from 3c. Marking scheme: GBM improved TP rate (0.67 vs 0.59).
\end{lstlisting}


\subsection{Exam Spring 2022}
Overall Datasets: \Robject{OJ} (\Rpackage{ISLR}), \Robject{Computers} (\Rpackage{Ecdat})

\textbf{\Large Task 1: Analysis of \Robject{OJ} Data - Predicting \Robject{Purchase}}
\textit{(Assume \Rcode{LoyalCH} unavailable except for 1f)}
\vspace{0.5em}

    \subsubsection{1a: Data preparation and splitting}
        \paragraph{Question}
        \textit{Remove the variable \Rcode{LoyalCH}. If necessary, recode categorical variables as factors and remove variables that cannot be used in the analysis. Base your reasoning to do this on the help function and by looking at the data. Split the data in a 50/50 split to create a training and a test dataset. Use the seed 8655, when you split the data, \Rcode{set.seed(8655)}.}
        \paragraph{Base R Solution (basert på løsningsforslag)}
\begin{lstlisting}[]
library(ISLR)
data(OJ)

# Remove LoyalCH and StoreID (StoreID is same as STORE after factor conversion)
oj_data_prep <- OJ[, !(names(OJ) %in% c("LoyalCH", "StoreID"))]

# Identify variables to convert to factor (<= 5 unique values in solution)
# card_func <- function(x) length(table(x))
# len_oj <- sapply(oj_data_prep, card_func)
# print(len_oj) 
# Purchase:2, SpecialCH:2, SpecialMM:2, Store7:2, STORE:5 should become factors
# WeekofPurchase:52, PriceCH:10, PriceMM:8, DiscCH:12, DiscMM:12, etc. are numeric

for (col_name in names(oj_data_prep)) {
  if (length(unique(oj_data_prep[[col_name]])) <= 5) {
    oj_data_prep[[col_name]] <- as.factor(oj_data_prep[[col_name]])
  }
}
# Verify: sapply(oj_data_prep, class)

# Split data
set.seed(8655) # As specified in this exam version
n_oj <- nrow(oj_data_prep)
train_idx_oj <- sample(1:n_oj, floor(n_oj/2))
train_oj <- oj_data_prep[train_idx_oj, ]
test_oj <- oj_data_prep[-train_idx_oj, ]

# print(paste("Train OJ dimensions:", paste(dim(train_oj), collapse="x")))
# print(paste("Test OJ dimensions:", paste(dim(test_oj), collapse="x")))
# Motivation: LoyalCH removed as per instruction. StoreID removed as STORE factor captures same info.
# Variables with few unique values and no inherent order converted to factors.
\end{lstlisting}
        \paragraph{Tidyverse Solution}
\begin{lstlisting}
library(ISLR); library(dplyr); library(purrr)
data(OJ)

oj_data_prep_tidy <- OJ %>%
  select(-LoyalCH, -StoreID) %>%
  mutate(across(where(~length(unique(.)) <= 5), as.factor))

set.seed(8655)
train_idx_oj_tidy <- sample(1:nrow(oj_data_prep_tidy), floor(nrow(oj_data_prep_tidy)/2))
train_oj_tidy <- oj_data_prep_tidy[train_idx_oj_tidy, ]
test_oj_tidy <- oj_data_prep_tidy[-train_idx_oj_tidy, ]
# glimpse(train_oj_tidy)
\end{lstlisting}

    \subsubsection{1b: Descriptive statistics for promising predictors}
        \paragraph{Question}
        \textit{Use descriptive statistics to detect promising predictors for \Rcode{Purchase}. Present the most interesting results as tables and graphs and comment on them.}
        \paragraph{Base R / Tidyverse Solution (Conceptual - Løsningsforslag fokuserer på plott og `by`)}
\begin{lstlisting}
# Assuming train_oj from 1a
# Visual exploration (Base R)
# par(mfrow=c(1,2)) # Example for specific plots
# plot(Purchase ~ STORE, data=train_oj, main="Purchase by Store")
# boxplot(PriceDiff ~ Purchase, data=train_oj, main="PriceDiff by Purchase")
# graphics.off()

# Numerical summaries (Base R)
# cat("Mean PctDiscCH by Purchase:\n")
# print(by(train_oj$PctDiscCH, train_oj$Purchase, mean, na.rm=TRUE))
# cat("Mean PctDiscMM by Purchase:\n")
# print(by(train_oj$PctDiscMM, train_oj$Purchase, mean, na.rm=TRUE))
# cat("Table of SpecialMM by Purchase:\n")
# print(table(train_oj$SpecialMM, train_oj$Purchase))

# Tidyverse for summaries
# library(dplyr)
# train_oj %>% group_by(Purchase) %>% 
#   summarise(across(c(PriceDiff, PctDiscCH, PctDiscMM), 
#                    list(mean=mean, median=median), na.rm=TRUE))
# train_oj %>% count(STORE, Purchase) %>% group_by(STORE) %>% mutate(prop = n/sum(n))

# Comments (basert på løsningsforslag):
# - STORE: Variation in brand purchased by store.
# - PriceDiff: Higher (MM more expensive) when CH bought.
# - PctDiscCH/PctDiscMM: Discounts seem to influence choice.
# - SpecialMM: Appears differently distributed between CH/MM purchases.
\end{lstlisting}

    \subsubsection{1c: Logistic regression}
        \paragraph{Question}
        \textit{Fit a logistic regression of \Rcode{Purchase} on all the variables you found promising in 1b, evaluate the predictions using test accuracy and a confusion matrix. If you, in the process, detect that you would modify your model, you should do so and explain why. Use the threshold 0.5 when you go from estimated probability to a prediction.}
        \paragraph{Base R Solution (Løsningsforslag bruker \Rcode{PriceDiff + STORE})}
\begin{lstlisting}
# Assuming train_oj, test_oj from 1a
# Løsningsforslag ender opp med PriceDiff og STORE etter å ha vurdert PctDisc-variablene som ikke-signifikante.
# Vi starter med de 'lovende' og fjerner evt. ikke-signifikante.
# Initial promising: STORE, PriceDiff, PctDiscCH, PctDiscMM, SpecialMM
# logit_oj_initial <- glm(Purchase ~ STORE + PriceDiff + PctDiscCH + PctDiscMM + SpecialMM, 
#                         data=train_oj, family=binomial)
# summary(logit_oj_initial) 
# % If PctDiscCH, PctDiscMM, SpecialMM are not significant, remove them. %

# Final model from solution:
logit_oj_final <- glm(Purchase ~ PriceDiff + STORE, data=train_oj, family=binomial)
# print(summary(logit_oj_final))

# Predictions on test set
probs_logit_oj_test <- predict(logit_oj_final, newdata=test_oj, type="response")
pred_logit_oj_test <- ifelse(probs_logit_oj_test > 0.5, "MM", "CH") # OJ levels: CH, MM
pred_logit_oj_test <- factor(pred_logit_oj_test, levels = levels(test_oj$Purchase))

# Confusion matrix and accuracy
conf_matrix_logit_oj <- table(Actual=test_oj$Purchase, Predicted=pred_logit_oj_test)
print(conf_matrix_logit_oj)
accuracy_logit_oj <- sum(diag(conf_matrix_logit_oj)) / sum(conf_matrix_logit_oj)
cat("Test Accuracy (Logistic Regression):", accuracy_logit_oj, "\n")
# prop_table_logit_oj <- prop.table(conf_matrix_logit_oj, margin=1)
# print(round(prop_table_logit_oj,3))
# Interpretation: The model predicts [value]% of CH purchases correctly and [value]% of MM purchases.
\end{lstlisting}

    \subsubsection{1d: Classification tree}
        \paragraph{Question}
        \textit{Fit a classification tree, using all variables (except the one(s) you removed in 1a), and plot it. Give an example of how to interpret a branch of the tree. Also, use the same evaluation measures as in 1c and evaluate the predictions. Use the threshold 0.5.}
        \paragraph{Base R Solution (using \Rpackage{tree})}
\begin{lstlisting}[]
library(tree)
# Assuming train_oj, test_oj from 1a (with all original predictors kept, except LoyalCH, StoreID)
tree_oj <- tree(Purchase ~ ., data=train_oj)
# summary(tree_oj)
# plot(tree_oj); text(tree_oj, pretty=0)

# Interpretation example (basert på løsningsforslagets trefigur):
# "If STORE is 'ae' (0 or 4 in factor levels) AND PriceDiff < -0.35 AND WeekofPurchase < 258.5,
# then the prediction is MM." (Values vil variere basert på seed).

# Predictions on test set
# predict() for tree with type="class" gives direct class predictions.
# For 0.5 threshold on probabilities, get probabilities first:
probs_tree_oj_test <- predict(tree_oj, newdata=test_oj, type="vector") # Gives matrix of probs
# Assuming Purchase is factor with levels CH, MM. probs_tree_oj_test[,2] is P(MM)
pred_tree_oj_test <- ifelse(probs_tree_oj_test[,2] > 0.5, "MM", "CH") 
pred_tree_oj_test <- factor(pred_tree_oj_test, levels = levels(test_oj$Purchase))

conf_matrix_tree_oj <- table(Actual=test_oj$Purchase, Predicted=pred_tree_oj_test)
print(conf_matrix_tree_oj)
accuracy_tree_oj <- sum(diag(conf_matrix_tree_oj)) / sum(conf_matrix_tree_oj)
cat("Test Accuracy (Classification Tree):", accuracy_tree_oj, "\n")
# prop_table_tree_oj <- prop.table(conf_matrix_tree_oj, margin=1)
# print(round(prop_table_tree_oj,3))
\end{lstlisting}

    \subsubsection{1e: Cross-validation for optimal threshold for tree}
        \paragraph{Question}
        \textit{In 1c and 1d you have used the threshold 0.5. Use cross-validation to find the threshold that maximizes the accuracy for the tree. Evaluate predictions from the tree using the best threshold. Are the predictions improved? Use the seed 4598 when you split the data for CV (this implies an inner CV split on the original training data).}
        \paragraph{Base R Solution (Validation set for threshold tuning on original training data)}
\begin{lstlisting}
# Using original train_oj and test_oj from 1a.
# To find optimal threshold, we need a validation set, split from original train_oj.
set.seed(4598) # Seed for splitting train_oj further
n_train_oj <- nrow(train_oj)
val_idx_oj <- sample(1:n_train_oj, floor(n_train_oj/2))
train2_oj <- train_oj[val_idx_oj, ]
valid_oj <- train_oj[-val_idx_oj, ]

# Fit tree on train2_oj
tree_for_thresh <- tree(Purchase ~ ., data=train2_oj)
probs_valid_oj <- predict(tree_for_thresh, newdata=valid_oj, type="vector")[,2] # P(MM)

thresholds <- seq(0.1, 0.9, by=0.05)
accuracies_thresh <- numeric(length(thresholds))

for (i in 1:length(thresholds)) {
  th <- thresholds[i]
  pred_class_thresh <- ifelse(probs_valid_oj > th, "MM", "CH")
  pred_class_thresh <- factor(pred_class_thresh, levels = levels(valid_oj$Purchase))
  accuracies_thresh[i] <- mean(pred_class_thresh == valid_oj$Purchase)
}

# plot(thresholds, accuracies_thresh, type="b", xlab="Threshold", ylab="Validation Accuracy")
best_threshold_idx <- which.max(accuracies_thresh)
optimal_threshold_oj <- thresholds[best_threshold_idx]
cat("Optimal threshold found:", optimal_threshold_oj, "\n")

# Evaluate original tree_oj (fit on full train_oj) on test_oj using this optimal_threshold_oj
probs_tree_fulltrain_test <- predict(tree_oj, newdata=test_oj, type="vector")[,2] # P(MM)
pred_tree_optimal_thresh <- ifelse(probs_tree_fulltrain_test > optimal_threshold_oj, "MM", "CH")
pred_tree_optimal_thresh <- factor(pred_tree_optimal_thresh, levels=levels(test_oj$Purchase))

conf_matrix_tree_optimal <- table(Actual=test_oj$Purchase, Predicted=pred_tree_optimal_thresh)
print(conf_matrix_tree_optimal)
accuracy_tree_optimal <- sum(diag(conf_matrix_tree_optimal)) / sum(conf_matrix_tree_optimal)
cat("Test Accuracy (Tree with Optimal Threshold):", accuracy_tree_optimal, "\n")
# Compare accuracy_tree_optimal with accuracy_tree_oj (from 1d using 0.5 threshold)
# Marking guide's plot actually uses the same tree (tree1) for validation, which is fine.
# Their plot shows threshold around 0.4-0.6 being optimal. Accuracy doesn't improve much.
\end{lstlisting}

    \subsubsection{1f: Analysis with \Rcode{LoyalCH}}
        \paragraph{Question}
        \textit{In a) you removed the variable \Rcode{LoyalCH}. Imagine now that this variable is actually available at the time of prediction. By augmenting your analysis with this variable both for the logistic regression and the classification tree, conclude if this variable contribute to predicting \Rcode{Purchase}.}
        \paragraph{Solution Outline}
        \textit{Repeat steps 1a-1d, but KEEP LoyalCH in the dataset. Compare test accuracies and model summaries.}
\begin{lstlisting}
# 1. Data Prep: Start with original OJ, remove only StoreID. Convert factors. Split.
#    oj_with_loyalch <- OJ[, names(OJ) != "StoreID"]
#    ... convert factors, split into train_oj_L, test_oj_L ...

# 2. Logistic Regression with LoyalCH:
#    logit_oj_L <- glm(Purchase ~ PriceDiff + STORE + LoyalCH, data=train_oj_L, family=binomial)
#    % (Or include all promising predictors + LoyalCH) %
#    summary(logit_oj_L) % Check significance of LoyalCH %
#    % Calculate test accuracy and confusion matrix on test_oj_L %
#    % Compare accuracy with result from 1c. %

# 3. Classification Tree with LoyalCH:
#    tree_oj_L <- tree(Purchase ~ ., data=train_oj_L)
#    summary(tree_oj_L) % See if LoyalCH is used in splits %
#    plot(tree_oj_L); text(tree_oj_L, pretty=0)
#    % Calculate test accuracy and confusion matrix on test_oj_L %
#    % Compare accuracy with result from 1d. %

# Conclusion: If LoyalCH is significant in logistic regression and/or used prominently
# in the tree, and if test accuracies improve notably, then LoyalCH contributes.
# (LoyalCH is expected to be a very strong predictor).
\end{lstlisting}

\vspace{1em}
\textbf{\Large Task 2: Analysis of \Robject{Computers} Data - Predicting \Robject{price}}
\vspace{0.5em}

    \subsubsection{2a: Data preparation and splitting}
        \paragraph{Question}
        \textit{Just as in 1a; if necessary, recode categorical variables as factors and remove variables that cannot be used in the analysis. Base your reasoning to do this on the help function and by looking at the data. Split the data in a 50/50 split to create a training and a test dataset. Use the seed 4598, when you split the data, \Rcode{set.seed(4598)}.}
        \paragraph{Base R Solution (basert på løsningsforslag)}
\begin{lstlisting}[]
library(Ecdat)
data(Computers)

# Recode categoricals (<= 5 unique values, from marking scheme)
# Inspect: sapply(Computers, function(x) length(unique(x)))
# screen (3), cd (2), multi (2), premium (2) should become factors
cols_to_factor_comp <- c("screen", "cd", "multi", "premium")
for (col in cols_to_factor_comp) {
  Computers[[col]] <- as.factor(Computers[[col]])
}
# sapply(Computers, class) # Verify

# Split data
set.seed(4598)
n_comp <- nrow(Computers)
train_idx_comp <- sample(1:n_comp, floor(n_comp/2))
train_comp <- Computers[train_idx_comp, ]
test_comp <- Computers[-train_idx_comp, ]
# print(paste("Train Computers dimensions:", paste(dim(train_comp), collapse="x")))
\end{lstlisting}
        \paragraph{Tidyverse Solution}
\begin{lstlisting}
library(Ecdat); library(dplyr)
data(Computers)
Computers_prep_tidy <- Computers %>%
  mutate(across(all_of(c("screen", "cd", "multi", "premium")), as.factor))

set.seed(4598)
train_idx_comp_tidy <- sample(1:nrow(Computers_prep_tidy), floor(nrow(Computers_prep_tidy)/2))
train_comp_tidy <- Computers_prep_tidy[train_idx_comp_tidy, ]
test_comp_tidy <- Computers_prep_tidy[-train_idx_comp_tidy, ]
# glimpse(train_comp_tidy)
\end{lstlisting}

    \subsubsection{2b: Descriptive statistics for promising predictors}
        \paragraph{Question}
        \textit{Use descriptive statistics to find promising predictors. Comment on your observations.}
        \paragraph{Base R / Tidyverse Solution (Conceptual - Løsningsforslag bruker \Rcode{plot(price~.,data=train)})}
\begin{lstlisting}
# Assuming train_comp from 2a
# Visual exploration (Base R)
# par(mfrow=c(3,3)) # Adjust layout as needed
# plot(price ~ ., data=train_comp) # As per marking guide
# graphics.off()
# Comment: "Speed, hd, ram show positive trends with price. Screen17 has higher price than screen15.
# cd=yes, multi=yes, premium=yes associated with higher price. Ads and trend also show relationships."

# Tidyverse for specific plots
# library(ggplot2)
# ggplot(train_comp, aes(x=speed, y=price)) + geom_point() + geom_smooth()
# ggplot(train_comp, aes(x=ram, y=price)) + geom_point() + geom_smooth()
# ggplot(train_comp, aes(x=screen, y=price)) + geom_boxplot()
\end{lstlisting}

    \subsubsection{2c: Linear regression for \Rcode{price}}
        \paragraph{Question}
        \textit{Run a linear regression on all explanatory variables. Interpret some of the coefficients. Evaluate the prediction performance on an appropriate measure.}
        \paragraph{Base R Solution}
\begin{lstlisting}
# Assuming train_comp, test_comp from 2a
lm_comp_full <- lm(price ~ ., data=train_comp)
summary_lm_comp <- summary(lm_comp_full)
print(summary_lm_comp)
# Interpretation: e.g., "A 1MHz increase in speed is associated with an expected price
# increase of [coef_speed] dollars, holding other factors constant (p=[p_val])."
# "Having a CD-ROM (cdyes) is associated with an expected price increase of 
# [coef_cdyes] compared to no CD-ROM, holding others constant (p=[p_val])."

# Evaluate performance: Test MSE
pred_lm_comp_test <- predict(lm_comp_full, newdata=test_comp)
mse_lm_comp_test <- mean((test_comp$price - pred_lm_comp_test)^2, na.rm=TRUE)
cat("Test MSE (Full LM for Computers):", mse_lm_comp_test, "\n")
# Marking guide result: MSE1 = 72200
\end{lstlisting}

    \subsubsection{2d: Linear regression of \Rcode{log(price)}}
        \paragraph{Question}
        \textit{Investigate if the predictions of \Rcode{price} would be better if we use a linear regression of \Rcode{log(price)} on the other variables.}
        \paragraph{Base R Solution}
\begin{lstlisting}
# Create log(price) variable, ensuring it's available in train and test
# train_comp_log <- train_comp
# test_comp_log <- test_comp
# train_comp_log$log_price <- log(train_comp_log$price)
# test_comp_log$log_price <- log(test_comp_log$price) # For actual values if needed, but predict() on log scale

# Fit model on training data using log_price, excluding original price
# lm_comp_log <- lm(log(price) ~ . - price, data=train_comp) # Simpler way
lm_comp_log <- lm(log(price) ~ speed + hd + ram + screen + cd + multi + premium + ads + trend, 
                  data=train_comp)
summary(lm_comp_log)

# Predict on test set (predictions will be on log scale)
pred_log_price_test <- predict(lm_comp_log, newdata=test_comp)
# Transform predictions back to original price scale
pred_price_from_log_test <- exp(pred_log_price_test)

# Calculate Test MSE on original price scale
mse_loglm_comp_test <- mean((test_comp$price - pred_price_from_log_test)^2, na.rm=TRUE)
cat("Test MSE (Log-Linear LM for Computers):", mse_loglm_comp_test, "\n")
# Marking guide result: MSE2 = 72610.7. In this case, log transform didn't improve MSE.
\end{lstlisting}

    \subsubsection{2e: GAM for \Rcode{price}}
        \paragraph{Question}
        \textit{In 2d you fitted a model where \Rcode{price} is a particular nonlinear function of the other variables. You should now investigate another non-linear models. First, fit a GAM-model, plot the result and evaluate the predictions. Is there a reason to not allow for some variables to have a non-linear relationship with \Rcode{price}?}
        \paragraph{Base R Solution (using \Rpackage{gam})}
\begin{lstlisting}
library(gam)
# Assuming train_comp, test_comp from 2a
# Formula for GAM: s() for continuous, direct for factors
# From marking guide: s(speed)+s(hd)+s(ram)+screen+cd+multi+premium+s(ads)+s(trend)
gam_comp1 <- gam(price ~ s(speed, df=4) + s(hd, df=4) + s(ram, df=4) + screen + cd + multi +
                 premium + s(ads, df=4) + s(trend, df=4), data=train_comp)
# summary(gam_comp1)

# Plot partial effects
# par(mfrow=c(3,3)) # To fit all plots, adjust as needed
# plot(gam_comp1, se=TRUE, col="blue", ask=TRUE)
# Reason to allow non-linear: Some relationships (e.g. price vs hd or speed) might
# exhibit diminishing returns or other non-linear patterns. Plots will confirm.

# Evaluate Test MSE
pred_gam_comp_test <- predict(gam_comp1, newdata=test_comp)
mse_gam_comp_test <- mean((test_comp$price - pred_gam_comp_test)^2, na.rm=TRUE)
cat("Test MSE (GAM for Computers):", mse_gam_comp_test, "\n")
# Marking guide result: MSE3 = 60947.91. GAM significantly improved predictions.
\end{lstlisting}

    \subsubsection{2f: Explanation of backfitting for GAMs}
        \paragraph{Question}
        \textit{Give a brief explanation of how backfitting works and in what situations it is possible to fit a generalized additive model with ordinary least square regression.}
        \paragraph{Explanation}
        \textit{
        \textbf{Backfitting Algorithm}:
        For a GAM model $y_i = \beta_0 + \sum_{j=1}^p f_j(x_{ij}) + \epsilon_i$:
        \begin{enumerate}
            \item Initialize: $\hat{\beta}_0 = \bar{y}$, and $\hat{f}_j(x_{ij}) = 0$ for all $j$ (or some initial guess, e.g., from linear model).
            \item Cycle through predictors $j=1, \dots, p, 1, \dots, p, \dots$ until convergence:
                \begin{itemize}
                    \item For current predictor $j$, compute partial residuals: $r_{ij} = y_i - \hat{\beta}_0 - \sum_{k \neq j} \hat{f}_k(x_{ik})$.
                    \item Update $\hat{f}_j$ by fitting a smoother (e.g., spline, loess) to the partial residuals $r_{ij}$ against $x_{ij}$. That is, $\hat{f}_j \leftarrow \text{smoother}(r_{.j} \sim x_{.j})$.
                    \item Often, $\hat{f}_j$ are centered to ensure identifiability.
                \end{itemize}
            \item Convergence is reached when the functions $\hat{f}_j$ do not change much between iterations.
        \end{enumerate}
        \textbf{GAM with OLS}:
        A GAM can be fitted with OLS if each function $f_j(x_j)$ can be represented by a linear combination of a set of basis functions. For example:
        \begin{itemize}
            \item \textbf{Polynomial Regression}: $f_j(x_j) = \beta_{j1}x_j + \beta_{j2}x_j^2 + \dots$. The "basis functions" are $x_j, x_j^2, \dots$.
            \item \textbf{Regression Splines}: $f_j(x_j)$ is represented by a sum of spline basis functions (e.g., truncated power basis, B-spline basis). The model $y = \beta_0 + \sum_j \sum_l \beta_{jl} b_{jl}(x_j) + \epsilon$ is then linear in the $\beta_{jl}$ coefficients and can be fit with OLS.
        \end{itemize}
        If non-parametric smoothers like smoothing splines (where $\lambda$ is chosen by GCV) or local regression are used within backfitting, the overall procedure is not a single OLS fit, but each step of backfitting might involve an OLS-like weighted least squares or penalized least squares.
        }

    \subsubsection{2g: Bagged trees for \Rcode{price}}
        \paragraph{Question}
        \textit{Use bagged trees to predict \Rcode{price}. Evaluate the predictions. Compute variable importance measures and comment on the results. Explain the logic behind the measure for variable importance that you are using. (Hint: If the computation takes a long time you can reduce the number of trees fitted using the argument \Rcode{ntree}.)}
        \paragraph{Base R Solution (using \Rpackage{randomForest})}
\begin{lstlisting}
library(randomForest)
# Assuming train_comp, test_comp from 2a
# For bagging, mtry = number of predictors.
# price ~ . means all other columns are predictors.
# ncol(train_comp) - 1 (if price is a column, and all others are predictors)
# If train_comp has 10 predictors plus 'price' and 'logp' (from 2d), p=10.
# Marking guide uses mtry=9, perhaps 'logp' was still in data. Assuming p=10 here.
num_predictors_comp <- ncol(train_comp) - 1 # Assuming price is the only response
if ("logp" %in% names(train_comp)) num_predictors_comp <- num_predictors_comp -1

set.seed(4598) # For reproducibility
bag_comp <- randomForest(price ~ . - logp, data=train_comp, # Exclude logp if it exists
                         mtry=num_predictors_comp, 
                         ntree=100, # Reduced from default 500 for speed as hinted
                         importance=TRUE)
# print(bag_comp)

# Evaluate predictions
pred_bag_comp_test <- predict(bag_comp, newdata=test_comp)
mse_bag_comp_test <- mean((test_comp$price - pred_bag_comp_test)^2, na.rm=TRUE)
cat("Test MSE (Bagging for Computers):", mse_bag_comp_test, "\n")
# Marking guide result: MSE4 = 28272.51 (significant improvement)

# Variable Importance
importance_bag_comp <- importance(bag_comp)
print(importance_bag_comp)
varImpPlot(bag_comp)
# Logic for IncNodePurity (default for regression):
# "IncNodePurity (or %IncMSE if type=1 is used) measures the total decrease in node impurity
# (RSS for regression trees) from splitting on that variable, averaged over all trees.
# A higher value indicates greater importance."
# Comment: "Ram, trend, speed, hd appear as most important variables."
\end{lstlisting}

    \subsubsection{2h: Explanation of bagging}
        \paragraph{Question}
        \textit{Give a brief explanation of bagging.}
        \paragraph{Explanation}
        \textit{
        \textbf{Bagging (Bootstrap Aggregating)}:
        Bagging is an ensemble learning technique designed to improve the stability and accuracy of machine learning algorithms, particularly those with high variance like decision trees. The process involves:
        \begin{enumerate}
            \item \textbf{Bootstrap Sampling}: Create $B$ new training datasets by sampling with replacement from the original training dataset. Each bootstrap sample has the same size as the original.
            \item \textbf{Model Fitting}: Fit a separate model (e.g., a decision tree) independently to each of the $B$ bootstrap samples. These trees are typically grown deep and not pruned.
            \item \textbf{Aggregation}:
                \begin{itemize}
                    \item For \textbf{regression}, the predictions from the $B$ models are averaged to get the final prediction.
                    \item For \textbf{classification}, a majority vote is taken among the $B$ model predictions.
                \end{itemize}
        \end{enumerate}
        The main benefit of bagging is variance reduction. By averaging many (potentially noisy) models fit to slightly different versions of the data, the overall variance of the ensemble prediction is reduced, often leading to improved predictive performance.
        }


\subsection{Exam Spring 2023}
Overall Dataset: \Robject{Churn.csv} (Provided on Canvas)
    \textit{Refer to variable definitions table at the end of this exam section.}

\textbf{\Large Task 1: Predict Average Bill (\Rcode{bill\_avg})}
\vspace{0.5em}

    \subsubsection{1a: Data preparation and splitting}
        \paragraph{Question}
        \textit{If necessary, recode categorical variables as factors and remove variables that cannot be used in the analysis. Also, remove variables if you mean that they are unreasonable to include. One such reason could be that a variable is not available at the time the prediction is made but there might be other reasons too. Motivate your choices in words. Split the data in a 50/50 split to create a training and a test dataset. Use the seed 86554354, when you split the data, \Rcode{set.seed(86554354)}.}
        \paragraph{Base R Solution (basert på løsningsforslag)}
\begin{lstlisting}
# Churn <- read.csv("Churn.csv") % \textit{(Assuming data is read)} %

# Remove id
xdata_churn <- Churn[, -1] % \textit{Assuming 'id' is the first column} %

# Recode specified variables to factors
factor_cols_churn <- c("is_tv_subscriber", "is_movie_package_subscriber", "churn")
for (col in factor_cols_churn) {
  xdata_churn[[col]] <- as.factor(xdata_churn[[col]])
}
# print(sapply(xdata_churn, class))

# Split data
set.seed(86554354) # Seed from exam question (v1 from marking instructions might differ)
n_churn <- nrow(xdata_churn)
train_idx_churn <- sample(1:n_churn, floor(n_churn/2))
train_churn <- xdata_churn[train_idx_churn, ]
test_churn <- xdata_churn[-train_idx_churn, ]

# print(paste("Train Churn dimensions:", paste(dim(train_churn), collapse="x")))
# print(paste("Test Churn dimensions:", paste(dim(test_churn), collapse="x")))
# Motivation: 'id' is an identifier and not predictive. 'is_tv_subscriber', 
# 'is_movie_package_subscriber', and 'churn' are categorical (0/1 originally)
# and explicitly converted to factors for proper handling by modeling functions.
# The question of whether 'churn' is available when predicting 'bill_avg' is raised
# in the marking scheme; it's assumed available for this task.
\end{lstlisting}
        \paragraph{Tidyverse Solution}
\begin{lstlisting}
# library(dplyr)
# Churn <- read.csv("Churn.csv")
# 
# xdata_churn_tidy <- Churn %>%
#   select(-id) %>%
#   mutate(across(all_of(c("is_tv_subscriber", "is_movie_package_subscriber", "churn")), as.factor))
# 
# set.seed(86554354)
# train_idx_churn_tidy <- sample(1:nrow(xdata_churn_tidy), floor(nrow(xdata_churn_tidy)/2))
# train_churn_tidy <- xdata_churn_tidy[train_idx_churn_tidy, ]
# test_churn_tidy <- xdata_churn_tidy[-train_idx_churn_tidy, ]
# glimpse(train_churn_tidy)
\end{lstlisting}

    \subsubsection{1b: Descriptive methods for \Rcode{bill\_avg}}
        \paragraph{Question}
        \textit{Use descriptive methods to find useful predictors for \Rcode{bill\_avg}. Write in words which R functions you used, present the most interesting results as tables and graphs and comment on them.}
        \paragraph{Base R / Tidyverse Solution (Conceptual - Marking guide uses \Rcode{plot(bill\_avg~., data=train)})}
\begin{lstlisting}
# Assuming train_churn from 1a
# Visual exploration (Base R as per marking guide)
# par(mfrow=c(3,3)) # To fit all plots (9 predictors excluding id and bill_avg itself)
# plot(bill_avg ~ ., data=train_churn)
# graphics.off()
# Comments (based on marking guide comment):
# "bill_avg seems to vary with all predictors, possibly with the exception of
# download_avg and upload_avg, where patterns are difficult to see visually."
# For factors like 'is_tv_subscriber', 'churn', boxplots are generated by plot(y~.).
# For numeric like 'subscription_age', scatterplots are generated.

# Tidyverse for specific plots (examples)
# library(ggplot2)
# ggplot(train_churn, aes(x=is_tv_subscriber, y=bill_avg)) + geom_boxplot()
# ggplot(train_churn, aes(x=subscription_age, y=bill_avg)) + geom_point() + geom_smooth()
\end{lstlisting}

    \subsubsection{1c: Best OLS for \Rcode{bill\_avg}}
        \paragraph{Question}
        \textit{Produce the best possible predictions of \Rcode{bill\_avg} using standard linear regression (OLS) and evaluate them. Motivate your choices of variables, how you evaluate the predictions and the evaluation measure used.}
        \paragraph{Base R Solution (Comparing full model vs. model without download/upload\_avg)}
\begin{lstlisting}
# Assuming train_churn, test_churn from 1a
# Model 1: Full model with all predictors
ols_full_churn <- lm(bill_avg ~ ., data=train_churn)
# summary(ols_full_churn)
pred_ols_full_churn <- predict(ols_full_churn, newdata=test_churn)
mse_ols_full_churn <- mean((test_churn$bill_avg - pred_ols_full_churn)^2, na.rm=TRUE)
cat("Test MSE (Full OLS for bill_avg):", mse_ols_full_churn, "\n")
# Marking guide value: 117.8603

# Model 2: Reduced model (based on 1b, removing download_avg, upload_avg)
ols_reduced_churn <- lm(bill_avg ~ . - download_avg - upload_avg, data=train_churn)
# summary(ols_reduced_churn)
pred_ols_reduced_churn <- predict(ols_reduced_churn, newdata=test_churn)
mse_ols_reduced_churn <- mean((test_churn$bill_avg - pred_ols_reduced_churn)^2, na.rm=TRUE)
cat("Test MSE (Reduced OLS for bill_avg):", mse_ols_reduced_churn, "\n")
# Marking guide value: 146.7376

# Motivation:
# Variables: Compared a full model with one excluding download_avg and upload_avg, based on visual
# inspection in 1b suggesting weak relationships.
# Evaluation Measure: Test MSE is used to assess predictive accuracy on unseen data.
# A lower Test MSE indicates better out-of-sample prediction.
# Conclusion: The full model performed much better (lower Test MSE) than the reduced model.
\end{lstlisting}

    \subsubsection{1d: LASSO regression for \Rcode{bill\_avg}}
        \paragraph{Question}
        \textit{Fit a LASSO regression with all variables. Here you should use the tools you have learned to find appropriate tuning parameters. Are you standardizing the predictors or not? Motivate your choice. Compare the estimated coefficients with the estimated coefficient from an OLS regression on all predictors. (Hint: Using categorical (factor) variables in LASSO, you will have to create dummy variables. \Rfunction{dummy\_cols} from \Rpackage{fastDummies} or \Rfunction{model.matrix} can be used. \Rfunction{glmnet} requires matrices as input).}
        \paragraph{Base R Solution (using \Rpackage{glmnet})}
\begin{lstlisting}
library(glmnet)
# Assuming train_churn, test_churn from 1a. Factors are already created.
# For glmnet, factors need to be converted to dummy variables.
# We use the processed data before splitting for model.matrix, then split X and y.
# xdata_churn was defined in 1a with factors.

# Create model matrix for predictors (converts factors to dummies)
x_matrix_churn <- model.matrix(bill_avg ~ . -1, data=xdata_churn) # -1 removes intercept if glmnet adds it
y_vector_churn <- xdata_churn$bill_avg

# Split X and y matrices into train/test using train_idx_churn from 1a
trainX_churn <- x_matrix_churn[train_idx_churn, ]
trainY_churn <- y_vector_churn[train_idx_churn]
testX_churn <- x_matrix_churn[-train_idx_churn, ]
testY_churn <- y_vector_churn[-train_idx_churn] % \textit{(Actually test_churn$bill_avg)} %

# Determine lambda by cross-validation (alpha=1 for LASSO)
set.seed(86554354) % \textit{Ensure consistent seed if CV is stochastic} %
cv_lasso_churn <- cv.glmnet(trainX_churn, trainY_churn, alpha=1, standardize=TRUE)
# standardize=TRUE is default and recommended for LASSO/Ridge when variables are on different scales.
# Motivation for standardization: LASSO penalty applies equally to all coefficients. If predictors
# have different scales, those with larger scales might be unfairly penalized or vice versa.
# Standardization puts all predictors on a common scale.

best_lambda_lasso_churn <- cv_lasso_churn$lambda.min
# print(paste("Best lambda (min):", best_lambda_lasso_churn))

# Fit LASSO with optimal lambda on training data
lasso_model_churn <- glmnet(trainX_churn, trainY_churn, alpha=1, lambda=best_lambda_lasso_churn, standardize=TRUE)
lasso_coeffs <- coef(lasso_model_churn)
# print("LASSO Coefficients:")
# print(lasso_coeffs)

# OLS model from 1c (fitted on train_churn, which has factors not dummies)
# ols_full_churn <- lm(bill_avg ~ ., data=train_churn)
# ols_coeffs <- coef(ols_full_churn)
# print("OLS Coefficients (factors handled by lm):")
# print(ols_coeffs)

# Comparison:
# Lasso coefficients are shrunk towards zero. Some might be exactly zero.
# OLS coefficients (for the dummy variables created from factors) will generally be larger in magnitude.
# Marking scheme showed factor variables were converted to 0/1 numeric for glmnet,
# and coefficients were very similar, none set to zero, indicating LASSO restriction not very large.
# This implies xdata in marking scheme was xdata_num after converting factors to numeric.
# If xdata_num <- xdata; for(i in 1:length(fa)) xdata_num[,fa[i]] <- as.numeric(xdata_num[,fa[i]])-1
# Then use xdata_num for trainX, trainy.
# cbind(ols_full_churn$coefficients, lasso_coeffs) % \textit{(Align names carefully for direct comparison)}%
\end{lstlisting}

    \subsubsection{1e: Evaluate LASSO model}
        \paragraph{Question}
        \textit{Compute predictions for \Rcode{bill\_avg} with the model fitted in 1d) and evaluate them with an appropriate measure.}
        \paragraph{Base R Solution}
\begin{lstlisting}
# Assuming lasso_model_churn (fitted on trainX_churn) and testX_churn from 1d
pred_lasso_churn_test <- predict(lasso_model_churn, newx=testX_churn)
mse_lasso_churn_test <- mean((test_churn$bill_avg - pred_lasso_churn_test)^2, na.rm=TRUE)
cat("Test MSE (LASSO for bill_avg):", mse_lasso_churn_test, "\n")
# Marking guide value: 117.8655 (very similar to OLS in this case)
\end{lstlisting}

    \subsubsection{1f: Regression tree for \Rcode{bill\_avg}}
        \paragraph{Question}
        \textit{Fit a regression tree to \Rcode{bill\_avg}. Explain the choices you are making. Interpret the tree.}
        \paragraph{Base R Solution (using \Rpackage{tree})}
\begin{lstlisting}
library(tree)
# Assuming train_churn from 1a (with factors)
# Choices: Using all predictors. Default tree.control parameters initially.
tree_bill_avg <- tree(bill_avg ~ ., data=train_churn)
# summary(tree_bill_avg)
# plot(tree_bill_avg)
# text(tree_bill_avg, pretty=0)

# Interpretation (based on marking scheme tree plot which is complex):
# - Follow branches based on predictor conditions.
# - Terminal nodes give the predicted average bill_avg for observations in that region.
# - Example from marking scheme plot:
#   "Lowest avg bill (4.589) for: download_avg < 333.2 AND download_over_limit < 0.5 AND 
#    upload_avg < 15.25 AND subscription_age < 0.075 (almost new customers)."
#   "Highest avg bill (264.6) for: download_avg >= 333.2 AND is_tv_subscriber=0 (No)
#    AND download_avg >= 861.5."
# Tree structure highlights important variables and interaction-like effects through sequential splits.
\end{lstlisting}

    \subsubsection{1g: Evaluate regression tree}
        \paragraph{Question}
        \textit{Predict \Rcode{bill\_avg} with the regression tree in 1f) and evaluate the predictions.}
        \paragraph{Base R Solution}
\begin{lstlisting}
# Assuming tree_bill_avg from 1f and test_churn from 1a
pred_tree_bill_avg_test <- predict(tree_bill_avg, newdata=test_churn)
mse_tree_bill_avg_test <- mean((test_churn$bill_avg - pred_tree_bill_avg_test)^2, na.rm=TRUE)
cat("Test MSE (Regression Tree for bill_avg):", mse_tree_bill_avg_test, "\n")
# Marking guide value: 120.2793 (comparable to OLS/LASSO, slightly worse)
\end{lstlisting}

    \subsubsection{1h: Random forest for \Rcode{bill\_avg}}
        \paragraph{Question}
        \textit{Fit a random forest to \Rcode{bill\_avg}. Plot a variable importance measure for the predictors, interpret it, and, briefly, explain the measure. (If computations are too slow, reduce \Rcode{ntree} or use smaller training data.)}
        \paragraph{Base R Solution (using \Rpackage{randomForest})}
\begin{lstlisting}
library(randomForest)
# Assuming train_churn from 1a
set.seed(86554354) % \textit{(Consistent seed)} %
# Marking guide used ntree=50 for speed
rf_bill_avg <- randomForest(bill_avg ~ ., data=train_churn, 
                            ntree=50, importance=TRUE, na.action=na.roughfix)
# importance=TRUE is needed for varImpPlot. na.action handles potential NAs if any.

# Variable Importance
# print(importance(rf_bill_avg))
varImpPlot(rf_bill_avg, main="Variable Importance for bill_avg (Random Forest)")
# Explanation of IncNodePurity (default for regression):
# "IncNodePurity measures the total decrease in node impurity (RSS for regression trees)
# from splitting on that variable, averaged over all trees in the forest.
# A higher value indicates that the variable is more important for partitioning the data
# and improving the homogeneity of nodes regarding bill_avg."
# Interpretation: "Based on the plot, [predictor1] and [predictor2] appear most important.
# download_avg and upload_avg, which had unclear patterns in initial plots, now show high
# importance, contrasting with the single tree and OLS findings (from marking guide)."
\end{lstlisting}

    \subsubsection{1i: Evaluate random forest}
        \paragraph{Question}
        \textit{Use the model in 1h) to predict \Rcode{bill\_avg} and evaluate the predictions.}
        \paragraph{Base R Solution}
\begin{lstlisting}
# Assuming rf_bill_avg from 1h and test_churn from 1a
pred_rf_bill_avg_test <- predict(rf_bill_avg, newdata=test_churn, na.action=na.pass)
mse_rf_bill_avg_test <- mean((test_churn$bill_avg - pred_rf_bill_avg_test)^2, na.rm=TRUE)
cat("Test MSE (Random Forest for bill_avg):", mse_rf_bill_avg_test, "\n")
# Marking guide value: 96.54472 (significantly lower than OLS, LASSO, single tree)
\end{lstlisting}

    \subsubsection{1j: Features of customers with high/low \Rcode{bill\_avg}}
        \paragraph{Question}
        \textit{Based on your analysis in 1a)-1i, what are the features of customers with high and, respectively, low average bills?}
        \paragraph{Solution Outline (Textual)}
        \textit{Synthesize findings from all models, focusing on consistent patterns and insights from the best performing model (Random Forest).}
        % Interpretation based on marking guide:
        % "Linear models/LASSO: High bill = no TV/movie, long subscription, short remaining_contract,
        % many service_failures, high download/upload, often over limit, churned.
        % Single Tree: More nuanced. High download_avg without TV_sub -> high bill. Low download_avg,
        % no limit exceeded, low upload, new sub -> low bill.
        % Random Forest (best model): download_avg and upload_avg are very important.
        % Low bill: Generally associated with low download/upload averages, not exceeding download limits,
        % and potentially being a newer subscriber with fewer service failures.
        % High bill: Associated with high download/upload averages, more service failures, longer
        % subscription age, and potentially being a churned customer or having TV/movie subscriptions
        % (though RF might show more complex interactions here than OLS implied)."
        % You need to refine this based on YOUR specific model outputs and varImpPlot.

\vspace{1em}
\textbf{\Large Task 2: Predict \Rcode{churn}}
\vspace{0.5em}

    \subsubsection{2a: Data formatting, descriptive statistics for \Rcode{churn}}
        \paragraph{Question}
        \textit{Make sure that all variables are on the right format for your analysis. Use tables and graphs and common sense to remove variables that you think will not be helpful. Motivate your choices thoroughly. Use descriptive statistics to find promising predictors for \Rcode{churn}.}
        \paragraph{Solution Outline (Base R / Tidyverse)}
\begin{lstlisting}
# Assuming xdata_churn from 1a (id removed, specified columns are factors)
# and train_churn (the training split)
# Data format should be okay from 1a if factors were made correctly.
# Remove unhelpful variables: 'bill_avg' might be considered a consequence of churn or post-churn,
# or a predictor. Marking guide implies it's kept. No other obvious removals for churn prediction.

# Descriptive stats on train_churn:
# Base R:
# cat_preds_churn <- names(train_churn)[sapply(train_churn, is.factor) & names(train_churn)!="churn"]
# num_preds_churn <- names(train_churn)[sapply(train_churn, is.numeric)]
# for (pred in c(cat_preds_churn, num_preds_churn)) {
#   if (is.factor(train_churn[[pred]])) {
#     print(paste("Table for churn vs", pred))
#     print(prop.table(table(Churn=train_churn$churn, Predictor=train_churn[[pred]]), margin=2))
#   } else {
#     #boxplot(train_churn[[pred]] ~ train_churn$churn, main=paste(pred, "by Churn"))
#   }
# }
# Tidyverse (as in marking guide for means):
# library(dplyr)
# train_churn %>% 
#   mutate(churn_numeric = as.numeric(churn)-1) %>% # if churn is factor "0","1" or "No","Yes"
#   group_by(churn) %>% 
#   summarise(across(where(is.numeric) & !is.factor, list(mean=mean)),
#             across(where(is.factor) & !matches("churn"), ~mean(as.numeric(.)-1, na.rm=TRUE))) %>%
#   t() %>% print()
# plot(remaining_contract ~ churn, data=train_churn, main="Remaining Contract by Churn")

# Promising predictors (from marking scheme):
# is_movie_package_subscriber (lower for churners), remaining_contract (much lower for churners),
# download_avg (lower for churners), upload_avg (lower for churners),
# download_over_limit (higher for churners).
\end{lstlisting}

    \subsubsection{2b: Bootstrap CI for P(churn=1)}
        \paragraph{Question}
        \textit{Let Y be a stochastic variable equal to 1 if a customer churn and 0 otherwise and let $p = P(Y = 1)$ be the unconditional probability to churn. The first 50 (original, not from the training or test data) observations of the variable \Rcode{churn} contains observations drawn from the stochastic variable Y. Use the bootstrap to compute a 95\% confidence interval for $p$. Compare the standard approximation $\hat{p} \pm 1.96 \sqrt{\hat{p}(1-\hat{p})/n}$ where $\hat{p}$ is the sample fraction of churners and $n$ is the number of observations.}
        \paragraph{Base R Solution}
\begin{lstlisting}
# Using original xdata_churn (after factor conversion in 1a)
# churn_numeric_orig <- as.numeric(xdata_churn$churn) - 1 # Convert factor to 0/1
# first_50_churn <- churn_numeric_orig[1:50]
# n_obs_50 <- 50

# Bootstrap function for proportion
prop_churn_stat <- function(data, index) {
  return(mean(data[index])) # Mean of 0/1 variable is proportion of 1s
}
set.seed(86554354) # Consistent seed
boot_churn_prop <- boot(data=first_50_churn, statistic=prop_churn_stat, R=1000)
# print(boot_churn_prop)

# Bootstrap Percentile CI
ci_perc_churn <- boot.ci(boot_churn_prop, type="perc")
print("Bootstrap Percentile CI for P(churn=1):")
print(ci_perc_churn) 
# E.g., (0.68, 0.90) from marking guide.

# Standard Normal Approximation CI
p_hat_50 <- mean(first_50_churn)
se_p_hat_50 <- sqrt(p_hat_50  (1 - p_hat_50) / n_obs_50)
ci_norm_approx_churn <- c(p_hat_50 - 1.96  se_p_hat_50, p_hat_50 + 1.96  se_p_hat_50)
print("Normal Approximation CI for P(churn=1):")
print(ci_norm_approx_churn)
# E.g., (0.689, 0.911) from marking guide.
# Comparison: "The results are very similar, indicating the normal approximation works well for this sample size and proportion."
\end{lstlisting}

    \subsubsection{2c: Logistic regression for \Rcode{churn}}
        \paragraph{Question}
        \textit{Fit a logistic regression with all variables to \Rcode{churn}. Interpret the coefficients in front of \Rcode{is\_tv\_subscriber} and \Rcode{is\_movie\_package\_subscriber}. For all coefficients, is the sign as you expected?}
        \paragraph{Base R Solution}
\begin{lstlisting}
# Assuming train_churn from 1a (with factors, including original 'churn' factor)
logit_churn_allvars <- glm(churn ~ ., data=train_churn, family=binomial)
summary_logit_churn <- summary(logit_churn_allvars)
print(summary_logit_churn)

# Interpretation of coefficients:
# coefs_churn <- coef(logit_churn_allvars)
# odds_ratios_churn <- exp(coefs_churn)
# For is_tv_subscriber1 (assuming '1' means Yes):
# OR_tv = exp(coefs_churn["is_tv_subscriber1"]) e.g., exp(-1.75) = 0.17
# "Holding other variables constant, customers with a TV subscription have 0.17 times
# the odds of churning compared to those without a TV subscription (i.e., 83% lower odds)."
# For is_movie_package_subscriber1:
# OR_movie = exp(coefs_churn["is_movie_package_subscriber1"]) e.g., exp(-0.06) = 0.94
# "Holding other variables constant, customers with a movie package have 0.94 times
# the odds of churning compared to those without (i.e., 6% lower odds)."
# Expected signs:
# Negative for TV/Movie sub (loyalty), subscription_age (more invested), remaining_contract (locked in).
# Positive for service_failure_count, download_over_limit.
# Ambiguous for bill_avg, download_avg, upload_avg (could be good service or high cost).
# Check significance (p-values). Marking guide: bill_avg, upload_avg not significant.
\end{lstlisting}

    \subsubsection{2d: Evaluate logistic regression, consider removing variables}
        \paragraph{Question}
        \textit{Use the logistic regression from 2c) to predict \Rcode{churn}. Evaluate the predictions in an appropriate way. Are the predictions improved if you remove some variables from the model?}
        \paragraph{Base R Solution}
\begin{lstlisting}
# Assuming logit_churn_allvars from 2c (fitted on train_churn) and test_churn
probs_logit_test_churn <- predict(logit_churn_allvars, newdata=test_churn, type="response")
# Default threshold 0.5 for initial evaluation
pred_class_logit_churn_05 <- ifelse(probs_logit_test_churn > 0.5, 
                                   levels(train_churn$churn)[2], levels(train_churn$churn)[1])
pred_class_logit_churn_05 <- factor(pred_class_logit_churn_05, levels=levels(train_churn$churn))

conf_matrix_logit_05 <- table(Actual=test_churn$churn, Predicted=pred_class_logit_churn_05)
print("Confusion Matrix (Full Model, Threshold 0.5):")
print(conf_matrix_logit_05)
# accuracy_logit_05 <- sum(diag(conf_matrix_logit_05)) / sum(conf_matrix_logit_05)
# print(paste("Accuracy (Full Model):", accuracy_logit_05))
# print("Proportions table (Full Model):")
# print(prop.table(conf_matrix_logit_05, margin=1)) 
# Marking scheme results for full model: (row proportions)
#   FALSE     TRUE
# 0 0.81606725 0.18393275
# 1 0.08262943 0.91737057 (Good at predicting non-churn, good at predicting churn when it occurs)

# Model with non-significant variables (bill_avg, upload_avg) removed:
logit_churn_reduced <- glm(churn ~ . -bill_avg -upload_avg, data=train_churn, family=binomial)
# summary(logit_churn_reduced)
probs_logit_reduced_test <- predict(logit_churn_reduced, newdata=test_churn, type="response")
pred_class_logit_reduced_05 <- ifelse(probs_logit_reduced_test > 0.5,
                                     levels(train_churn$churn)[2], levels(train_churn$churn)[1])
pred_class_logit_reduced_05 <- factor(pred_class_logit_reduced_05, levels=levels(train_churn$churn))
conf_matrix_logit_reduced_05 <- table(Actual=test_churn$churn, Predicted=pred_class_logit_reduced_05)
print("Confusion Matrix (Reduced Model, Threshold 0.5):")
print(conf_matrix_logit_reduced_05)
# print("Proportions table (Reduced Model):")
# print(prop.table(conf_matrix_logit_reduced_05, margin=1))
# Marking scheme: "removal of non-significant variables did not improve the test confusion matrix much."
# (Compare the prop.tables or specific metrics like sensitivity/specificity).
\end{lstlisting}

    \subsubsection{2e: Random forest for \Rcode{churn}}
        \paragraph{Question}
        \textit{Use a random forest to predict \Rcode{churn} and evaluate the predictions. (If computations are too slow, reduce \Rcode{ntree} or use smaller training data.)}
        \paragraph{Base R Solution (using \Rpackage{randomForest})}
\begin{lstlisting}
library(randomForest)
# Assuming train_churn, test_churn from 1a
set.seed(86554354) % \textit{(Consistent seed)} %
rf_churn <- randomForest(churn ~ ., data=train_churn, 
                         ntree=50, # Reduced for speed as per hint/marking scheme
                         importance=TRUE, na.action=na.roughfix)

# Predict classes directly
pred_class_rf_churn <- predict(rf_churn, newdata=test_churn, na.action=na.pass)
conf_matrix_rf_churn <- table(Actual=test_churn$churn, Predicted=pred_class_rf_churn)
print("Confusion Matrix (Random Forest):")
print(conf_matrix_rf_churn)
# accuracy_rf_churn <- sum(diag(conf_matrix_rf_churn)) / sum(conf_matrix_rf_churn)
# print(paste("Accuracy (Random Forest):", accuracy_rf_churn))
# print("Proportions table (Random Forest):")
# print(prop.table(conf_matrix_rf_churn, margin=1))
# Marking scheme results (row proportions):
#   FALSE     TRUE
# 0 0.94564187 0.05435813
# 1 0.06290371 0.93709629
# "The random forest improved the predictions considerably" (compared to logistic regression).
# Higher true positive for churn (0.937 vs 0.917) and higher true negative (0.945 vs 0.816).
\end{lstlisting}

    \subsubsection{2f: Typical features of customers who churn}
        \paragraph{Question}
        \textit{Based on your analysis in 2a)-2e, what are the typical features of customers who churn?}
        \paragraph{Solution Outline (Textual)}
        \textit{Synthesize findings from descriptive stats, logistic regression (significant coefficients, signs), and random forest (variable importance if examined, though not explicitly asked for plot here).}
        % Marking guide: "According to the logistic regression, a churning customer has
        % no movie package, has been a customer for a short time only, has a short
        % remaining contract, has many service failures, low download and upload averages
        % and is often over the download limit."
        % Your answer should confirm these from your logistic regression output (signs of sig. coefs)
        % and might add insights from RF if it highlighted different aspects or confirmed these strongly.

\subsection{Exam Spring 2024}
    \textit{For variable definitions, see the table at the end of this exam section.}

\textbf{\Large Task 1: Methodological Topics}
\vspace{0.5em}

    \subsubsection{1a: Explain R-function}
        \paragraph{Question}
        \textit{Explain what the following R-function is doing.}
\begin{lstlisting}[caption={R-function f for 1a (Exam 2024)}]
f <- function(x0, x, y, K=3) {
  d <- abs(x - x0)       
  o <- order(d)[1:K]     
  xl <- x[o]             
  yl <- y[o]             
  xydata <- data.frame(xl=xl, yl=yl) 
  reg <- lm(yl ~ xl, data=xydata)    
  ypred <- predict(reg, newdata=data.frame(xl=x0, yl=1)) % \textit{(yl=1 is placeholder)} %
  return(ypred)
}
\end{lstlisting}
        \paragraph{Explanation}
        \textit{
        This R function \Rcode{f(x0, x, y, K=3)} implements a local regression prediction method.
        1. It takes a target predictor value \Rcode{x0}, a vector of training predictor values \Rcode{x}, corresponding training response values \Rcode{y}, and the number of neighbors \Rcode{K} (default 3).
        2. It calculates the absolute distances (\Rcode{d}) between \Rcode{x0} and all values in \Rcode{x}.
        3. It finds the indices (\Rcode{o}) of the \Rcode{K} values in \Rcode{x} that are closest to \Rcode{x0}.
        4. It selects these \Rcode{K} closest predictor values (\Rcode{xl}) and their corresponding response values (\Rcode{yl}).
        5. It creates a data frame \Rcode{xydata} from these \Rcode{K} neighbors.
        6. It fits a simple linear regression model (\Rcode{reg}) of \Rcode{yl} on \Rcode{xl} using only these \Rcode{K} neighbors.
        7. Finally, it uses this locally fitted linear model to predict the response (\Rcode{ypred}) for the original target point \Rcode{x0}.
        The method resembles K-Nearest Neighbors (KNN) regression but instead of averaging the responses of the neighbors, it fits a linear model to them for prediction. The \Rcode{yl=1} in \Rfunction{predict} is a placeholder and not used when predicting for a new \Rcode{xl}.
        }

    \subsubsection{1b: Optimal K for function f using LOOCV}
        \paragraph{Question}
        \textit{Consider the following small dataset: x = [1, 2, 3, 4, 5, 6, 7, 8, 9, 10], y = [5.26, 9.13, 11.17, 15.64, 25.32, 25.55, 41.39, 48.17, 58.65, 68.24]. Use leave-one-out cross-validation (LOOCV) to determine the optimal K in the function \Rcode{f} for this dataset.}
        \paragraph{Base R Solution (basert på løsningsforslag)}
\begin{lstlisting}
x_1b <- c(1, 2, 3, 4, 5, 6, 7, 8, 9, 10)
y_1b <- c(5.26, 9.13, 11.17, 15.64, 25.32, 25.55, 41.39, 48.17, 58.65, 68.24)
n_1b <- length(x_1b)

# Function f from 1a (ensure it's defined)
f <- function(x0, x, y, K=3) {
  d <- abs(x - x0); o <- order(d)[1:K]; xl <- x[o]; yl <- y[o]
  xydata <- data.frame(xl=xl, yl=yl); reg <- lm(yl ~ xl, data=xydata)
  return(predict(reg, newdata=data.frame(xl=x0))) # Removed yl=1 placeholder
}

# LOOCV function (as per solution structure)
loo_cv_func <- function(K_val, x_full, y_full) {
  n_loo <- length(x_full)
  squared_errors <- numeric(n_loo)
  for (i in 1:n_loo) {
    x_train <- x_full[-i]
    y_train <- y_full[-i]
    x_test_single  <- x_full[i]
    y_test_single  <- y_full[i]
    
    # Ensure K is not greater than number of available training points for local lm
    K_to_use <- min(K_val, length(x_train))
    if (K_to_use < 2) K_to_use <- 2 # lm needs at least 2 points for y~x
    
    pred_loo <- f(x_test_single, x_train, y_train, K=K_to_use)
    squared_errors[i] <- (y_test_single - pred_loo)^2
  }
  return(mean(squared_errors))
}

# Iterate for different K values (solution implies K from 2 to 10, but K=n-1 for LOOCV training set)
# K can range from 2 (min for lm) up to n-1 (using all training points for local lm)
possible_K_vals <- 2:(n_1b-1) 
loocv_mse_values <- sapply(possible_K_vals, loo_cv_func, x_full=x_1b, y_full=y_1b)

# Plot results and find optimal K
plot(possible_K_vals, loocv_mse_values, type="b", xlab="K (Number of Neighbors)", 
     ylab="LOOCV MSE", main="LOOCV to find Optimal K")
optimal_K_1b <- possible_K_vals[which.min(loocv_mse_values)]
points(optimal_K_1b, min(loocv_mse_values), col="red", pch=19, cex=1.5)
cat("Optimal K found by LOOCV:", optimal_K_1b, "\n") 
# Exam solution PDF indicates K=5 is optimal.
\end{lstlisting}

    \subsubsection{1c: Plot predictions with optimal K}
        \paragraph{Question}
        \textit{Plot $y$ against $x$ and add a line with the predictions based on the optimal $K$.}
        \paragraph{Base R Solution}
\begin{lstlisting}
# Assuming x_1b, y_1b from 1b, and optimal_K_1b = 5
optimal_K_plot <- 5 
predictions_1c <- numeric(n_1b)
# To generate a smooth line, we'd typically predict on a grid of x0 values.
# However, to match the solution's likely intent (predicting on training points for the line):
for (i in 1:n_1b) {
  # Here, for plotting the fit on training data, f uses the full dataset x_1b, y_1b as its "training"
  # to find neighbors for x_1b[i]. This is for visualizing the fitted function.
  predictions_1c[i] <- f(x_1b[i], x_1b, y_1b, K=optimal_K_plot)
}

plot(x_1b, y_1b, xlab="x", ylab="y", main=paste("Data and Fitted Line with K =", optimal_K_plot))
order_x_1b <- order(x_1b)
lines(x_1b[order_x_1b], predictions_1c[order_x_1b], col="red", lwd=2)
legend("topleft", legend=paste("Fit with K =", optimal_K_plot), col="red", lty=1, lwd=2)
\end{lstlisting}

    \subsubsection{1d: Modify function f for multiple predictors}
        \paragraph{Question}
        \textit{In the function \Rcode{f}, there is only one predictor. One way to allow for more than one predictor is to compute \Rcode{d} in the f-function in (a) differently. Explain how such a modification can be done; exemplify with a case with two predictors.}
        \paragraph{Explanation and Conceptual R Modification}
        \textit{
        To adapt function \Rcode{f} for multiple predictors ($p > 1$), the distance calculation \Rcode{d} needs to be changed from \Rcode{abs(x - x0)} to a multivariate distance. The most common is Euclidean distance.
        
        \textbf{Example with two predictors ($X_1, X_2$)}:
        Let input \Rcode{x0} be a vector \Rcode{c(x01, x02)}, and \Rcode{x} be a matrix where rows are observations and columns are $X_1, X_2$.
        The modified \Rcode{d} calculation within \Rcode{f} would be:
        }
\begin{lstlisting}[caption={Conceptual modification for distance in f (2 predictors)}]
# Inside function f_multi(x0_vec, x_matrix, y_vec, K)
# x0_vec is c(x0_pred1, x0_pred2)
# x_matrix has columns for pred1, pred2

# Calculate Euclidean distances
# diff_sq <- sweep(x_matrix, 2, x0_vec, "-")^2 # (x_i1-x01)^2, (x_i2-x02)^2 for each row
# d_multi <- sqrt(rowSums(diff_sq))

# The rest of the logic for finding K nearest neighbors (o, xl, yl) remains similar,
# but xl will be a matrix of K rows and p columns.
# The lm call becomes: reg <- lm(yl ~ ., data=as.data.frame(xl))
# And newdata for predict needs to be a data frame with column names matching xl.
# newdata_pred <- as.data.frame(matrix(x0_vec, nrow=1, 
#                                      dimnames=list(NULL, colnames(x_matrix))))
\end{lstlisting}
        \textit{
        Crucially, if predictors are on different scales, they should be standardized before calculating Euclidean distances to prevent variables with larger scales from dominating the distance metric. The linear model \Rcode{lm(yl ~ .)} would then be a multiple linear regression on the $K$ selected neighbors' $p$ (standardized, if applicable) predictor values.
        }

    \subsubsection{1e: Implement backfitting for this method}
        \paragraph{Question}
        \textit{Another way to allow for several predictors is to use backfitting. Explain how you would implement backfitting for this method (you do not need to do it).}
        \paragraph{Explanation}
        \textit{
        Backfitting allows fitting an additive model $Y = \beta_0 + f_1(X_1) + f_2(X_2) + \dots + f_p(X_p) + \epsilon$, where each $f_j$ is estimated using a univariate smoother, such as our function \Rcode{f} (from 1a, adapted for single predictor use).
        
        \textbf{Backfitting Algorithm with function \Rcode{f} (example for $p=2$ predictors $X_1, X_2$)}:
        \begin{enumerate}
            \item \textbf{Initialize}: $\hat{\beta}_0 = \text{mean}(Y)$. Initialize $\hat{f}_1(x_{i1}) = 0$ and $\hat{f}_2(x_{i2}) = 0$ for all observations $i$.
            \item \textbf{Iterate} until convergence (i.e., $\hat{f}_j$ functions stabilize):
                \begin{enumerate}
                    \item \textbf{Update $\hat{f}_1(X_1)$}:
                        Compute partial residuals: $r_{i1} = y_i - \hat{\beta}_0 - \hat{f}_2(x_{i2})$.
                        For each unique value $x_{u1}$ of $X_1$ (or for each $x_{i1}$):
                        $\hat{f}_1(x_{u1}) \leftarrow \text{Call } f(x_{u1}, X_1, r_1, K)$. % (Here x=X1, y=r1) %
                        (Typically, one would re-center: $\hat{f}_1 \leftarrow \hat{f}_1 - \text{mean}(\hat{f}_1)$).
                    \item \textbf{Update $\hat{f}_2(X_2)$}:
                        Compute partial residuals: $r_{i2} = y_i - \hat{\beta}_0 - \hat{f}_1(x_{i1})$ (using updated $\hat{f}_1$).
                        For each unique value $x_{u2}$ of $X_2$:
                        $\hat{f}_2(x_{u2}) \leftarrow \text{Call } f(x_{u2}, X_2, r_2, K)$. % (Here x=X2, y=r2) %
                        (Re-center: $\hat{f}_2 \leftarrow \hat{f}_2 - \text{mean}(\hat{f}_2)$).
                \end{enumerate}
        \end{enumerate}
        The final prediction for a new observation $(\mathbf{x}_{01}, \mathbf{x}_{02})$ is $\hat{Y}_0 = \hat{\beta}_0 + \hat{f}_1(\mathbf{x}_{01}) + \hat{f}_2(\mathbf{x}_{02})$. The function \Rcode{f} from 1a acts as the univariate smoother for each predictor on its partial residuals. The optimal $K$ for each $f_j$ can be chosen using CV, possibly within each backfitting iteration or as a global parameter.
        }

\vspace{1em}
\textbf{\Large Task 2: Analysis of \Robject{airline.csv} Data - Customer Satisfaction}
\vspace{0.5em}
    \textit{Target variable: \Rcode{satisfaction}. Often converted to binary for classification.}

    \subsubsection{2a: Recode categorical variables and split data}
        \paragraph{Question}
        \textit{If necessary, recode categorical variables as factors. Motivate your choices in words. (Implicit: Split data for training/testing as per general instructions and solution PDF structure for this exam).}
        \paragraph{Base R Solution (based on solution PDF)}
\begin{lstlisting}
airline <- read.csv("airline.csv", stringsAsFactors = FALSE) % \textit{Read data} %
# Convert 'satisfaction' to factor (assuming 'satisfied' vs 'dissatisfied' for classification)
# This step depends on how 'satisfaction' is defined in the original CSV.
# If it's text "satisfied"/"dissatisfied", it's already good for factor.
# If numeric, might need binning. Let's assume it becomes a factor.
# airline$satisfaction <- as.factor(airline$satisfaction)

# Recode variables with <= 7 unique values as factors (as per solution for this exam)
for (i in 1:ncol(airline)) {
  if (length(unique(airline[,i])) <= 7 && !is.factor(airline[,i])) {
    airline[,i] <- as.factor(airline[,i])
  }
}
# Example: Gender, Customer_Type, Type_of_Travel, Class are likely candidates.
# Satisfaction scores (0-5) might also be converted if not already factors.
# Motivation: Ensure categorical predictors are treated as such by modeling functions.

# Split data (as per solution PDF structure for this exam)
set.seed(123) # From solution PDF for this task
n_airline <- nrow(airline)
train_idx_airline <- sample(1:n_airline, floor(n_airline/2))
training_airline <- airline[train_idx_airline, ]
test_airline <- airline[-train_idx_airline, ]
# sapply(training_airline, class)
\end{lstlisting}
        \paragraph{Tidyverse Solution}
\begin{lstlisting}
library(dplyr)
airline_tidy <- read.csv("airline.csv", stringsAsFactors = FALSE) %>%
  mutate(across(where(~length(unique(.)) <= 7 & !is.factor(.)), as.factor))
  # May need specific mutate for 'satisfaction' if it's not binary textual
  # e.g. mutate(satisfaction = factor(ifelse(satisfaction_score >= 4, "satisfied", "dissatisfied")))

set.seed(123)
train_idx_airline_tidy <- sample(1:nrow(airline_tidy), floor(nrow(airline_tidy)/2))
training_airline_tidy <- airline_tidy[train_idx_airline_tidy, ]
test_airline_tidy <- airline_tidy[-train_idx_airline_tidy, ]
# glimpse(training_airline_tidy)
\end{lstlisting}

    \subsubsection{2b: Motivation for data usage (Inference: "Why dissatisfied?")}
        \paragraph{Question}
        \textit{In tasks (a)-(g) the methods and models are used to answer the question "Why are some customers dissatisfied?". With this in mind, motivate your choice to evaluate the models on all/training/test data.}
        \paragraph{Solution/Motivation (based on exam solution PDF)}
        \textit{
        For tasks (a)-(g), which focus on understanding \textit{why} customers are dissatisfied (an inferential goal rather than pure out-of-sample prediction), one could argue for using the entire dataset to maximize statistical power for inference.
        However, the provided solution chooses to \textbf{split the data into training and test sets from the start, and perform all descriptive analysis and initial model fitting (for inference) on the training data only}.
        The motivations for this approach are:
        \begin{enumerate}
            \item \textbf{Comparability}: To allow direct comparison of model performance (e.g., R-squared, variable importance) with models developed later for the explicit prediction task (h-l), which \textit{must} be evaluated on test data.
            \item \textbf{Preventing Data Leakage}: Using only training data for initial exploration and model specification for the inferential part acts as a safeguard against inadvertently using information from the test set to guide these early modeling choices. This ensures that when the test set is used for the later prediction task, it is truly "unseen".
        \end{enumerate}

        }

    \subsubsection{2c: Descriptive statistics for}
        \paragraph{Question}
        \textit{Use descriptive statistics to find variables associated with \Rcode{satisfaction}. First, explain which type of descriptive statistics (types of tables and figures) that you are using. Give examples of R code but do not show all code and output if you are doing the same thing many times. Summarize your results as text, mentioning the relevant numbers.}
        \paragraph{Base R Solution (Conceptual based on exam solution)}
\begin{lstlisting}
# Assuming training_airline from 2a. Assume 'satisfaction' is a binary factor ("satisfied", "dissatisfied").
# Overall satisfaction rate:
# print(prop.table(table(training_airline$satisfaction))) # e.g., 55% satisfied

# Function for conditional proportions (as per exam solution)
proptab_satisfaction <- function(data, predictor_col_name) {
  tbl <- table(data[[predictor_col_name]], data$satisfaction)
  return(round(100  prop.table(tbl, margin = 1), 1)) # Proportion satisfied/dissatisfied per predictor level
}
# Example: proptab_satisfaction(training_airline, "Gender")
#            satisfaction
# Gender      dissatisfied satisfied
#   Female            35        65
#   Male              56        44
# (Values from exam solution PDF for illustration)

# For numeric variables: Boxplots and grouped summaries
# boxplot(Age ~ satisfaction, data=training_airline, ylab="Age")
# by(training_airline$Flight_Distance, training_airline$satisfaction, summary)

# Summary of results (textual, based on exam solution PDF):
# - Overall 55% satisfied.
# - Gender: Females (65% satisfied) more satisfied than Males (44%).
# - Customer_Type: Loyal customers more satisfied.
# - Type_of_Travel, Class: Business travelers/class more satisfied.
# - Specific satisfaction Qs (Inflight_entertainment, Cleanliness, etc.): Positively associated with overall satisfaction.
# - Age: Satisfied customers tend to be older on average.
# - Flight_Distance, Delays: No strong or obvious association with overall satisfaction from boxplots/summaries.
\end{lstlisting}
        \paragraph{Tidyverse Solution (Conceptual)}
\begin{lstlisting}
# library(dplyr); library(ggplot2)
# Assuming training_airline_tidy from 2a.
# training_airline_tidy %>% count(satisfaction) %>% mutate(prop = n/sum(n))

# Conditional proportions
# training_airline_tidy %>%
#   count(Gender, satisfaction) %>%
#   group_by(Gender) %>%
#   mutate(prop = round(100  n / sum(n), 1)) %>% print()

# Boxplots
# ggplot(training_airline_tidy, aes(x=satisfaction, y=Age, fill=satisfaction)) + geom_boxplot()
# Grouped summaries
# training_airline_tidy %>% group_by(satisfaction) %>% 
#   summarise(mean_age = mean(Age, na.rm=TRUE), median_dist = median(Flight_Distance, na.rm=TRUE))
\end{lstlisting}

\subsection{2d: Logistic Regression (Why dissatisfied?)}

\begin{lstlisting}[caption={2d: Logistic Regression - Inferential (airline)}]
logreg_infer <- glm(satisfaction ~ Gender + Customer_Type + Age + Type_of_Travel +
    Class + Flight_Distance + Food_and_drink + Inflight_entertainment +
    Online_support + Baggage_handling + Checkin_service + Arrival_Delay_in_Minutes +
    Inflight_wifi_service + Leg_room_service + On_board_service + Gate_location + Seat_comfort,
    data=training_airline, family=binomial)

probs_logreg_infer_test <- predict(logreg_infer, newdata=test_airline, type="response")
preds <- ifelse(probs_logreg_infer_test > 0.5, "satisfied", "dissatisfied")
conf_matrix <- table(Actual = test_airline$satisfaction, Predicted = preds)
print(conf_matrix)
\end{lstlisting}

\subsection{2e: Classification Tree (Why dissatisfied?)}

\begin{lstlisting}[caption={2e: Classification Tree - Inferential}]
library(tree)
tree_model <- tree(satisfaction ~ Gender + Customer_Type + Age + Type_of_Travel +
    Class + Flight_Distance + Food_and_drink + Inflight_entertainment +
    Online_support + Baggage_handling + Checkin_service + Arrival_Delay_in_Minutes +
    Inflight_wifi_service + Leg_room_service + On_board_service + Gate_location + Seat_comfort,
    data=training_airline)
plot(tree_model)
text(tree_model, pretty=0)
\end{lstlisting}

\subsection{2f: Random Forest (Why dissatisfied?)}

\begin{lstlisting}[caption={2f: Random Forest - Inferential}]
library(randomForest)
rf_model <- randomForest(satisfaction ~ Gender + Customer_Type + Age + Type_of_Travel +
    Class + Flight_Distance + Food_and_drink + Inflight_entertainment +
    Online_support + Baggage_handling + Checkin_service + Arrival_Delay_in_Minutes +
    Inflight_wifi_service + Leg_room_service + On_board_service + Gate_location + Seat_comfort,
    data=training_airline, importance=TRUE)
varImpPlot(rf_model)
\end{lstlisting}

\subsection{2g: Why are some customers dissatisfied?}

\textit{Customers who are dissatisfied often report low satisfaction with inflight entertainment, seat comfort, online support, and baggage handling. These variables were consistently important in the models. Loyal customers and business class passengers are generally more satisfied. Demographic variables like age and gender also play a role.}

\subsection{2h: Assumptions for Prediction Task (Pre-flight variables)}

\textit{Variables known before flight: Gender, Age, Customer Type, Type of Travel, Class, and Flight Distance. Satisfaction ratings and delay variables are post-flight and not available for prediction before departure.}

\subsection{2i: Motivation for Train/Test Split in Prediction}

\textit{Since the goal is to predict dissatisfaction for new customers, it is crucial to evaluate the model on unseen test data to assess generalization performance.}

\subsection{2j: Logistic Regression (Prediction)}

\begin{lstlisting}[caption={2j: Logistic Regression - Prediction}]
logreg_pred <- glm(satisfaction ~ Gender + Customer_Type + Age + Type_of_Travel + Class + Flight_Distance,
    data=training_airline, family=binomial)
probs_pred <- predict(logreg_pred, newdata=test_airline, type="response")
preds <- ifelse(probs_pred > 0.5, "satisfied", "dissatisfied")
conf_matrix <- table(Actual = test_airline$satisfaction, Predicted = preds)
print(conf_matrix)
\end{lstlisting}

\subsection{2k: Random Forest (Prediction)}

\begin{lstlisting}[caption={2k: Random Forest - Prediction}]
rf_pred <- randomForest(satisfaction ~ Gender + Customer_Type + Age + Type_of_Travel + Class + Flight_Distance,
    data=training_airline, importance=TRUE)
preds_rf <- predict(rf_pred, newdata=test_airline)
conf_matrix <- table(Actual = test_airline$satisfaction, Predicted = preds_rf)
print(conf_matrix)
\end{lstlisting}

\subsection{2l: Future Data Collection}

\textit{To improve predictions, collect data on pre-flight interactions: booking channel, time between booking and flight, seat selection, and support tickets. These can act as proxies for future dissatisfaction and help model potential outcomes more accurately.}



\section{Tutorials}
    \textit{This section summarizes key tasks and R code from the provided exercise sets, assuming they correspond to the course tutorials.}

    \subsection{Tutorial 1 }
        \subsubsection{Task 1: Linear Regression and KNN (One Predictor)}
            \paragraph{Context} Given a small dataset (x, y pairs).
            \paragraph{1a: Prediction with Linear Regression (Manual/Calculator)}
                \textit{Question: What is the prediction for x = 3 if the linear regression model is used?}
                \textit{Solution Approach:} Manually calculate $\hat{\alpha}_0, \hat{\alpha}_1$ using formulas for simple linear regression, then predict $\hat{y} = \hat{\alpha}_0 + \hat{\alpha}_1 \cdot 3$.
\begin{lstlisting}[caption={Manual OLS Calculation (Conceptual - Ex Set 1, T1a)}]
# x_t1 <- 1:5
# y_t1 <- c(1.88, 4.54, 10.12, 9.14, 11.26)
# n_t1 <- length(x_t1)
# mean_x_t1 <- mean(x_t1)
# mean_y_t1 <- mean(y_t1)
# 
# beta1_hat_t1 <- sum((x_t1 - mean_x_t1)  (y_t1 - mean_y_t1)) / sum((x_t1 - mean_x_t1)^2)
# beta0_hat_t1 <- mean_y_t1 - beta1_hat_t1  mean_x_t1
# 
# prediction_x3_t1 <- beta0_hat_t1 + beta1_hat_t1  3
# print(prediction_x3_t1) % \textit{Solution gives 7.388} %
\end{lstlisting}
            \paragraph{1b: Fit Linear Regression with \Rfunction{lm} and Predict}
                \textit{Question: Use \Rfunction{lm} to fit the same model and \Rfunction{predict} for x=3.}
\begin{lstlisting}[caption={Using lm and predict (Ex Set 1, T1b)}]
# x_t1 <- 1:5
# y_t1 <- c(1.88, 4.54, 10.12, 9.14, 11.26)
# model_t1b <- lm(y_t1 ~ x_t1)
# predict_x3_t1b <- predict(model_t1b, newdata = data.frame(x_t1 = 3))
# print(predict_x3_t1b) % \textit{Solution gives 7.388} %
\end{lstlisting}
            \paragraph{1c: KNN Prediction}
                \textit{Question: Prediction for x=3 if KNN with K=1, K=3, K=5 is used.}
                \textit{Solution Approach (Manual/Calculator):}
                For K=1: Find y-value corresponding to x closest to 3.
                For K=3: Find 3 y-values for x's closest to 3, then average them.
                For K=5: Average all 5 y-values.
\begin{lstlisting}[caption={KNN Predictions (Conceptual - Ex Set 1, T1c)}]
# y_t1 <- c(1.88, 4.54, 10.12, 9.14, 11.26)
# K=1 (x=3 is closest): 10.12
# K=3 (x=2,3,4 are closest): mean(c(4.54, 10.12, 9.14)) % \textit{Result: 7.933333} %
# K=5 (all points): mean(y_t1) % \textit{Result: 7.388} %
\end{lstlisting}
            \paragraph{1d: Understanding KNN R-function}
                \textit{Question: Explain each row of the provided KNN R-function.}
\begin{lstlisting}[caption={KNN R-function (Ex Set 1, T1d)}]
knn_func_t1d <- function(x0, x, y, K=20) { % \textit{Note: Exercise sheet has K=20, example uses K=3} %
  d <- abs(x-x0)                 % \textit{# Calculate absolute distances from x0 to all x} %
  o <- order(d)[1:K]             % \textit{# Get indices of the K smallest distances} %
  ypred <- mean(y[o])            % \textit{# Average y-values of the K nearest neighbors} %
  return(ypred)                  % \textit{# Return the prediction} %
}
# Experiment: K_exp=3; x0_exp=3; x_exp=1:5; y_exp=c(1.88,4.54,10.12,9.14,11.26)
# d_exp <- abs(x_exp - x0_exp) % Result: 2 1 0 1 2 %
# o_exp <- order(d_exp)[1:K_exp] % Result: 3 2 4 (indices for x=3, x=2, x=4) %
# ypred_exp <- mean(y_exp[o_exp]) % Result: mean(y[c(3,2,4)]) = 7.933333 %
\end{lstlisting}
            \paragraph{1e: Choosing Method Based on Plot}
                \textit{Question: Which method (LR or KNN K=1,2,3) for a given scatter plot? Why?}
                \textit{Solution Approach:} If plot shows strong linear trend, Linear Regression is preferred for efficiency and interpretability. If non-linear, KNN might be better, with K chosen to balance bias/variance.

        \subsubsection{Task 2: Data Splitting and Basic Linear Model}
            \paragraph{2a: Load, Summarize, Help for \Robject{College}}
\begin{lstlisting}
library(ISLR)
# summary(College)
# help(College)
\end{lstlisting}
            \paragraph{2b: 50/50 Training/Test Split}
\begin{lstlisting}
set.seed(123)
n_college <- nrow(College)
train_indicator_college <- sample(1:n_college, size=floor(n_college/2))
train_college <- College[train_indicator_college,]
test_college <- College[-train_indicator_college,]
\end{lstlisting}
            \paragraph{2c: Fit Linear Model for \Rcode{Apps ~ Private + Accept}}
\begin{lstlisting}
lm_apps_t2c <- lm(Apps ~ Private + Accept, data=train_college)
# summary(lm_apps_t2c)
\end{lstlisting}
            \paragraph{2d: Compute Training MSE}
\begin{lstlisting}
pred_train_t2d <- predict(lm_apps_t2c, newdata=train_college)
mse_train_t2d <- mean((train_college$Apps - pred_train_t2d)^2)
# print(mse_train_t2d) % \textit{Solution shows 1437492} %
\end{lstlisting}
            \paragraph{2e: Model with \Rcode{Accept} only, compute Training MSE}
\begin{lstlisting}
lm_apps_t2e <- lm(Apps ~ Accept, data=train_college)
pred_train_t2e <- predict(lm_apps_t2e, newdata=train_college)
mse_train_t2e <- mean((train_college$Apps - pred_train_t2e)^2)
# print(mse_train_t2e) % \textit{Solution shows 1440012, larger than 2d, as expected} %
\end{lstlisting}
            \paragraph{2f: Compute Test MSE for models from 2c and 2e}
\begin{lstlisting}
pred_test_t2c <- predict(lm_apps_t2c, newdata=test_college)
mse_test_t2c <- mean((test_college$Apps - pred_test_t2c)^2)
# print(paste("Test MSE (Private + Accept):", mse_test_t2c)) % \textit{Sol: 1849668} %

pred_test_t2e <- predict(lm_apps_t2e, newdata=test_college)
mse_test_t2e <- mean((test_college$Apps - pred_test_t2e)^2)
# print(paste("Test MSE (Accept only):", mse_test_t2e)) % \textit{Sol: 1854227} %
# Model with Private + Accept has slightly better test MSE here.
\end{lstlisting}
            \paragraph{2h: KNN for \Rcode{Apps ~ Accept}, compare Test MSE}
\begin{lstlisting}
# Using knn_func_t1d from earlier, assuming K=3 as per solution's use of it
# Need to apply it for each test observation
# x0_knn_t2h <- test_college$Accept
# x_knn_t2h <- train_college$Accept
# y_knn_t2h <- train_college$Apps
# 
# pred_knn_t2h <- sapply(x0_knn_t2h, knn_func_t1d, x=x_knn_t2h, y=y_knn_t2h, K=3)
# mse_knn_t2h <- mean((test_college$Apps - pred_knn_t2h)^2)
# print(paste("Test MSE (KNN K=3 for Apps ~ Accept):", mse_knn_t2h)) % \textit{Sol: 4473360} %
# KNN performs much worse than linear regression here.
\end{lstlisting}

        \subsubsection{Task 3: Exercise 3.10a-f from ISLR Book}
            \textit{This task involves fitting and interpreting a multiple linear regression model for \Rcode{Sales ~ Price + Urban + US} on the \Rpackage{ISLR}::\Robject{Carseats} dataset. Steps include fitting, interpreting coefficients, identifying significant predictors, and comparing models.}
\begin{lstlisting}
# library(ISLR)
# data(Carseats)
# (a) Fit model
# model_carseats_310a <- lm(Sales ~ Price + Urban + US, data=Carseats)
# (b) Interpret coefficients
# summary(model_carseats_310a)
# (c) Write equation
# (d) Identify significant predictors (Price, US based on p-values)
# (e) Fit model with significant predictors only
# model_carseats_310e <- lm(Sales ~ Price + US, data=Carseats)
# (f) Compare R-squared and Adjusted R-squared (very similar for these two models).
# (g) Optional: Split, train, test.
# set.seed(12345)
# n_cs <- nrow(Carseats)
# train_idx_cs <- sample(1:n_cs, n_cs/2)
# train_cs_310 <- Carseats[train_idx_cs,]
# test_cs_310 <- Carseats[-train_idx_cs,]
# lm3_cs <- lm(Sales ~ Price + Urban + US, data=train_cs_310)
# pred3_cs <- predict(lm3_cs, newdata=test_cs_310)
# mse3_cs <- mean((test_cs_310$Sales - pred3_cs)^2) % Sol: 6.662185 %
# lm4_cs <- lm(Sales ~ Price + US, data=train_cs_310)
# pred4_cs <- predict(lm4_cs, newdata=test_cs_310)
# mse4_cs <- mean((test_cs_310$Sales - pred4_cs)^2) % Sol: 6.580072 %
# Reduced model slightly better on this test split.
\end{lstlisting}

    \subsection{Tutorial 2 }
        \subsubsection{Task 1: Logistic Regression and KNN for Binary Classification}
            \paragraph{Context} Small dataset (x=1:7, y binary).
            \paragraph{1a: Fit Logistic Regression, predict for x0=4}
\begin{lstlisting}
# x_t2 <- 1:7; y_t2 <- c(0,1,0,1,1,1,1)
# data_t2 <- data.frame(x=x_t2, y=as.factor(y_t2)) % \textit{Ensure y is factor for glm binomial} %
# m1_t2 <- glm(y ~ x, data=data_t2, family="binomial")
# summary(m1_t2)
# Estimated equation: P(Y=1|X=x) = 1 / (1 + exp(-(coef(m1_t2)[1] + coef(m1_t2)[2]x)))
# For x0=4: predict(m1_t2, newdata=data.frame(x=4), type="response") % Sol: 0.8628, predict Y=1 %
\end{lstlisting}
            \paragraph{1b: Plot probabilities from Logistic Regression}
\begin{lstlisting}
# plot(x_t2, as.numeric(y_t2)-1, ylab="P(Y=1) or Y") # Plot 0/1 Y values
# probs_m1_t2 <- predict(m1_t2, type="response")
# lines(x_t2, probs_m1_t2, col="blue")
\end{lstlisting}
            \paragraph{1c: KNN probability estimation for x0=4 with K=1,3,5}
\begin{lstlisting}
# knn_prob_t2 <- function(x0, x_train, y_train, K_val) {
#   d <- abs(x_train - x0)
#   o <- order(d)[1:K_val]
#   return(mean(as.numeric(y_train[o])-1)) # Mean of 0/1 y-values
# }
# knn_prob_t2(4, x_t2, data_t2$y, K=1) % Sol: 1 (y at x=4 is 1) %
# knn_prob_t2(4, x_t2, data_t2$y, K=3) % Sol: mean(y at x=3,4,5) = mean(c(0,1,1))=0.667 %
# knn_prob_t2(4, x_t2, data_t2$y, K=5) % Sol: mean(y at x=2,3,4,5,6)=mean(c(1,0,1,1,1))=0.8 %
\end{lstlisting}
            \paragraph{1d: Plot KNN probabilities for K=1,3,5 and logistic}
\begin{lstlisting}
# plot(x_t2, as.numeric(data_t2$y)-1, ylab="P(Y=1) or Y", xlab="x")
# for (K_val_plot in c(1,3,5)) {
#   probs_knn_plot <- sapply(x_t2, knn_prob_t2, x_train=x_t2, y_train=data_t2$y, K_val=K_val_plot)
#   lines(x_t2, probs_knn_plot, lty=K_val_plot+1, col=K_val_plot+1)
# }
# lines(x_t2, probs_m1_t2, col="blue", lty=1) # Logistic
# legend("bottomright", legend=c("Logistic", "KNN K=1", "KNN K=3", "KNN K=5"),
#        col=c("blue", 2, 3, 4), lty=c(1,2,3,4))
\end{lstlisting}
            \paragraph{1e: Confusion matrices for logistic and KNN (K=3)}
\begin{lstlisting}
# Logistic predictions (threshold 0.5)
# pred_log_t2e <- ifelse(probs_m1_t2 > 0.5, 1, 0)
# table(Actual=as.numeric(data_t2$y)-1, Predicted=pred_log_t2e)
# prop.table(table(Actual=as.numeric(data_t2$y)-1, Predicted=pred_log_t2e), margin=1)

# KNN K=3 predictions
# probs_knn3_t2e <- sapply(x_t2, knn_prob_t2, x_train=x_t2, y_train=data_t2$y, K_val=3)
# pred_knn3_t2e <- ifelse(probs_knn3_t2e > 0.5, 1, 0)
# table(Actual=as.numeric(data_t2$y)-1, Predicted=pred_knn3_t2e)
# prop.table(table(Actual=as.numeric(data_t2$y)-1, Predicted=pred_knn3_t2e), margin=1)
# Solution indicates identical predictions for this small dataset.
\end{lstlisting}

        \subsubsection{Task 3: Logistic Regression vs LDA (\Robject{Default} data)}
            \paragraph{3a: Fit Logistic and LDA, 50/50 split}
\begin{lstlisting}
# library(ISLR); library(MASS)
# data(Default)
# Default$default_numeric <- ifelse(Default$default == "Yes", 1, 0) % \textit{Make numeric for comparison} %

# set.seed(1) % (Or a seed specified in tutorial if different) %
# n_def <- nrow(Default)
# train_idx_def <- sample(1:n_def, floor(n_def/2))
# train_def <- Default[train_idx_def, ]
# test_def  <- Default[-train_idx_def, ]

# logreg_def <- glm(default ~ balance + income + student, data=train_def, family="binomial")
# lda_def <- lda(default ~ balance + income + student, data=train_def)
#
# prob_logreg_def_test <- predict(logreg_def, newdata=test_def, type="response")
# prob_lda_def_test <- predict(lda_def, newdata=test_def)$posterior[, "Yes"] % \textit{Or correct class index} %
# head(data.frame(LogRegProb=prob_logreg_def_test, LDAProb=prob_lda_def_test))
\end{lstlisting}
            \paragraph{3b: Predict test data, confusion matrices. Remove variables?}
\begin{lstlisting}
# threshold <- 0.5
# pred_log_class <- ifelse(prob_logreg_def_test > threshold, "Yes", "No")
# conf_log <- table(Actual=test_def$default, Predicted=pred_log_class)
# print("Logistic Confusion Matrix:"); print(conf_log); print(prop.table(conf_log,1))
# 
# pred_lda_class <- ifelse(prob_lda_def_test > threshold, "Yes", "No")
# conf_lda <- table(Actual=test_def$default, Predicted=pred_lda_class)
# print("LDA Confusion Matrix:"); print(conf_lda); print(prop.table(conf_lda,1))
# Solution note: Both good at Y=0, not Y=1. Lowering threshold might help.
# Removing variables: Try fitting models without 'student' or 'income' if they were
# not significant or if LDA assumptions are violated by them. Compare results.
\end{lstlisting}
            \paragraph{3c: Suitability of predictors for LDA. Other methods?}
            \textit{Question: Are student, balance, income all suitable for LDA? Other methods?}
            \textit{Solution Notes:
            - LDA assumes normality of predictors within classes and equal covariance.
            - `balance` and `income` (continuous) might be reasonably normal. `student` (binary factor) violates normality.
            - If normality/equal covariance is strongly violated, Logistic Regression is more robust.
            - QDA could be better if covariance matrices differ between default/non-default groups.
            - Non-parametric methods like KNN or Trees don't make these distributional assumptions.
            }

        \subsubsection{Task 4 : Cross-Validation Methods (\Robject{Auto} data)}
            \textit{This task explains data prep for \Robject{Auto} (create binary `y` from `mpg`, `age` from `year`, remove `mpg,name,year`) and then applies validation set, LOOCV, and k-fold CV to a logistic regression model for `y`.}
\begin{lstlisting}[caption={CV Methods Example (Ex Set 2, T4)}]
# library(ISLR)
# Auto_t4 <- Auto
# Auto_t4$y <- as.factor(ifelse(Auto_t4$mpg > median(Auto_t4$mpg), "high", "low"))
# Auto_t4$age <- 83 - Auto_t4$year
# Auto_t4 <- Auto_t4[, !(names(Auto_t4) %in% c("mpg", "name", "year"))]

# (b) Validation set (50/50 split)
# set.seed(1) % (Assume a seed) %
# n_auto_t4 <- nrow(Auto_t4)
# train_idx_t4 <- sample(1:n_auto_t4, floor(n_auto_t4/2))
# train_auto_t4 <- Auto_t4[train_idx_t4, ]
# test_auto_t4  <- Auto_t4[-train_idx_t4, ]
# m1_t4 <- glm(y ~ ., data=train_auto_t4, family="binomial")
# prob_t4_val <- predict(m1_t4, newdata=test_auto_t4, type="response")
# pred_t4_val <- ifelse(prob_t4_val > 0.5, "high", "low")
# # table(test_auto_t4$y, pred_t4_val); mean(pred_t4_val != test_auto_t4$y)

# (d) LOOCV (Manual loop shown in exercise, or use boot::cv.glm)
# library(boot)
# m_all_t4 <- glm(y ~ ., data=Auto_t4, family="binomial")
# cv_error_loo_t4 <- cv.glm(Auto_t4, m_all_t4, K=nrow(Auto_t4))$delta[1] % (Requires cost function for error rate) %
# Manual loop from exercise set:
# error_rate_LOO_t4 <- mean(pred_LOO_t4_class != Auto_t4$y) 
# % (Where pred_LOO_t4_class is from the loop. Solution gets 0.089 (high=0.923, low=0.117 error prop)) %

# (e) 8-fold CV (Manual loop shown in exercise, or use boot::cv.glm with K=8)
# error_rate_k8_t4 <- mean(pred_k8_t4_class != Auto_t4$y)
# % (Solution gets 0.094 (high=0.903, low=0.117 error prop)) %

# (f) Preference for predictors: All vs. some (e.g., age + weight)
# Fit LOOCV with reduced model: y ~ age + weight
# Compare LOOCV error rates. Solution: Reduced model was worse.
\end{lstlisting}

    \subsection{Tutorial 3 }
        \subsubsection{Task 1: Shrinkage Methods (\Robject{College} data)}
            \textit{Predict \Rcode{Apps} using \Rcode{Accept, Top10perc, Expend}. Compare OLS, Ridge, Lasso.}
            \paragraph{1a: Compare OLS, Ridge, Lasso for different $\lambda$s}
\begin{lstlisting}
# library(ISLR2); library(glmnet)
# xdata_t3 <- subset(College, select=c("Apps", "Accept", "Top10perc", "Expend"))
# x_mat_t3 <- as.matrix(xdata_t3[,-1])
# y_vec_t3 <- xdata_t3[,1]

# OLS
# ols_t3 <- lm(Apps ~ Accept + Top10perc + Expend, data=xdata_t3)
# coef_ols_t3 <- coef(ols_t3)

# Ridge (example lambda=100)
# ridge_t3_100 <- glmnet(x_mat_t3, y_vec_t3, alpha=0, lambda=100)
# coef_ridge_t3_100 <- coef(ridge_t3_100)
# % (Repeat for other lambdas) %

# Lasso (example lambda=100)
# lasso_t3_100 <- glmnet(x_mat_t3, y_vec_t3, alpha=1, lambda=100)
# coef_lasso_t3_100 <- coef(lasso_t3_100)
# % (Repeat for other lambdas. Compare how coefficients shrink/go to zero) %
\end{lstlisting}
            \paragraph{1b: Optimal $\lambda$ using \Rfunction{cv.glmnet}}
                \textit{Explain what \Rfunction{cv.glmnet} does: Performs k-fold CV (default 10) over a grid of $\lambda$ values to find $\lambda$ that minimizes CV error (e.g., MSE).}
\begin{lstlisting}[caption={Optimal Lambda (Ex Set 3, T1b)}]
# cv_ridge_t3 <- cv.glmnet(x_mat_t3, y_vec_t3, alpha=0)
# lambda_ridge_opt_t3 <- cv_ridge_t3$lambda.min % Sol: 364.8993 %
# cv_lasso_t3 <- cv.glmnet(x_mat_t3, y_vec_t3, alpha=1)
# lambda_lasso_opt_t3 <- cv_lasso_t3$lambda.min
\end{lstlisting}
            \paragraph{1c: Re-estimate with optimal $\lambda$, compare coefficients}
\begin{lstlisting}
# ridge_opt_t3 <- glmnet(x_mat_t3, y_vec_t3, alpha=0, lambda=lambda_ridge_opt_t3)
# lasso_opt_t3 <- glmnet(x_mat_t3, y_vec_t3, alpha=1, lambda=lambda_lasso_opt_t3)
# print(cbind(OLS=coef_ols_t3, Ridge=coef(ridge_opt_t3), Lasso=coef(lasso_opt_t3)))
\end{lstlisting}
            \paragraph{1d: Evaluate OLS, Ridge, Lasso by testMSE (validation set)}
\begin{lstlisting}
# n_t3 <- nrow(xdata_t3)
# set.seed(1) % (Assume a seed) %
# ind_t3 <- sample(1:n_t3, size=floor(n_t3/2))
# train_t3 <- xdata_t3[ind_t3,]; test_t3 <- xdata_t3[-ind_t3,]
# Xtrain_t3 <- as.matrix(train_t3[,-1]); ytrain_t3 <- train_t3[,1]
# Xtest_t3  <- as.matrix(test_t3[,-1]);  ytest_t3  <- test_t3[,1]

# OLS_train_t3 <- lm(Apps ~ ., data=train_t3)
# pred_ols_t3 <- predict(OLS_train_t3, newdata=test_t3)
# mse_ols_t3 <- mean((ytest_t3 - pred_ols_t3)^2) % Sol: 1293261 %

# Ridge (using lambda_ridge_opt_t3 from full data CV for simplicity, ideally re-CV on train_t3)
# ridge_train_t3 <- glmnet(Xtrain_t3, ytrain_t3, alpha=0, lambda=lambda_ridge_opt_t3)
# pred_ridge_t3 <- predict(ridge_train_t3, newx=Xtest_t3)
# mse_ridge_t3 <- mean((ytest_t3 - pred_ridge_t3)^2) % Sol: 1399610 %
# % (LASSO similarly. Solution: OLS better than Ridge here.) %
\end{lstlisting}

        \subsubsection{Task 2: Non-linear Models (Small Dataset)}
            \textit{Polynomial regression vs. regression splines.}
            \paragraph{2a: Plot y against x}
            \paragraph{2b: Polynomial regression K=7, plot predictions}
\begin{lstlisting}
# x_t3_task2 <- 1:8
# y_t3_task2 <- c(8.88, 4.54, 5.12, 1.14, 2.26, 8.43, 10.92, 14.47)
# data_t3_task2 <- data.frame(x=x_t3_task2, y=y_t3_task2)
# poly7_t3 <- lm(y ~ poly(x, 7, raw=TRUE), data=data_t3_task2) % (raw=TRUE for x, x^2...) %
# plot(data_t3_task2$x, data_t3_task2$y)
# lines(data_t3_task2$x, predict(poly7_t3), col="red") % (Perfect fit as N=P) %
\end{lstlisting}
            \paragraph{2c: Regression spline (manual basis) with knots at 3, 6. Plot.}
\begin{lstlisting}
# h_func <- function(x, xi) pmax((x-xi)^3, 0)
# spline_manual_t3 <- lm(y ~ x + I(x^2) + I(x^3) + h_func(x,3) + h_func(x,6), data=data_t3_task2)
# lines(data_t3_task2$x, predict(spline_manual_t3), col="blue")
# % (Expect worse predictions for new x than poly7, especially outside [1,8]) %
\end{lstlisting}
            \paragraph{2d: Regression spline with \Rfunction{bs} (knots 3,6). Compare preds.}
\begin{lstlisting}
# library(splines)
# spline_bs_t3 <- lm(y ~ bs(x, knots=c(3,6)), data=data_t3_task2)
# lines(data_t3_task2$x, predict(spline_bs_t3), col="green")
# % (Predictions should be identical to 2c if basis is equivalent, which it is for cubic) %
\end{lstlisting}
            \paragraph{2e: LOOCV for different knot placements}
\begin{lstlisting}
# testMSE_spline <- function(knot1, knot2, x_data, y_data) {
#   # ... (LOOCV loop as in earlier examples, fitting lm with bs(x, knots=c(knot1,knot2))) ...
#   return(mean_SE_from_LOOCV)
# }
# # testMSE_spline(3, 6, x_t3_task2, y_t3_task2) % Sol: 2044 %
# # testMSE_spline(3.5, 5.5, x_t3_task2, y_t3_task2) % Sol: 377 (better) %
\end{lstlisting}

        \subsubsection{Task 3: Extension of KNN (Nadaraya-Watson)}
            \textit{Implement KNN as weighted average, then Nadaraya-Watson.}
            \paragraph{3a: Properties of weights $w_i$} (Sum to 1, large $w_i$ for small $|x_i-x_0|$).
            \paragraph{3b: Implement KNN as weighted average (\Rcode{knn2})}
            \paragraph{3c: Modify knn2 to Nadaraya-Watson nw), plot for different $h$.}
\begin{lstlisting}
# weight_nw <- function(x0, x_vec, h_val=0.2) dnorm((x_vec-x0)/h_val) / sum(dnorm((x_vec-x0)/h_val))
# nw_func <- function(x0, x_vec, y_vec, h_val=0.2) {
#   w <- weight_nw(x0, x_vec, h_val)
#   return(sum(wy_vec))
# }
# # Simulate x_sim, y_sim from exercise
# # set.seed(12); x_sim <- seq(-50,50,length=100)
# # y_sim <- 1-0.05x_sim-0.02x_sim^2+0.003x_sim^3+50rnorm(100)
# # plot(x_sim, y_sim)
# # pred_nw_h1 <- sapply(x_sim, nw_func, x_vec=x_sim, y_vec=y_sim, h_val=1)
# # lines(x_sim, pred_nw_h1, col="red")
# # pred_nw_h5 <- sapply(x_sim, nw_func, x_vec=x_sim, y_vec=y_sim, h_val=5)
# # lines(x_sim, pred_nw_h5, col="blue")
\end{lstlisting}
            \paragraph{3d: Meaning of bandwidth $h$} (Controls "localness" of estimator).

    \subsection{Tutorial 4}
        \subsubsection{Task 1 : Regression Trees (\Robject{Hitters} data)}
            \textit{Predict \Rcode{logSalary} using \Rcode{Hits} and \Rcode{Years}.}
            \paragraph{1a: Explain RSS function and \Rfunction{optimise} for first split.}
\begin{lstlisting}
# library(ISLR); data(Hitters)
# Hitters <- Hitters[!is.na(Hitters$Salary),]
# Hitters$logSalary <- log(Hitters$Salary)
# RSS_func_t4 <- function(s, x1_vec, y_vec) {
#   yR1 <- mean(y_vec[x1_vec < s], na.rm=TRUE)
#   yR2 <- mean(y_vec[x1_vec >= s], na.rm=TRUE)
#   rss <- sum((y_vec[x1_vec < s] - yR1)^2, na.rm=TRUE) + 
#          sum((y_vec[x1_vec >= s] - yR2)^2, na.rm=TRUE)
#   return(rss)
# }
# opt_hits <- optimise(RSS_func_t4, lower=0, upper=250, 
#                      x1_vec=Hitters$Hits, y_vec=Hitters$logSalary) 
# % $min=117.5, $obj=160.97 %
# opt_years <- optimise(RSS_func_t4, lower=0, upper=50, 
#                       x1_vec=Hitters$Years, y_vec=Hitters$logSalary) 
# % $min=4.5, $obj=115.06 %
# # Years gives lower RSS, so first split is on Years < 4.5
\end{lstlisting}
            \paragraph{1b: Fit tree with \Rfunction{tree}, plot. Compare first split.}
\begin{lstlisting}
# library(tree)
# tree1_t4 <- tree(logSalary ~ Hits + Years, data=Hitters)
# plot(tree1_t4); text(tree1_t4, pretty=0)
# % First split in tree should be Years < 4.5 (or very close) %
\end{lstlisting}
            \paragraph{1c: Predict for new observation (Hits=125, Years=2.5)}
            \textit{Follow path in tree1\_t4: Years=2.5 (<4.5) -> Left. Hits=125 (>114 if that's next split) -> Right in subtree. Read leaf value. Sol: 5.264}
            \paragraph{1d: Fit tree with all predictors }
\begin{lstlisting}
# tree2_t4 <- tree(logSalary ~ . - Salary, data=Hitters) # Exclude original Salary
# summary(tree1_t4) % RSS is deviance: 69.06 for tree1 %
# summary(tree2_t4) % RSS is deviance: 43.03 for tree2 (lower as expected) %
# rss1_t4 <- sum(residuals(tree1_t4)^2)
# rss2_t4 <- sum(residuals(tree2_t4)^2)
\end{lstlisting}

        \subsubsection{Task 2: Bagging and Random Forest (\Robject{Hitters} data)}
            \textit{Benchmark: \Rcode{tree2} from T1d.}
            \paragraph{2a: TestMSE for \Rcode{tree2} (validation set)}
\begin{lstlisting}
# set.seed(123)
# n_hit <- nrow(Hitters)
# ind_hit <- sample(1:n_hit, size=floor(n_hit/2))
# train_hit <- Hitters[ind_hit,]; test_hit <- Hitters[-ind_hit,]
# tree2_train_t4 <- tree(logSalary ~ . - Salary, data=train_hit)
# pred2_test_t4 <- predict(tree2_train_t4, newdata=test_hit)
# mse2_test_t4 <- mean((test_hit$logSalary - pred2_test_t4)^2) % Sol: 0.428 %
\end{lstlisting}
            \paragraph{2b: Bagging of \Rcode{tree2}. Variable Importance.}
\begin{lstlisting}
# library(randomForest)
# set.seed(123) % (Ensure consistent seed for RF if comparing) %
# # For bagging, mtry = number of predictors (19 for Hitters excluding Salary, logSalary)
# bagging_t4 <- randomForest(logSalary ~ . - Salary, data=train_hit, mtry=19, importance=TRUE)
# predbag_t4 <- predict(bagging_t4, newdata=test_hit)
# msebag_t4 <- mean((test_hit$logSalary - predbag_t4)^2) % Sol: 0.3015 %
# varImpPlot(bagging_t4)
# % x-axis is MeanDecreaseGini (for classification) or IncNodePurity (RSS decrease for regression).
#    Help says for regression: total decrease in node impurities (RSS) from splitting on the variable,
#    averaged over all trees. CRuns most important. %
\end{lstlisting}
            \paragraph{2c: Random Forest. How is it an extension of bagging?}
\begin{lstlisting}
# library(randomForest)
# set.seed(123)
# rf_t4 <- randomForest(logSalary ~ . - Salary, data=train_hit, mtry=3, importance=TRUE) % (mtry=3 from solution) %
# predrf_t4 <- predict(rf_t4, newdata=test_hit)
# mserf_t4 <- mean((test_hit$logSalary - predrf_t4)^2)
# % Random Forest is an extension because at each split, only a random subset (mtry) of predictors
#    is considered, decorrelating trees and potentially improving variance reduction. %
\end{lstlisting}

        \subsubsection{Task 3: Boosting (\Robject{Hitters} data)}
            \textit{Predict \Rcode{logSalary} with boosted tree. Compute testMSE.}
\begin{lstlisting}
# library(gbm)
# # Using train_hit, test_hit from 2a
# set.seed(123)
# boost_t4 <- gbm(logSalary ~ . - Salary, data=train_hit, distribution="gaussian",
#                 n.trees=1000, interaction.depth=4, shrinkage=0.01)
# predgbm_t4 <- predict(boost_t4, newdata=test_hit, n.trees=1000)
# msegbm_t4 <- mean((test_hit$logSalary - predgbm_t4)^2) % Sol: 0.2666 %
# % Boosting often yields lowest MSE among tree ensembles here. %
\end{lstlisting}

        \subsubsection{Task 4: Support Vector Machines (Simulated Data `xy.csv`)}
            \textit{Classes "1" and "-1".}
            \paragraph{4a: Possible to find maximal marginal classifier from plot?} (No, classes overlap).
            \paragraph{4b: Recreate plot from `xy.csv`.}
            \paragraph{4c: Use SVC (linear kernel) to classify. Plot.}
\begin{lstlisting}
# xy_data_t4 <- read.csv("xy.csv") 
# names(xy_data_t4) <- c("x1", "x2", "y_class") % (Assuming column names) %
# xy_data_t4$y_factor <- as.factor(xy_data_t4$y_class)
# library(e1071)
# svmfit_linear_t4 <- svm(y_factor ~ x1 + x2, data=xy_data_t4, kernel="linear", 
#                         cost=10, scale=FALSE)
# plot(svmfit_linear_t4, xy_data_t4, x2 ~ x1) % (Adjust formula for plot if needed) %
\end{lstlisting}
            \paragraph{4d: Meaning of tuning parameter C=10.} (Allows for some misclassifications/margin violations, controls trade-off between margin width and violations. Larger C = narrower margin, fewer violations).
            \paragraph{4e: SVM with polynomial kernel. Better separation?}
\begin{lstlisting}
# svmfit_poly_t4 <- svm(y_factor ~ x1 + x2, data=xy_data_t4, kernel="polynomial", 
#                       cost=10, scale=FALSE) % (Default degree=3) %
# plot(svmfit_poly_t4, xy_data_t4, x2 ~ x1)
# % Polynomial kernel can create non-linear boundary, might separate training data better,
#    but risk of overfitting. %
\end{lstlisting}

        \subsubsection{Task 5: PCA (\Robject{advertising.csv})}
            \textit{PCA on TV, radio, newspaper. Interpret first two PCs from loadings.}
\begin{lstlisting}
# advertising_t5 <- read.csv("advertising.csv")[, c("TV", "radio", "newspaper")]
# pca_t5 <- prcomp(advertising_t5, scale.=TRUE, center.=TRUE)
# summary(pca_t5)
# print(pca_t5$rotation)
# % PC1: High loading on TV (-0.999), small on others. Represents overall ad spend dominated by TV.
#    PC2: High positive on newspaper (0.93), moderate positive on radio (0.36), small on TV. Represents
#    print media focus vs. TV. (Signs might be flipped but interpretation similar). %
\end{lstlisting}

        \subsubsection{Task 6: Cluster Analysis (\Robject{USArrests} data)}
            \paragraph{6a: K-means clustering (K=3) on Murder, Assault. Plot.}
\begin{lstlisting}
# data(USArrests)
# x_km_t6 <- scale(USArrests[, c("Murder", "Assault")]) # Scale data
# set.seed(1) % (For reproducibility of kmeans) %
# km3_t6 <- kmeans(x_km_t6, centers=3, nstart=20)
# plot(x_km_t6, col=km3_t6$cluster, pch=19, main="K-Means (K=3) on Murder & Assault")
# points(km3_t6$centers, col=1:3, pch=4, cex=3, lwd=3)
# text(x_km_t6, labels=rownames(USArrests), col=km3_t6$cluster, pos=3, cex=0.7)
\end{lstlisting}
            \paragraph{6b: Hierarchical clustering. Compare.}
\begin{lstlisting}
# Using scaled data x_km_t6 from above
# dist_t6 <- dist(x_km_t6)
# hclust_complete_t6 <- hclust(dist_t6, method="complete")
# plot(hclust_complete_t6, main="Hierarchical Clustering (Complete Linkage)", cex=0.7)
# clusters_hclust_k3_t6 <- cutree(hclust_complete_t6, k=3)
# table(KMeans=km3_t6$cluster, Hierarchical=clusters_hclust_k3_t6)
# % Compare cluster assignments. Differences due to different algorithms/criteria. %
\end{lstlisting}

\section{Assignment: Compulsory Assignment BAN404 (Spring 2025x)}
\textit{Predicting log(CEO Salary) using the \Rcode{ceosal2} dataset.}
\vspace{0.5em}

    \subsection{Initial Data Loading and Preparation}
        \paragraph{Loading Packages and Data}
\begin{lstlisting}
# Ensure wooldridge is installed: install.packages("wooldridge")
library(wooldridge)
library(dplyr)     # For tidyverse data manipulation
library(ggplot2)   # For tidyverse plotting
library(glmnet)    # For LASSO
library(FNN)       # For knn() function (alternative to manual for Task 3)
library(gam)       # For GAMs
library(boot)      # For cv.glm (though LOOCV can be manual for lm)

data(ceosal2)
# View data structure and summary
# str(ceosal2)
# summary(ceosal2)
# help(ceosal2) # To understand variables
\end{lstlisting}

        \paragraph{Data Splitting (50/50 Train/Test)}
            \textit{As per the assignment's suggested solution, an initial 50/50 split is performed.}
\begin{lstlisting}[caption={Splitting data into 50/50 training and test sets}]
set.seed(1234) # Seed from assignment
n_total <- nrow(ceosal2)
ntrain <- floor(n_total/2)
train_indices <- sample(1:n_total, size=ntrain)

# These original splits might be used if subsequent processing is done per task
# For consistency with assignment solution, a global processed dataset is often made first
original_train_data <- ceosal2[train_indices,]
original_test_data <- ceosal2[-train_indices,]
\end{lstlisting}

    \subsection{Task 1: Descriptive Statistics and Predictor Investigation}
        \paragraph{Question}
        \textit{Describe relevant features of the output variable (\Rcode{lsalary}) with descriptive statistics (tables and graphs). Also use descriptive statistics to investigate promising predictors for \Rcode{lsalary}. Remove variables that cannot possibly be available when the salary of a new CEO should be predicted.}

        \paragraph{Solution}
            \subparagraph{Processing Data for Task 1}
                \textit{Assumptions based on provided solution: New CEO hired internally. \Rcode{comten} (company tenure) and \Rcode{comtensq} are available. \Rcode{ceoten} (CEO tenure in current company) and \Rcode{ceotensq} are NOT available for a new CEO. \Rcode{salary} (untransformed) is obviously not available. \Rcode{college} is removed due to low variability in the training data example in the solution.}
\begin{lstlisting}[caption={Data processing for Task 1 (Applied to whole dataset first, then split)}]
# Create the 'lsalary' variable if it's not already present (it is in ceosal2)
# ceosal2$lsalary is log(salary)

# Variables to remove
vars_to_remove_t1 <- c("salary", "ceoten", "ceotensq", "college")
ceosal_task1_filtered <- ceosal2[, !(names(ceosal2) %in% vars_to_remove_t1)]

# Convert 'grad' to factor
ceosal_task1_filtered$grad <- as.factor(ceosal_task1_filtered$grad)

# Now split the PROCESSED data
train_t1 <- ceosal_task1_filtered[train_indices,]
test_t1  <- ceosal_task1_filtered[-train_indices,]
\end{lstlisting}

            \subparagraph{Descriptive Statistics for \Rcode{lsalary} (on \Rcode{train\_t1})}
\begin{lstlisting}
# Base R
summary(train_t1$lsalary)
hist(train_t1$lsalary, main="Histogram of log(Salary) - Training Data", xlab="log(Salary)", col="lightblue", breaks=12)
boxplot(train_t1$lsalary, main="Boxplot of log(Salary) - Training Data", ylab="log(Salary)")

# Tidyverse
# library(dplyr); library(ggplot2)
# train_t1 %>% select(lsalary) %>% summary()
# ggplot(train_t1, aes(x=lsalary)) + 
#   geom_histogram(aes(y=..density..), fill="lightblue", color="black", bins=12) + 
#   geom_density(alpha=0.2, fill="red") +
#   ggtitle("Distribution of log(Salary) - Training Data")
\end{lstlisting}
            \textit{Interpretation: Comment on the distribution of \Rcode{lsalary} - is it symmetric, skewed, any outliers? The log transformation often makes salary data more symmetric.}

            \subparagraph{Investigating Promising Predictors for \Rcode{lsalary} (on \Rcode{train\_t1})}
\begin{lstlisting}
# Base R (as in solution)
# par(mfrow=c(3,3)) # Adjust layout based on number of predictors remaining
# plot(lsalary ~ ., data=train_t1) 
# graphics.off()

# Tidyverse (example for a few predictors)
# ggplot(train_t1, aes(x=lsales, y=lsalary)) + geom_point() + geom_smooth(method="lm") + ggtitle("lsalary vs lsales")
# ggplot(train_t1, aes(x=grad, y=lsalary)) + geom_boxplot() + ggtitle("lsalary vs grad")
\end{lstlisting}
            \textit{Interpretation of plots:
            - \Rcode{lsales} shows a clear positive linear relationship.
            - \Rcode{sales} also positive, might capture non-linearity if \Rcode{lsales} also included.
            - \Rcode{mktval}, \Rcode{lmktval} show positive trends.
            - For categorical \Rcode{grad}: Check if means/medians differ significantly.
            }
\begin{lstlisting}[caption={Statistical test for 'grad' (Task 1)}]
# Base R
# by(train_t1$lsalary, train_t1$grad, summary)
# t_test_grad_0 <- t.test(train_t1$lsalary[train_t1$grad == 0]) # If enough observations
# t_test_grad_1 <- t.test(train_t1$lsalary[train_t1$grad == 1])
# print(t_test_grad_0$conf.int)
# print(t_test_grad_1$conf.int)
# if (abs(t_test_grad_0$estimate - t_test_grad_1$estimate) is small and CIs overlap substantially,
# then 'grad' might not be a strong predictor alone).

# Tidyverse
# train_t1 %>% group_by(grad) %>% summarise(mean_lsalary = mean(lsalary), sd_lsalary = sd(lsalary))
\end{lstlisting}
            \textit{Conclusion from Task 1 based on solution: \Rcode{lsales} is the most promising. Others warrant investigation in modeling. \Rcode{grad} seems weak.}

    \subsection{Task 2: Linear Regression and LASSO}
        \paragraph{Question}
        \textit{Use linear regression (on all or a subset of the predictors) and LASSO to predict \Rcode{lsalary}. Evaluate the predictions on test data using an appropriate error measure (MSE).}
        \paragraph{Solution}
            \textit{We use \Rcode{train\_t1} and \Rcode{test\_t1} from the end of Task 1 preparation.}
            \subparagraph{OLS with All Predictors}
\begin{lstlisting}
ols_model_all <- lm(lsalary ~ ., data=train_t1)
# summary(ols_model_all) # For coefficient details

ols_all_preds_test <- predict(ols_model_all, newdata=test_t1)
ols_all_mse_test <- mean((test_t1$lsalary - ols_all_preds_test)^2)
cat("Test MSE (OLS All Predictors):", ols_all_mse_test, "\n")
# Solution had Test MSE = 0.3044908
\end{lstlisting}
            \subparagraph{OLS with \Rcode{lsales} Only}
\begin{lstlisting}
ols_model_lsales <- lm(lsalary ~ lsales, data=train_t1)
# summary(ols_model_lsales)

ols_lsales_preds_test <- predict(ols_model_lsales, newdata=test_t1)
ols_lsales_mse_test <- mean((test_t1$lsalary - ols_lsales_preds_test)^2)
cat("Test MSE (OLS lsales Only):", ols_lsales_mse_test, "\n")
# Solution had Test MSE = 0.309084
\end{lstlisting}
            \textit{Comment: The test MSEs are very similar, suggesting \Rcode{lsales} captures most of the predictive power. The solution mentions LOOCV for a more robust comparison, where the \Rcode{lsales}-only model was better.}

            \subparagraph{LASSO Regression}
\begin{lstlisting}
library(glmnet)
# Prepare matrices for glmnet (factors need to be dummified)
# 'grad' is the only factor in train_t1/test_t1
x_train_lasso_t2 <- model.matrix(lsalary ~ . , data=train_t1)[,-1] # -1 to remove intercept column
y_train_lasso_t2 <- train_t1$lsalary
x_test_lasso_t2  <- model.matrix(lsalary ~ . , data=test_t1)[,-1]

set.seed(1234) # For cv.glmnet
cv_lasso_t2 <- cv.glmnet(x_train_lasso_t2, y_train_lasso_t2, alpha=1, standardize=TRUE)
best_lambda_t2 <- cv_lasso_t2$lambda.min
# print(paste("Best lambda for LASSO:", best_lambda_t2))

lasso_model_t2 <- glmnet(x_train_lasso_t2, y_train_lasso_t2, alpha=1, 
                         lambda=best_lambda_t2, standardize=TRUE)
# print("LASSO Coefficients:")
# print(coef(lasso_model_t2)) 
# Solution notes its LASSO (on its training split) picked lsales and lmktval.

lasso_preds_test_t2 <- predict(lasso_model_t2, newx=x_test_lasso_t2)
lasso_mse_test_t2 <- mean((test_t1$lsalary - lasso_preds_test_t2)^2)
cat("Test MSE (LASSO):", lasso_mse_test_t2, "\n")
# Solution LOOCV MSE for LASSO was 0.3777, worse than lsales-only OLS.
# Compare your test MSE here with the OLS models.
\end{lstlisting}

    \subsection{Task 3: KNN Regression}
        \paragraph{Question}
        \textit{Predict \Rcode{lsalary} by KNN, using only \Rcode{lsales} as a predictor. Determine $K$ using leave-one-out cross-validation. Also, evaluate the predictions and compare with the results in 2.}
        \paragraph{Solution}
            \textit{The assignment solution uses LOOCV on the full processed dataset (\Rcode{ceosal\_final\_filtered}) to determine K. For strictness, K should be determined using only the training portion (\Rcode{train\_t1}) or an inner CV loop if the final evaluation is on \Rcode{test\_t1}. We follow the provided solution's approach for K determination for comparability with its results.}
            \subparagraph{Determine Optimal K using LOOCV}
\begin{lstlisting}
# Using ceosal_final_filtered from Task 1 for LOOCV to find K
# (This uses data that will also be in the "test" part of the original 50/50 split)
knn_reg_func_t3 <- function(x0, x_train_vec, y_train_vec, K_val) {
  d <- abs(x_train_vec - x0)
  o <- order(d)
  # Ensure K_val is not larger than the number of available neighbors
  K_actual <- min(K_val, length(o))
  return(mean(y_train_vec[o[1:K_actual]]))
}

lsales_all_t3 <- ceosal_task1_filtered$lsales % \textit{Using the dataset after initial processing} %
lsalary_all_t3 <- ceosal_task1_filtered$lsalary
n_all_t3 <- nrow(ceosal_task1_filtered)
k_max_t3 <- 40 # As per solution
loocv_mse_k_t3 <- numeric(k_max_t3 - 1) 

for (k_val_t3 in 2:k_max_t3) {
  se_fold_t3 <- numeric(n_all_t3)
  for (i in 1:n_all_t3) {
    xtrain_knn_t3 <- lsales_all_t3[-i]
    ytrain_knn_t3 <- lsalary_all_t3[-i]
    xtest_knn_t3  <- lsales_all_t3[i]
    ytest_knn_t3  <- lsalary_all_t3[i]
    
    pred_knn_fold_t3 <- knn_reg_func_t3(xtest_knn_t3, xtrain_knn_t3, ytrain_knn_t3, K_val=k_val_t3)
    se_fold_t3[i] <- (ytest_knn_t3 - pred_knn_fold_t3)^2
  }
  loocv_mse_k_t3[k_val_t3-1] <- mean(se_fold_t3)
}

# plot(2:k_max_t3, loocv_mse_k_t3, type="b", xlab="K", ylab="LOOCV MSE", main="LOOCV for KNN K selection")
optimal_K_t3 <- (2:k_max_t3)[which.min(loocv_mse_k_t3)]
# cat("Optimal K from LOOCV on full processed data:", optimal_K_t3, "\n") 
# Solution found K=24 (index 23 in their 2:kmax loop).
min_loocv_mse_knn_t3 <- min(loocv_mse_k_t3)
# cat("LOOCV MSE for KNN with K=", optimal_K_t3, ":", min_loocv_mse_knn_t3, "\n")
# Solution result for K=24: 0.2840139.
\end{lstlisting}
            \subparagraph{Comparison with Task 2 Results}
            \textit{
            LOOCV MSE for KNN (K=[optimal\_K\_t3]): [min\_loocv\_mse\_knn\_t3]
            From solution's LOOCV for Task 2:
            - OLS (lsales only) LOOCV MSE: 0.2685442
            - LASSO LOOCV MSE: 0.3777567
            - OLS (all predictors) LOOCV MSE: 0.4253121
            Conclusion: KNN with only \Rcode{lsales} (optimal K) performs better than LASSO and full OLS in terms of LOOCV MSE, but slightly worse than the simple OLS model using only \Rcode{lsales}.
            }

    \subsection{Task 4: KNN with Multiple Predictors and GAM}
        \paragraph{Question (Part 1 - KNN with Multiple Predictors)}
        \textit{Explain how you would use KNN if you had more than one predictor. Two alternative ways to do this have been mentioned in the course. If you would like to you can also write R-code to do this but this is not required to pass the assignment.}
        \paragraph{Solution (Textual)}
        \textit{(Same textual explanation as in Hypothetical Exam 2025 for: 1. Multivariate Distance Metric for KNN, 2. GAM Framework with KNN as Univariate Smoother via Backfitting).}

        \paragraph{Question (Part 2 - GAM)}
        \textit{Argue for which, if any, variables might have a nonlinear relationship with \Rcode{lsalary} and find a good model within the generalized additive models (GAM) framework. Evaluate the predictions.}
        \paragraph{Solution (using \Rpackage{gam})}
            \textit{From Task 1, plots suggested \Rcode{sales} and \Rcode{mktval} might have non-linear relationships. The solution investigates these using LOOCV.}
\begin{lstlisting}[caption={GAM for lsalary (Task 4)}]
library(gam)
# Using ceosal_final_train, ceosal_final_test (or ceosal_task1_filtered for LOOCV as in solution)

# Visual inspection on training data (as in solution)
# gam_inspect_t4 <- gam(lsalary ~ s(sales, df=4) + s(mktval, df=4) + lsales + lmktval + 
#                                 age + grad + comten + profits + comtensq + profmarg, 
#                       data=train_t1) % \textit{(Using train_t1, consistent with solution's Task 1 plots)} %
# par(mfrow=c(1,2))
# plot(gam_inspect_t4, se=TRUE, terms=c("s(sales, df = 4)", "s(mktval, df = 4)"))
# graphics.off()
# Solution comment based on this: "contribution of non-linear mktval looks questionable".

# LOOCV for GAM with s(sales) and other predictors linearly.
# (Using the full processed dataset ceosal_task1_filtered for LOOCV as per solution)
n_gam_t4 <- nrow(ceosal_task1_filtered)
se_gam_s_sales_t4 <- numeric(n_gam_t4)

# Formula based on solution: only s(sales) is non-linear, others linear.
# (The solution eventually only uses s(sales) in its final GAM LOOCV comparison)
formula_gam_s_sales <- lsalary ~ s(sales, df=4) 
# To compare with other models, it's better to include other linear terms if they were in OLS.
# For this example, let's test the simple s(sales) model as in the solution's final GAM part.

# for(i in 1:n_gam_t4) {
#   train_gam_fold_t4 <- ceosal_task1_filtered[-i,]
#   test_gam_fold_t4  <- ceosal_task1_filtered[i,]
#   
#   # Ensure factor levels are handled if factors were included in formula_gam_s_sales
#   # For lsalary ~ s(sales, df=4), this is not an issue.
#   gam_fit_fold_t4 <- gam(formula_gam_s_sales, data=train_gam_fold_t4)
#   pred_gam_fold_t4 <- predict(gam_fit_fold_t4, newdata=test_gam_fold_t4)
#   se_gam_s_sales_t4[i] <- (test_gam_fold_t4$lsalary - pred_gam_fold_t4)^2
# }
# mse_gam_s_sales_loocv_t4 <- mean(se_gam_s_sales_t4)
# cat("LOOCV MSE (GAM with s(sales, df=4) only):", mse_gam_s_sales_loocv_t4, "\n")
# Solution result for GAM with s(sales) only: 0.2918272.

# Comparison:
# KNN (K=24, lsales only) LOOCV MSE: 0.2840139
# OLS (lsales only) LOOCV MSE: 0.2685442
# Conclusion: "GAM with s(sales) is slightly worse than KNN with lsales, and also worse than
# simple OLS with lsales. The log-transformation in lsales seems to capture the main
# non-linearity effectively for this dataset."
\end{lstlisting}

\end{document}
